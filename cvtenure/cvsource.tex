\begin{center}
  {\LARGE {\bf Erin Sheldon}}
  {\large {\it Curriculum Vitae}}
\end{center}

%
% Address information.
%


\noindent
{Bldg 510} \hfill {Office: (631) 344-3117}

\noindent 
{Brookhaven National Laboratory} \hfill {Fax: (631) 344-7142}

\noindent
{PO BOX 5000} \hfill {erin.sheldon@gmail.com}

\noindent
{Upton, NY 11973-5000} %\hfill {erin.sheldon@gmail.com}


% old style
%\noindent
%{Home: 58-12 Queens Blvd.,  \#8-N} \hfill {Work: Bldg 510}

%\noindent
%{Woodside, NY 11377} \hfill {Brookhaven National Laboratory}

%\noindent
%{(917) 209-7332} \hfill {Upton, NY 11973}

%\noindent
%{erin.sheldon@gmail.com} \hfill {(631) 344-3117}

%\noindent
%{http://cheops1.uchicago.edu/esheldon} \hfill {Last Modified: \today}


\ssp
\ssp
\noindent
\makebox[1.25in][l]{{\large \bf Position}}
{{\bf Brookhaven National Laboratory} }

\noindent
\makebox[1.25in][l]{}
{Physicist \hfill {\small \it 2012--present}}


\noindent
\makebox[1.25in][l]{}
{Associate Physicist \hfill {\small \it 2010-2012}}

\noindent
\makebox[1.25in][l]{}
{Assistant Physicist \hfill {\small \it 2008-2010}}



\ssp
\ssp
\noindent
\makebox[1.25in][l]{{\large \bf Education}}
{\bf University of Michigan}
\hfill
\makebox[1in][r]{\small \it September 1997--August 2002}
\normalsize

\tsp
\noindent
\makebox[1.25in][l]{}
\parbox{5.40in}{
Ph.D., Physics\newline
Thesis advisor: Prof. Timothy McKay\newline
Title of thesis: ``Galaxies, Luminosity, and Mass: Gravitational Lensing Measurements of the Correlation between Dark and Luminous Matter''
}

\isp
\noindent
\makebox[1.25in][l]{}
{\bf University of Missouri}
\hfill
\makebox[1in][r]{\small \it Aug 1992 -- May 1997}
\normalsize

\tsp
\noindent
\makebox[1.25in][l]{}
\parbox{5.40in}{B.S. {\it Magna cum laude}, Physics and Mathematics}

%
% Experience...
%

\ssp
\ssp
\noindent
\makebox[1.25in][l]{\large \bf Research}
	{Postdoctoral Fellow, Center for Cosmology and Particle Physics}
        \newline
\makebox[1.25in][l]{}{~~~~~~~~~~~~~~~~~~~~~~~~~~~~~New York University}
        \hfill
        \makebox[1in][r]{{\small \it Sep 2005--Sep 2008}}
\newline
\makebox[1.25in][l]{}
	{Postdoctoral Fellow, Kavli Institute for Cosmological Physics}
	\newline
\makebox[1.25in][l]{}{~~~~~~~~~~~~~~~~~~~~~~~~~~~~~University of Chicago}
	\hfill
	\makebox[1in][r]{{\small \it Aug 2002--Aug 2005}}
\newline
\makebox[1.25in][l]{}
	{{Research Assistant, University of Michigan}}
	\hfill
	\makebox[1in][r]{\small \it Sep 1998--Aug 2002}
\newline
\makebox[1.25in][l]{}
	{Research Assistant, Fermilab}
	\hfill
\makebox[1in][r]{\small \it Jun 1998--Sep 1998}
\newline
\makebox[1.25in][l]{}
	{Hughes Fellow, University of Missouri}
	\hfill
	\makebox[1in][r]{\small \it May 1996--Sep 1997}
\newline
\makebox[1.25in][l]{}
	{A\&S Research Fellow, University of Missouri}
	\hfill
	\makebox[1in][r]{\small \it Summer 1995}
\newline
\makebox[1.25in][l]{}
	{Hughes Fellow, University of Missouri}
	\hfill
	\makebox[1in][r]{\small \it Fall 1994--Spr 1995}

\ssp
\ssp
\noindent
\parbox[l]{1.25in}{{\bf Awards}}
\parbox[t]{5.40in}{
KICP Postdoctoral Fellowship \hfill {\small \it 2002-05} \newline
Michigan Space Grant Consortium grant  \hfill 
       {\small \it 2000-01 and 01-02 }\newline
%Hughes Undergraduate Summer Research Fellowship, University of
%       Missouri \hfill {\small \it 1995-96} \newline 
A\&S Summer Research Fellowship, University of Missouri \hfill 
       {\small \it Summer 1995}\newline
Hughes Undergraduate Research Fellowship, University of
       Missouri  \hfill {\small \it 1994-95, 1995-96}\newline
O.M. Stewart Physics Scholarship, University
       of Missouri \hfill {\small \it 1993-97}\newline
Curators Scholarship, University of Missouri 
       (Full Tuition) \hfill {\small \it 1992-97}
}

\ssp
\ssp
\noindent
\parbox[l]{1.25in}{{\bf Leadership Roles}}
\parbox[t]{5.40in}{

Co-convener {\it DES Shear Pipeline Development and Testing Group} \hfill {\small \it 2014-} \newline

Coordinator, {\it BOSS Target Selection (Galaxies, Quasars, Stars) } \hfill {\small \it 2009-2012} \newline

}
\ssp
\ssp
\noindent
\parbox[l]{1.25in}{{\bf Invited Talks}}
\parbox[t]{5.40in}{

Stanford Colloquium {\it Cluster Lensing Measurements with DES Data} \hfill {\small \it Jan 2015} \newline

U. Penn  {\it Cluster Lensing Results from Early DES} \hfill {\small \it Jan 2015} \newline

Mcgill University, declined {\it Early DES Lensing} \hfill {\small \it Fall 2014} \newline

Fermilab {\it Perfecting Weak Lensing Measurements} \hfill {\small \it Apr. 2014} \newline

American Physical Society, declined, {\it DES Lensing} \hfill {\small \it Apr. 2014} \newline

U. Michigan {\it  Cosmology from SDSS Galaxy Cluster Lensing} \hfill {\small \it 2010} \newline

Yale Colloquium {\it Dark and Luminous Matter in the Universe}     \hfill {\small \it 2008} \newline

U.C. Davis {\it Mass Profiles of Clusters in the SDSS from Weak Lensing} \hfill {\small \it 2007} \newline

Princeton {\it Mass-to-light Ratios of SDSS Clusters from Weak Lensing} \hfill {\small \it 2007} \newline

Ohio State U. {\it Mass-to-light Ratios of Clusters} \hfill {\small \it 2007} \newline

Great Lakes Cosmology Workshop, Ann Arbor Mi {\it Constraints on Bias from Weak Lensing in the SDSS} \hfill {\small \it 2005} \newline

NYU {\it Weak Lensing with SDSS Galaxy Clusters} \hfill {\small \it 2005} \newline

Fermilab, Fundamental Physics from Clusters of Galaxies {\it Cluster Masses from Lensing in the SDSS} \hfill {\small \it 2004} \newline

Lensing Workshop, OSU {\it Weak Lensing with SDSS Galaxies} \hfill {\small \it 2004} \newline

U.C. Berkeley {\it Weak Lensing with SDSS Galaxies} \hfill {\small \it 2004} \newline

AAS Special Session: Large Scale Structure and Galaxy Properties in the SDSS {\it Galaxy Biasing and Mass-to-light Ratios from Weak Lensing in the SDSS} \hfill {\small \it 2003} \newline

AAS Special Session: Large-scale Structure with the SDSS {\it Weak Lensing results for Galaxies and Clusters} \hfill {\small \it 2002} \newline

U. Pittsburgh, {\it Weak Lensing with SDSS Galaxies} \hfill {\small \it 2002} \newline

Fermilab, {\it Constraints on Bias from Weak Lensing in the SDSS} \hfill {\small \it 2002} \newline
}
