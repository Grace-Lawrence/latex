\documentclass[usegraphicx,usenatbib]{mn2e}
%\documentclass[useAMS,usegraphicx,usenatbib]{mn2e}

\usepackage{verbatim}
\usepackage{color}
\usepackage[normalem]{ulem} % for striking out with \sout
\usepackage{amsmath} % for boldsymbol
\usepackage{times}
\usepackage{mathabx}
\usepackage{enumitem}

% A comment block

%\newcommand{\comment}[1]{}

% For color
\newcommand{\mpname}[1]{#1_color.eps}
\newcommand{\clraitoff}{red}
\newcommand{\lumblack}{(black)}
\newcommand{\lumblue}{(blue)}
\newcommand{\lumred}{(red)}
\newcommand{\vdisred}{(red-dashed curve)}
\newcommand{\vdisblue}{(blue-solid curve)}

% For bw
%\newcommand{\mpname}[1]{#1.eps}
%\newcommand{\clraitoff}{}
%\newcommand{\lumblack}{}
%\newcommand{\lumblue}{}
%\newcommand{\lumred}{}
%\newcommand{\vdisred}{(dashed curve)}
%\newcommand{\vdisblue}{(solid curve)}

\newcommand{\umag}{$u$}
\newcommand{\gmag}{$g$}
\newcommand{\rmag}{$r$}
\newcommand{\imag}{$i$}
\newcommand{\zmag}{$z$}
\newcommand{\gmr}{$g-r$}



\newcommand{\gammat}{$\gamma_T$}
\newcommand{\gammacross}{$\gamma_\times$}
\newcommand{\deltasig}{$\Delta \Sigma$}
\newcommand{\deltaplus}{$\Delta \Sigma_+$}
\newcommand{\deltacross}{$\Delta \Sigma_\times$}
\newcommand{\deltarho}{$\Delta \rho$}
\newcommand{\movr}{$M(<r)$}
\newcommand{\sigmacrit}{$\Sigma_{crit}$}

\newcommand{\photoz}{photo-z}
\newcommand{\photozs}{photo-zs}

\newcommand{\tlum}{$L^{tot}$}
\newcommand{\tngal}{$N_{gal}^{tot}$}

\newcommand{\lstarlim}{$0.4 L_*$}
\newcommand{\lvir}{$L_{200}$}
\newcommand{\nvir}{$N_{200}$}
\newcommand{\rvir}{$r_{200}^{gals}$}

\newcommand{\ngal}{$N_{gal}$}
\newcommand{\maxbcg}{maxBCG}
\newcommand{\numNgalBins}{12}
\newcommand{\numLumBins}{16}

\newcommand{\tngalAperture}{2$h^{-1}$ Mpc}

\newcommand{\photo}{\texttt{PHOTO}}
\newcommand{\astrop}{\texttt{ASTRO}}
\newcommand{\mt}{\texttt{MT}}
\newcommand{\spectro}{\texttt{SPECTRO}}
\newcommand{\spectroone}{\texttt{SPECTRO1d}}
\newcommand{\spectrotwo}{\texttt{SPECTRO2d}}
\newcommand{\target}{\texttt{TARGET}}

\newcommand{\lenszmax}{0.3}
\newcommand{\lenszmin}{0.05}

\newcommand{\photoversion}{\texttt{v5\_4}}

%\def\eone{e$_1$}
%\def\etwo{e$_2$}
\newcommand{\etan}{e$_+$}
\newcommand{\erad}{e$_\times$}
\newcommand{\eclass}{\texttt{ECLASS}}
\newcommand{\eclasscut}{-0.06}
\newcommand{\gmrcut}{0.7}

\newcommand{\hrs}{$^{\mathrm h}$}
\newcommand{\minutes}{$^{\mathrm m}$}

\newcommand{\ugriz}{$u, g, r, i, z$}
\newcommand{\polarization}{polarization}

\newcommand{\wgm}{$w_{gm}$}
\newcommand{\wgg}{$w_{gg}^p$}
\newcommand{\wmm}{$w_{mm}$}
\newcommand{\xigg}{$\xi_{gg}$}
\newcommand{\ximm}{$\xi_{mm}$}
\newcommand{\xigm}{$\xi_{gm}$}

\newcommand{\numspec}{127,001}
\newcommand{\numspecvlim}{10,277}
\newcommand{\numrand}{1,270,010}
\newcommand{\numspectot}{278,192}
\newcommand{\numvdis}{49,024}
%\newcommand{\numsource}{10,259,949}
% hirata: 
\newcommand{\nummask}{1,815,043}
\newcommand{\numTenMpc}{132,473}
\newcommand{\numThirtyMpc}{101,221}
\newcommand{\numsource}{27,912,891}

\newcommand{\numpairsTenMpc}{2,670,898,177}
\newcommand{\altnumpairsTenMpc}{2.7 billion}
\newcommand{\numpairsThirtyMpc}{14,818,082,122}
\newcommand{\altnumpairsThirtyMpc}{14.8 billion}



\newcommand{\xirmax}{$\xi_{gm}(R_{max})$}


\def\eps@scaling{1.0}% 

\newcommand{\sn}{$S/N$}
\newcommand{\Msn}{$(S/N)_{\textrm{matched}}$}
\newcommand{\Tsn}{$(S/N)_{\textrm{size}}$}
\newcommand{\fsn}{$(S/N)_{\textrm{flux}}$}

% stolen from the BA14 source
\newcommand{\vecg}{\mbox{\boldmath $g$}}
\newcommand{\vecD}{\mbox{\boldmath $D$}}
\newcommand{\vecQ}{\mbox{\boldmath $Q$}}
\newcommand{\matR}{\mbox{$\bf R$}}
\newcommand{\matC}{\mbox{$\bf C$}}
\newcommand{\bnabg}{ \boldsymbol{\nabla_g}}

\newcommand{\desreq}{$4\times 10^{-3}$}
\newcommand{\lsstreq}{$2\times 10^{-3}$}

\newcommand{\sersic}{S\'{e}rsic}

\newcommand{\lognormscatt}{30}

\newcommand{\mnras}{MNRAS}%
\newcommand{\apj}{ApJ}%
\newcommand{\aj}{AJ}%
\newcommand{\pasp}{PASP}%
\newcommand{\jcp}{J.~Chem.~Phys.}

\newcommand{\mcal}{metacalibration}
\newcommand{\Mcal}{Metacalibration}
\newcommand{\mcalR}{$R$}

%\slugcomment{Last revision \today}
%\shortauthors{Sheldon}
%\shorttitle{Bayesian Shear Estimation}


\title{Correlated Noise Effects in \Mcal\ Lensing Shear Estimation}

\author[Erin S. Sheldon]{Erin S. Sheldon\thanks{E-mail: erin.sheldon@gmail.com}\\
Brookhaven National Laboratory, Bldg 510, Upton, New York 11973}

\begin{document}

\maketitle

\begin{abstract}

I derive corrections for correlated noise effects in \mcal.   

\end{abstract}


\begin{keywords}                                                                    
    cosmology: observations,
    gravitational lensing: weak,
    dark energy
\end{keywords} 

\section{Introduction} \label{sec:intro}

\section{\Mcal} \label{sec:algo}

Suppose we have a biased shear estimator $E$.  We can expand this estimator
in a Taylor series around the true shear
\begin{eqnarray}
    E(\gamma) & = & E(0) + \gamma ~ \frac{ \partial E }{ \partial \gamma }\bigg|_{\gamma=0}  + ... \nonumber \\
      & \approx & \gamma ~ \frac{ \partial E }{ \partial \gamma } \bigg|_{\gamma=0}  \\
      & \equiv & \gamma ~ R \nonumber
\end{eqnarray}
where $\gamma$ is denotes the true gravitational shear.

The essence of \mcal\ \citep{HuffMcal} is to estimate the shear response
\mcalR\ for a shear estimator $E$ from image data.  This is accomplished using
a numerical derivative.  The original image $I$ is de-convolved from the point
spread function (PSF), sheared, and re-convolved with a slightly larger PSF.  A
slightly larger PSF is used to suppress the amplified noise from high
frequencies.  This is \mcal\ process is repeated for a positive and negative
shear, which can be used to form a central derivative.  We can represent this
as a series of operations on the observed image $I$:
\begin{equation}
    I(\gamma) = I \Asterisk P^{-1} \oplus \gamma \Asterisk P_{d}
\end{equation}
where $\Asterisk$ represents convolution, $\oplus$ represents shearing,
and $P$ represents the point spread function.  De-convolution
is represented as ``inverse``, $P^{-1}$.  The slightly larger PSF, or
dilated PSF, is represented by $P_{d}$.

To form the central derivative we shear by a small positive and negative
amount $h$
\begin{equation} \label{eq:Rnum}
    R = \frac{E(\gamma_+) - E(\gamma_-)}{2 h}.
\end{equation}
The shear estimator is also derived from the re-convolved
images
\begin{equation} \label{eq:estimator}
    E = \frac{E(\gamma_+) + E(\gamma_-)}{2}.
\end{equation}
One could also used an unsheared, re-convolved image for the
shear estimator, but it should be
equivalent to equation \ref{eq:estimator} if the \mcal\ steps
introduce no additive errors.

\section{Contamination of the Response by Correlated Noise} \label{sec:contam}

In presence of noise, the \mcal\ images will contain
contributions from de-convolved, sheared and re-convolved noise:
\begin{eqnarray}
    I^{\prime}(\gamma) & = & (I + I_\eta) \Asterisk P^{-1} \oplus \gamma \Asterisk P_{d} \nonumber \\
    & = & I(\gamma) + I_\eta(\gamma),
\end{eqnarray}
where $I_\eta$ is the ``noise image''.  The de-convolution correlates the noise
across the image.  This correlated noise is sheared, and then re-convolved by
the dilated PSF.  This correlated noise term should cancel in the estimator in
equation \ref{eq:estimator}, because the correlated noise field is averaged
over both positive and negative shears.  But this term will not cancel when
calculating the response, which is the difference of positive and negative
shears.  We thus expect a contamination in the measured response
\begin{equation}
    R^{\prime}  =  R + R_{\eta}.
\end{equation}

\section{Corrections for Correlated Noise} \label{sec:corr}

We propose to simulate the effects of the correlated noise on the measurement
of the response.  We generate images with the right noise level and example
the response of this noise to a shear.  We then subtract the mean noise
response from the mean measured response, averaged over all galaxies.

For shear estimators based on raw moments, without any form of iteration or
fitting, it would be sufficient to simply measure the moments of the resulting
sheared correlated noise field.

For shear estimators based on non-linear fitting, we propose the following
process:
\begin{enumerate}[label=\arabic*.]

    \item Measure the response without correlated noise
    
        \begin{enumerate}[label*=\arabic*.]
            \item Render the best fit model, including convolution by the point-spread
                function

            \item Perform the \mcal\ image processing steps, without noise added.

            \item Add an appropriate amount of noise to all images, using the
                same noise pattern $I_\eta$ for every image.

            \item Measure the response for this model $R^{model}$.
        \end{enumerate}

    \item Measure the response with correlated noise
    
        \begin{enumerate}[label*=\arabic*.]
            \item Render the best fit model, including convolution by the point-spread
                function

            \item Add an appropriate noise patter $I_\eta$ to the image, the
                same pattern used for measuring the response without correlated
                noise.

            \item Perform the \mcal\ image processing steps on this noisy
                image.

            \item Measure the response for this model $R^{model}_{\eta}$.

        \end{enumerate}

\end{enumerate}

The measurement with correlated noise will be the sum of the response
without correlated noise and the response of the correlated noise field
\begin{equation}
    R^{model}_\eta = R^{model} + R_\eta
\end{equation}
Thus an estimate of the contribution from correlated noise can
be calculated by averaging 
\begin{equation}
    \langle R_\eta \rangle = \langle R^{model}_\eta \rangle - \langle R^{model} \rangle,
\end{equation}
which can be subtracted to recover the unbiased mean response
\begin{equation}
    \langle R \rangle = \langle R^\prime \rangle - \langle R_\eta \rangle.
\end{equation}

This correction does not depend on the details of the model used, since it is
only the difference in response that is desired.  For consistency, we choose to
use the model that was used in the shear estimator, and we fit the generated
model images using the same model.

\subsection{Corrections to $R_{PSF}$}

A similar correction may exist for the $R_{PSF}$, but in general this should
be quite small, since the noise is not sheared.

\section{Cancellation in Ring Configuration}

This bias does not seem to appear in GREAT3.  I would not expect it to cancel
for 90 degree rotated pairs of galaxies unless their noise maps were also
identical but rotated.


\bibliographystyle{mn2e}
% Bib database
\bibliography{apj-jour,astroref}

\end{document}

