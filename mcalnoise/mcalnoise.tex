\documentclass[usegraphicx,usenatbib]{mn2e}
%\documentclass[useAMS,usegraphicx,usenatbib]{mn2e}

\usepackage{verbatim}
\usepackage{color}
\usepackage[normalem]{ulem} % for striking out with \sout
\usepackage{amsmath} % for boldsymbol
\usepackage{times}
\usepackage{mathabx}
\usepackage{enumitem}

% A comment block

%\newcommand{\comment}[1]{}

% For color
\newcommand{\mpname}[1]{#1_color.eps}
\newcommand{\clraitoff}{red}
\newcommand{\lumblack}{(black)}
\newcommand{\lumblue}{(blue)}
\newcommand{\lumred}{(red)}
\newcommand{\vdisred}{(red-dashed curve)}
\newcommand{\vdisblue}{(blue-solid curve)}

% For bw
%\newcommand{\mpname}[1]{#1.eps}
%\newcommand{\clraitoff}{}
%\newcommand{\lumblack}{}
%\newcommand{\lumblue}{}
%\newcommand{\lumred}{}
%\newcommand{\vdisred}{(dashed curve)}
%\newcommand{\vdisblue}{(solid curve)}

\newcommand{\umag}{$u$}
\newcommand{\gmag}{$g$}
\newcommand{\rmag}{$r$}
\newcommand{\imag}{$i$}
\newcommand{\zmag}{$z$}
\newcommand{\gmr}{$g-r$}



\newcommand{\gammat}{$\gamma_T$}
\newcommand{\gammacross}{$\gamma_\times$}
\newcommand{\deltasig}{$\Delta \Sigma$}
\newcommand{\deltaplus}{$\Delta \Sigma_+$}
\newcommand{\deltacross}{$\Delta \Sigma_\times$}
\newcommand{\deltarho}{$\Delta \rho$}
\newcommand{\movr}{$M(<r)$}
\newcommand{\sigmacrit}{$\Sigma_{crit}$}

\newcommand{\photoz}{photo-z}
\newcommand{\photozs}{photo-zs}

\newcommand{\tlum}{$L^{tot}$}
\newcommand{\tngal}{$N_{gal}^{tot}$}

\newcommand{\lstarlim}{$0.4 L_*$}
\newcommand{\lvir}{$L_{200}$}
\newcommand{\nvir}{$N_{200}$}
\newcommand{\rvir}{$r_{200}^{gals}$}

\newcommand{\ngal}{$N_{gal}$}
\newcommand{\maxbcg}{maxBCG}
\newcommand{\numNgalBins}{12}
\newcommand{\numLumBins}{16}

\newcommand{\tngalAperture}{2$h^{-1}$ Mpc}

\newcommand{\photo}{\texttt{PHOTO}}
\newcommand{\astrop}{\texttt{ASTRO}}
\newcommand{\mt}{\texttt{MT}}
\newcommand{\spectro}{\texttt{SPECTRO}}
\newcommand{\spectroone}{\texttt{SPECTRO1d}}
\newcommand{\spectrotwo}{\texttt{SPECTRO2d}}
\newcommand{\target}{\texttt{TARGET}}

\newcommand{\lenszmax}{0.3}
\newcommand{\lenszmin}{0.05}

\newcommand{\photoversion}{\texttt{v5\_4}}

%\def\eone{e$_1$}
%\def\etwo{e$_2$}
\newcommand{\etan}{e$_+$}
\newcommand{\erad}{e$_\times$}
\newcommand{\eclass}{\texttt{ECLASS}}
\newcommand{\eclasscut}{-0.06}
\newcommand{\gmrcut}{0.7}

\newcommand{\hrs}{$^{\mathrm h}$}
\newcommand{\minutes}{$^{\mathrm m}$}

\newcommand{\ugriz}{$u, g, r, i, z$}
\newcommand{\polarization}{polarization}

\newcommand{\wgm}{$w_{gm}$}
\newcommand{\wgg}{$w_{gg}^p$}
\newcommand{\wmm}{$w_{mm}$}
\newcommand{\xigg}{$\xi_{gg}$}
\newcommand{\ximm}{$\xi_{mm}$}
\newcommand{\xigm}{$\xi_{gm}$}

\newcommand{\numspec}{127,001}
\newcommand{\numspecvlim}{10,277}
\newcommand{\numrand}{1,270,010}
\newcommand{\numspectot}{278,192}
\newcommand{\numvdis}{49,024}
%\newcommand{\numsource}{10,259,949}
% hirata: 
\newcommand{\nummask}{1,815,043}
\newcommand{\numTenMpc}{132,473}
\newcommand{\numThirtyMpc}{101,221}
\newcommand{\numsource}{27,912,891}

\newcommand{\numpairsTenMpc}{2,670,898,177}
\newcommand{\altnumpairsTenMpc}{2.7 billion}
\newcommand{\numpairsThirtyMpc}{14,818,082,122}
\newcommand{\altnumpairsThirtyMpc}{14.8 billion}



\newcommand{\xirmax}{$\xi_{gm}(R_{max})$}


\def\eps@scaling{1.0}% 

\newcommand{\sn}{$S/N$}
\newcommand{\Msn}{$(S/N)_{\textrm{matched}}$}
\newcommand{\Tsn}{$(S/N)_{\textrm{size}}$}
\newcommand{\fsn}{$(S/N)_{\textrm{flux}}$}

% stolen from the BA14 source
\newcommand{\vecg}{\mbox{\boldmath $g$}}
\newcommand{\vecD}{\mbox{\boldmath $D$}}
\newcommand{\vecQ}{\mbox{\boldmath $Q$}}
\newcommand{\matR}{\mbox{$\bf R$}}
\newcommand{\matC}{\mbox{$\bf C$}}
\newcommand{\bnabg}{ \boldsymbol{\nabla_g}}

\newcommand{\desreq}{$4\times 10^{-3}$}
\newcommand{\lsstreq}{$2\times 10^{-3}$}


\newcommand{\mnras}{MNRAS}%
\newcommand{\apj}{ApJ}%
\newcommand{\apjs}{ApJS}%
\newcommand{\aj}{AJ}%
\newcommand{\pasp}{PASP}%
\newcommand{\jcp}{J.~Chem.~Phys.}

\newcommand{\mcal}{metacalibration}
\newcommand{\Mcal}{Metacalibration}
\newcommand{\mcalR}{$R$}
\newcommand{\mcalRpsf}{$R^{p}$}
\newcommand{\mcalRpsfnoise}{$R^{p}_\eta$}
\newcommand{\mcalRo}{$R_o$}
\newcommand{\mcalRnoise}{$R_\eta$}

\newcommand{\mcalRmodel}{$R^{model}$}
\newcommand{\mcalRnoisemodel}{$R^{model}_\eta$}

%\slugcomment{Last revision \today}
%\shortauthors{Sheldon}
%\shorttitle{Bayesian Shear Estimation}


\title{Correction for Noise Effects in \Mcal\ Weak Lensing Shear Estimation}

\author[Erin S. Sheldon]{Erin S. Sheldon\thanks{E-mail: erin.sheldon@gmail.com}\\
Brookhaven National Laboratory, Bldg 510, Upton, New York 11973}

\begin{document}

\maketitle

\begin{abstract}

I implement corrections for sheared correlated noise effects in \mcal.   

\end{abstract}


\begin{keywords}                                                                    
    cosmology: observations,
    gravitational lensing: weak,
    dark energy
\end{keywords} 

\section{Introduction} \label{sec:intro}

\section{\Mcal} \label{sec:algo}

Suppose we have a biased shear estimator $E$.  We can expand this estimator
in a Taylor series around the true shear
\begin{align}
    E(\gamma) &= E(\gamma=0) + \gamma ~ \frac{ \partial E }{ \partial \gamma }\bigg|_{\gamma=0}  + ... \nonumber \\
      & \approx  \gamma ~ \frac{ \partial E }{ \partial \gamma } \bigg|_{\gamma=0}  \\
      & \equiv  \gamma ~ \mbox{\mcalR} \nonumber
\end{align}
where $\gamma$ is denotes the true gravitational shear.  We call \mcalR\
the shear response.

The essence of \mcal\ \citep{HuffMcal} is to estimate the shear response
\mcalR\ for a shear estimator $E$ directly from image data.  This is accomplished using
a numerical derivative.  The original image $I$ is de-convolved from the point
spread function (PSF), sheared, and re-convolved with a slightly larger PSF.  A
slightly larger PSF is used to suppress the amplified noise at high
frequency.  This \mcal\ process is repeated for a positive and negative
shear, which can be used to form a central derivative.  We can represent this
as a series of operations on the observed image $I$:
\begin{equation}
    I(\gamma) = I \Asterisk P^{-1} \oplus \gamma \Asterisk P_{d}
\end{equation}
where $\Asterisk$ represents convolution, $\oplus$ represents shearing,
and $P$ represents the point spread function.  De-convolution
is represented as ``inverse``, $P^{-1}$.  The slightly larger PSF, or
dilated PSF, is represented by $P_{d}$.

To form the central derivative we shear by a small positive and negative
amount $\gamma$
\begin{equation} \label{eq:Rnum}
    R = \frac{E(+\gamma) - E(-\gamma)}{2 \gamma}.
\end{equation}
The shear estimator can also be derived from the re-convolved
images
\begin{equation} \label{eq:estimator}
    E = \frac{E(+\gamma) + E(-\gamma)}{2}.
\end{equation}

For a constant shear, the response and shear estimators can be averaged
separately and combined to recover the mean shear:
\begin{equation}
    \gamma = \frac{ \langle E \rangle }{\langle \mbox{\mcalR} \rangle}.
\end{equation}
In \cite{HuffMcal} a more sophisticated inference was used, in order to deal
with the relatively large variance in \mcalR\ associated with the particular
estimator used therein.

\section{Contamination of the Response by Correlated Noise} \label{sec:contam}

In the presence of noise, the observed image can be written $I_o=I+\eta$, where $\eta$
is the ``noise image``.  The \mcal\ sheared images $I_o(\gamma)$ will contain
contributions from de-convolved, sheared and re-convolved noise:
\begin{align}
    I_{o}(\gamma) &= (I + \eta) \Asterisk P^{-1} \oplus \gamma \Asterisk P_{d} \nonumber \\
    &= I(\gamma) + \eta(\gamma).
\end{align}
The de-convolution correlates the noise
across the image.  This correlated noise is sheared, and then re-convolved by
the dilated PSF, producing the sheared correlated noise image $\eta(\gamma)$.

We expect the sheared correlated noise to partly cancel in the estimator in
equation \ref{eq:estimator}, because the effect is averaged over both positive
and negative shears. This quantity is further averaged over many galaxies.

However, the sheared correlated noise term will not cancel when calculating the
response, which is the difference of positive and negative shears.  This
remainder is amplified due to division by $2 \gamma$ to form the central
derivative.  We thus expect the observed response \mcalRo\ to be contaminated
by the response of the correlated, sheared noise \mcalRnoise\
\begin{equation}
    \mbox{\mcalRo}  =  R + \mbox{\mcalRnoise}
\end{equation}

\begin{comment}
\subsection{Expected Level of Contamination}

I think the following is not correct.

The correlated noise is produced by de-convolving and re-convolving by the psf,
so it may be correlated with the PSF ellipticity at some level.  Let's assume
the induced ellipticity is $f \times e^{PSF}$ in the absence of shearing, with
$f$ small.  We expect this to cancel in the response calculation, due to the
subtraction, but add to the estimator $E$.  This should, however, be
partly dealt with by subtracting $R^{PSF}$ (see below).

If we introduce shearing, then the sheared noise will cancel in the estimator
$E$ but not the response.

This sheared part of the correlated noise is suppressed by a factor of
$\gamma$, but in forming the derivative another factor of $1/\gamma$ is
applied.  Thus we expect the overall contamination $R_\eta$ to be of
the order $f \times e^{PSF}$.

\end{comment}

\section{Correction for Sheared, Correlated Noise} \label{sec:corr}

\subsection{Raw Moments}

For shear estimators based on raw moments, without any form of iteration or
fitting, it would be sufficient to simply measure the moments of example
sheared correlated noise fields and subtract the mean response from the
responses measured on the real images.

\subsection{Nonlinear Fitting}

For non-linear fitting, we propose two methods to correct for the effects of
the sheared, correlated noise on the measured response.  The first is based on
subtracting a sheared, correlated noise field that has been sheared with the
opposite sign.  The second is based on simulating the effect on modeled images.

\subsubsection{Subtracting a Sheared, Correlated Random Noise Field}

In this method, we generate example random noise fields sheared with the
opposite sign of those used in the \mcal\ procedure:
\begin{equation}
    \tilde{\eta}(-\gamma) = \tilde{\eta} \Asterisk P^{-1} \oplus (-\gamma) \Asterisk P_{d}.
\end{equation}
This noise image can then be subtracted from the $I(\gamma)_o$ images to
statistically remove the effects of sheared, correlated noise.
\begin{equation}
    \tilde{I}(\gamma) = I_o(\gamma) - \tilde{\eta}(-\gamma),
\end{equation}
where $\tilde{I}(\gamma)$ is an approximation for the desired image
$I(\gamma)$.    When used in a large ensemble, these $\tilde{I}(\gamma)$
can be used to measure the correct \mcal\ for the population.
It is important that the random noise field be subtracted after
the \mcal\ procedures have been applied, not before.

This procedure increases the noise by a factor of $\sqrt 2$.  Thus, both the
response and estimator should be measured on the $\tilde{I}(\gamma)$ images, so
that the response is representative of the correct noise level.  To reduce the
noise in the fits, many random fields can be generated and subtracted from the
original images, and these can all then be fit simultaneously.  This will
improve the precision of the fit, at the cost of increased computation time.


\subsubsection{Simulating Models}

In this method, we generate model images with the right noise level for a
given galaxy, and then measure the response of the noise due to the
convolutions and shears used in \mcal.  We then subtract the mean sheared noise
response from the mean measured response, averaged over some set of galaxy
images. The process is as follows:
\begin{enumerate}[label=\arabic*.]

    \item Measure the response without correlated noise
    
        \begin{enumerate}[label*=\arabic*.]
            \item Render the best fit model, including convolution by the point-spread
                function

            \item Perform the \mcal\ image processing steps, without noise added.

            \item Add an appropriate amount of noise to all images, using the
                same noise pattern $I_\eta$ for every image.

            \item Measure the response for this model \mcalRmodel.
        \end{enumerate}

    \item Measure the response with correlated noise
    
        \begin{enumerate}[label*=\arabic*.]
            \item Render the best fit model, including convolution by the point-spread
                function

            \item Add an appropriate noise pattern $I_\eta$ to the image, the
                same pattern used for measuring the response without correlated
                noise.

            \item Perform the \mcal\ image processing steps on this noisy
                image.

            \item Measure the response for this model \mcalRnoisemodel.

        \end{enumerate}

    \item Subtract the correlated noise response

\end{enumerate}

The measurement with correlated noise will be the sum of the response
without correlated noise plus the response of the correlated noise field
\begin{equation}
    \mbox{\mcalRnoisemodel} = \mbox{\mcalRmodel} + \mbox{\mcalRnoise}
\end{equation}
This measurement is quite noisy for a single galaxy, but we
can estimate the mean correlated noise response for an ensemble
of galaxies
\begin{equation}
    \langle \mbox{\mcalRnoise} \rangle = \langle \mbox{\mcalRnoisemodel} \rangle - \langle \mbox{\mcalRmodel} \rangle.
\end{equation}
Each entry used in this average corresponds to the best fit model
and noise properties for a galaxy in the sample.

The response \mcalRnoise\ can be subtracted to recover an estimate of the mean
response without correlated noise
\begin{equation}
    \langle \mbox{\mcalR} \rangle = \langle \mbox{\mcalRo} \rangle - \langle \mbox{\mcalRnoise} \rangle.
\end{equation}

\subsection{Corrections to \mcalRpsf}

A correction may also be calculated for the PSF response term, which is
calculated by shearing the PSF image instead of the galaxy image \citep{HuffMcal}
\begin{equation}
    \mbox{\mcalRpsf} = \frac{E(+\gamma^{p}) - E(-\gamma^{p})}{2 \gamma^{p}},
\end{equation}
where $\gamma^{p}$ is the shear applied to the PSF.  This
term adds to the estimator, multiplied by the PSF shape
\begin{equation}
    E = \mbox{\mcalR} \gamma + \mbox{\mcalRpsf} e^{p} 
\end{equation}
This term can be subtracted to approximately correct for additive PSF errors.

In calculating \mcalRpsf, the image is de-convolved and re-convolved by these
sheared PSFs. The noise is thus correlated by convolutions, but in slightly
different ways due to the differently sheared PSFs.  We thus do not expect this
effect to cancel in calculating \mcalRpsf. The resulting correlated noise
contamination term we may denote \mcalRpsfnoise.

We expect this correction to be sub-dominant to \mcalRnoise, as \mcalRpsfnoise\
is suppressed in the product $\langle$\mcalRpsfnoise$ \times e^{p}\rangle$.

\section{Application to Realistic Image Simulations}

\section{Cancellation in Ring Configurations}

This bias was not seen by \cite{HuffMcal} using the GREAT3 simulations
\citep{great3}.

The only substantial difference between the GREAT3 sims and our sims is that
GREAT3 galaxies were generated in a ``ring configuration''.  The simulation
included two identical images of each galaxy, rotated by 90 degrees with
respect to a one another, in order to cancel shape noise.

It is not clear to me that \mcalRnoise\ should cancel in a ring configuration,
unless the noise maps for the rotated galaxies were also identical and rotated.



\bibliographystyle{mn2e}
% Bib database
\bibliography{apj-jour,astroref}

\end{document}

