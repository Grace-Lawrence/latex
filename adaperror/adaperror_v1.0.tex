
%\documentclass{aastex}

%% -- This is what phil used
%\documentclass[manuscript]{aastex}
%\usepackage{aastexug}

%% -- preprint produces a one-column, single-spaced document:
\documentclass[preprint]{aastex}

%% -- preprint2 produces a double-column, single-spaced document:
% \documentclass[preprint2]{aastex}

%% -- EMULATE APJ
%\documentclass{aastex}
%\usepackage{emulateapj5}

\newcommand{\PSbox}[3]{\mbox{\special{psfile=#1}\hspace{#2}\rule{0pt}{#3}}}
\newcommand{\pfg}{\vspace{0.3cm}}

%%
%% Some definitions
%%

\def\rp{$r^\prime$}
\def\gp{$g^\prime$}
\def\ip{$i^\prime$}
\def\up{$u^\prime$}
\def\zp{$z^\prime$}

%\slugcomment{To appear in \apj. Last revision 22-FEB-2001.}

%\shorttitle{SDSS/RASS Cluster Lensing}
%\shortauthors{Sheldon, et al.}

\begin{document}

\title{Uncertainties in Moments}

\author{ Erin Scott Sheldon }
\affil{Department of Physics, University of Michigan, 500 East University, Ann
Arbor, Mi 48109}
\email{esheldon@umich.edu}

\keywords{}

\section{Introduction}\label{intro}

Basically our error estimates for the moments seem to be too small.  I've
written out the the derivation of formulas that are being used. I don't claim
to know much about statistics, so I hope you will look these over and see
what might be wrong. I've also included a comparison of these formulas with
simulation a in order to demonstrate the problem.

\section{Uncertainties in Weighted Moments}

Weighted moments:
\begin{eqnarray}
\langle x^2-y^2 \rangle & = & { 1 \over N } \sum w(x,y) I(x,y) (x^2-y^2) \label{eq1} \\
\langle 2xy \rangle & = & { 1 \over N } \sum w(x,y) I(x,y) 2xy  \label{eq2} \\
\langle x^2+y^2 \rangle & = & { 1 \over N } \sum w(x,y) I(x,y) (x^2+y^2) \label{eq3}  \\
                      N & = & \sum w(x,y) I(x,y)
\end{eqnarray}

Now the uncertainties. The uncertainty in equation \ref{eq1} will be
$Var(x^2)+Var(y^2) + 2Cov(x,y)$. As an example, lets find the variance in the average
of $x^2$. Let $a_i = w_i I_i x_i^2/N$. Assuming we are sky noise limited:
\begin{eqnarray}
Var(\langle x^2 \rangle) & = & Var( \sum a_i ) = \sum Var(a_i) \\
                Var(a_i) & \approx & \left({ {\partial a_i} \over {\partial I_i}
                }\right)^2 \sigma_{sky}^2 \\
\left({ {\partial a_i} \over {\partial I_i}}\right)^2 & = & {1 \over N^2 } w_i^2(x_i^2 -
                { {x_i^2I_i} \over N})^2 \\
Var(\langle x^2 \rangle) & = & {\sigma_{sky}^2 \over N^2 } \sum w_i^2(x_i^2 -
                { {x_i^2I_i} \over N})^2 = {\sigma_{sky}^2 \over N^2 } \sum
                w_i^2 x_i^4 (1 - { {I_i} \over N})^2 \label{eq4}
\end{eqnarray}
The equations will be similar for $y^2$ and $2xy$. For error in the output
moments, we must add an additional factor of 2 because we save the moments of
the adaptive weight which is 2 times as big as the moment of the object. 
In the PHOTO code, Phil approximated equation \ref{eq4} as:
\begin{equation}
Var(\langle x^2 \rangle) = {\sigma_{sky}^2 \over N^2 } \sum w_i^2(x_i^2 -
                \langle x^2 \rangle)^2
\end{equation}
I'm not sure why he did so, but it certainly looks more like a standard error
estimator than my formula. The two formulas agree pretty
well, with his formula giving somewhat smaller errors.
%The formula for the covariance will be:
%\begin{equation}
%Cov(x,y) = {\sigma_{sky}^2 \over N^2 } \sum w_i^2 x_i^2 y_i^2 (1 - { {I_i} \over N})^2
%\end{equation}
%or maybe this instead:
%\begin{equation}
%Cov(x,y) = {\sigma_{sky}^2 \over N^2 } \sum w_i^2 (x_i y_i - \langle xy \rangle)^2
%\end{equation}
%The $2Cov(x,y)$ part will be about a fourth as big as $Var(x^2) +
%Var(y^2)$. Because we don't want to save this covariance in the outputs, I
%propose we save equations \ref{eq1}, \ref{eq2}, \ref{eq3} and their errors.

\section{Comparison with Simulation}

A simple simulation with gaussian objects was run. Objects were created with
various ellipticities, orientations and signal to noise. Figure
\ref{testerror1} shows one realization of ellipticity and position angle. Each
signal to noise bin has 150 realizations. Clearly the formal error is smaller
than the measured error, by as much as a factor of 2.  The slope is most likely
because the assumption of sky noise limited is becoming less correct. Results
are similar for different orientations and ellipticities. The formal errors
do not include the covariance, but I've tested it with them and it doesn't
seem to make much difference, about 10\%.


\begin{figure}
\plotone{test_error_form_theta15_one.ps}
\figcaption{Measured scatter divided by formal error in e1 and e2 as a function of
object signal to noise. Objects had ellipticity of .3, oriented with
15 degrees relative to the column direction. The slope is most likely
because the assumption of sky noise limited is becoming less correct.
The formal error clearly underestimates the error.} \label{testerror1}
\end{figure}


\end{document}
