\documentclass[12pt,preprint]{aastex}

\usepackage{verbatim}
\usepackage{color}

\usepackage[normalem]{ulem} % for striking out with \sout
% A comment block

%\newcommand{\comment}[1]{}

% For color
\newcommand{\mpname}[1]{#1_color.eps}
\newcommand{\clraitoff}{red}
\newcommand{\lumblack}{(black)}
\newcommand{\lumblue}{(blue)}
\newcommand{\lumred}{(red)}
\newcommand{\vdisred}{(red-dashed curve)}
\newcommand{\vdisblue}{(blue-solid curve)}

% For bw
%\newcommand{\mpname}[1]{#1.eps}
%\newcommand{\clraitoff}{}
%\newcommand{\lumblack}{}
%\newcommand{\lumblue}{}
%\newcommand{\lumred}{}
%\newcommand{\vdisred}{(dashed curve)}
%\newcommand{\vdisblue}{(solid curve)}

\newcommand{\umag}{$u$}
\newcommand{\gmag}{$g$}
\newcommand{\rmag}{$r$}
\newcommand{\imag}{$i$}
\newcommand{\zmag}{$z$}
\newcommand{\gmr}{$g-r$}



\newcommand{\gammat}{$\gamma_T$}
\newcommand{\gammacross}{$\gamma_\times$}
\newcommand{\deltasig}{$\Delta \Sigma$}
\newcommand{\deltaplus}{$\Delta \Sigma_+$}
\newcommand{\deltacross}{$\Delta \Sigma_\times$}
\newcommand{\deltarho}{$\Delta \rho$}
\newcommand{\movr}{$M(<r)$}
\newcommand{\sigmacrit}{$\Sigma_{crit}$}

\newcommand{\photoz}{photo-z}
\newcommand{\photozs}{photo-zs}

\newcommand{\tlum}{$L^{tot}$}
\newcommand{\tngal}{$N_{gal}^{tot}$}

\newcommand{\lstarlim}{$0.4 L_*$}
\newcommand{\lvir}{$L_{200}$}
\newcommand{\nvir}{$N_{200}$}
\newcommand{\rvir}{$r_{200}^{gals}$}

\newcommand{\ngal}{$N_{gal}$}
\newcommand{\maxbcg}{maxBCG}
\newcommand{\numNgalBins}{12}
\newcommand{\numLumBins}{16}

\newcommand{\tngalAperture}{2$h^{-1}$ Mpc}

\newcommand{\photo}{\texttt{PHOTO}}
\newcommand{\astrop}{\texttt{ASTRO}}
\newcommand{\mt}{\texttt{MT}}
\newcommand{\spectro}{\texttt{SPECTRO}}
\newcommand{\spectroone}{\texttt{SPECTRO1d}}
\newcommand{\spectrotwo}{\texttt{SPECTRO2d}}
\newcommand{\target}{\texttt{TARGET}}

\newcommand{\lenszmax}{0.3}
\newcommand{\lenszmin}{0.05}

\newcommand{\photoversion}{\texttt{v5\_4}}

%\def\eone{e$_1$}
%\def\etwo{e$_2$}
\newcommand{\etan}{e$_+$}
\newcommand{\erad}{e$_\times$}
\newcommand{\eclass}{\texttt{ECLASS}}
\newcommand{\eclasscut}{-0.06}
\newcommand{\gmrcut}{0.7}

\newcommand{\hrs}{$^{\mathrm h}$}
\newcommand{\minutes}{$^{\mathrm m}$}

\newcommand{\ugriz}{$u, g, r, i, z$}
\newcommand{\polarization}{polarization}

\newcommand{\wgm}{$w_{gm}$}
\newcommand{\wgg}{$w_{gg}^p$}
\newcommand{\wmm}{$w_{mm}$}
\newcommand{\xigg}{$\xi_{gg}$}
\newcommand{\ximm}{$\xi_{mm}$}
\newcommand{\xigm}{$\xi_{gm}$}

\newcommand{\numspec}{127,001}
\newcommand{\numspecvlim}{10,277}
\newcommand{\numrand}{1,270,010}
\newcommand{\numspectot}{278,192}
\newcommand{\numvdis}{49,024}
%\newcommand{\numsource}{10,259,949}
% hirata: 
\newcommand{\nummask}{1,815,043}
\newcommand{\numTenMpc}{132,473}
\newcommand{\numThirtyMpc}{101,221}
\newcommand{\numsource}{27,912,891}

\newcommand{\numpairsTenMpc}{2,670,898,177}
\newcommand{\altnumpairsTenMpc}{2.7 billion}
\newcommand{\numpairsThirtyMpc}{14,818,082,122}
\newcommand{\altnumpairsThirtyMpc}{14.8 billion}



\newcommand{\xirmax}{$\xi_{gm}(R_{max})$}


\newcommand{\M}{\textbf{M}}
\newcommand{\X}{\textbf{X}}
\newcommand{\Dx}{\ensuremath{\Delta x}}
\newcommand{\Dy}{\ensuremath{\Delta y}}

\newcommand{\downloadURL}{{\tt http://www.sdss3.org/dr8/data\_access.php\#VAC}}
\newcommand{\datamodelURL}{{\tt http://data.sdss3.org/datamodel/files/BOSS\_PHOTOOBJ/photoz-weight/pofz.html}}
\def\eps@scaling{1.0}% 

\slugcomment{Last revision \today}
\shortauthors{Sheldon}
\shorttitle{Gaussian Mixtures for Shape Measurement}

\begin{document}

\title{Gaussian Mixtures for Shape Measurement}

\author{
Erin S. Sheldon\altaffilmark{1}
}

\altaffiltext{1}{Brookhaven National Laboratory, Bldg 510, Upton, New York 11973}


\begin{abstract}

GMix

\end{abstract}

\section{Gaussian Mixtures} \label{sec:gmix}

The two dimensional Gaussian intensity function is defined as
\begin{equation}
G = \frac{1}{2 \pi \sqrt{|\M|} } ~~ \textrm{exp}\left( -\frac{1}{2} \X^T \M^{-1} \X \right)
\end{equation}
where \M\ is the covariance matrix
\begin{equation}
\M = \left( \begin{array}{cc}
\M_{xx} & \M_{xy} \\
\M_{xx} & \M_{yy} \end{array} \right)
\end{equation}
and \X\ is the position vector relative to the mean of the Gaussian
\begin{equation}
\X = \left( \begin{array}{c}
\Dx \\
\Dy \end{array} \right),
\end{equation}
where we have defined the positions relative to the center $(x_0,y_0)$
as $\Delta x=(x-x_0)$ and $\Delta y=(y-y_0)$.

Under convolution with a Gaussian PSF $P$ we get an observed model intensity
profile
\begin{equation}
G_o = G * P
\end{equation}
In this case the covariance matrices simply add
\begin{equation}
\M_o = \M + \M_{P}
\end{equation}

We can fit a number of gaussians to a galaxy or star image.  There is 
no guarantee this will be a good representation, as sums of Gaussians
do not form a complete set of functions.   This model can be represented
as
\begin{equation}
I = \sum_{i=1}^{N_{g}} p_i G_i
\end{equation}

If the PSF is also represented as a sum of gaussians, then the observed
light intensity profile can be modeled as
\begin{eqnarray}
I_o & = & \sum_{i=1}^{N_{g}} p_i \sum_{j=1}^{N_{P}} G_i * P_j \\
    & = & \sum_{i=1}^{N_{g}} p_i \sum_{j=1}^{N_{P}} \frac{1}{2 \pi \sqrt{|\M + \M_P|} } ~~ \textrm{exp}\left( -\frac{1}{2} \X^T (\M+\M_P)^{-1} \X \right) \\
\end{eqnarray}
where the $P_j$ are normalized such that the integral over
the plane is unity
\begin{equation}
\int \sum_{j=1}^{N_{P}} P_j dA = 1.
\end{equation}

\subsection{Alternative Parametrization}

An alternative parametrization of the covariance matrix is in
terms of the ellipticity parameters and size
\begin{equation}
\M = \frac{T}{2}\left( \begin{array}{cc}
1+e_1 & e_2 \\
e_2 & 1-e_1 \end{array} \right),
\end{equation}
where the ellipticity and size $T$ are defined in terms
of the elements of the covariance matrix as
\begin{eqnarray}
T & = & M_{xx} + M_{yy} \\
e_1 & = & \frac{M_{xx}-M_{yy}}{T} \\
e_2 & = & \frac{2 M_{xy}}{T} \\
e & = & \sqrt{e_1^2 + e_2^2}.
\end{eqnarray}
The $e_1$ and $e_2$ are bounded within $[-1,1]$ and total ellipticity $e$ is
within $[0,1]$.  Note in this parametrization, 
the argument of the exponential is
\begin{equation}
\frac{1}{2} \X^T \M^{-1} \X = \frac{\Dx^2 (1-e_1) - 2 \Dx \Dy~e_2 + \Dy^2 (1+e_1)}{T (1-e^2)}
\end{equation}


\subsection{Jacobian}

The Jacobian is the set of first derivatives of model with respect
to the parameters:
\begin{equation}
J_i = \frac{\partial G}{\partial \beta_i}
\end{equation}

For a single Gaussian, the elements of the Jacobian can be readily
calculated.  For the parameters $x_0,y_0,\theta,T,p$ we find
\begin{eqnarray}
\frac{1}{G} \frac{\partial G}{\partial x_0}
    & = & 2 \frac{ \Dx (1-e_1) - \Dy e_2 }{T (1-e^2)} \\
\frac{1}{G} \frac{\partial G}{\partial y_0}
    & = & 2 \frac{ \Dy (1+e_1) - \Dx e_2 }{T (1-e^2)} \\
\frac{1}{G} \frac{\partial G}{\partial e_1}
  & = & \frac{e_1}{1-e^2} + \frac{\Dx^2-\Dy^2}{T (1-e^2)^2} (1-e^2 + 2 e_1^2) \\
\frac{1}{G} \frac{\partial G}{\partial e_2}
  & = & \frac{e_2}{1-e^2} + \frac{2 \Dx \Dy}{T (1-e^2)^2} (1-e^2 + 2 e_2^2) \\
\frac{1}{G} \frac{\partial G}{\partial T}
  & = & \frac{-1}{T} \left( 1 - \frac{1}{2} \X^T \M^{-1} \X  \right)  \\
\frac{1}{G} \frac{\partial G}{\partial p}
  & = & \frac{1}{p}
\end{eqnarray}

\subsection{Jacobian in the Presence of a PSF}

For $x_0,y_0,T$ and $p$, we can use the same Jacobian equations for the pre-PSF
case, evaluated with the convolved covariance matrix. 

For $e_1$ and $e_2$ we modify the jacobians according to the following equations
\begin{eqnarray}
\frac{\partial G_o}{\partial e_1} 
 & = & \frac{\partial G_o}{\partial e_1^o} \frac{\partial e_1^o}{\partial e_1} 
    =  R \frac{\partial G_o}{\partial e_1^o}\\
\frac{\partial G_o}{\partial e_2} 
 & = & \frac{\partial G_o}{\partial e_2^o} \frac{\partial e_2^o}{\partial e_2}
    =  R \frac{\partial G_o}{\partial e_2^o}\\
R & = & \frac{T}{T + T_P} = \frac{T}{T_o}.
\end{eqnarray}
where quantities sub- or super-scripted with $o$ are convolved or ``observed''
quantities, and we have defined the resolution parameter $R$.

\section{Examples}

In this section we show some fits to example profiles.

\subsection{Exponential Disk}

An exponential disk can be represented reasonably well by three Gaussians
as shown in Fig. \ref{fig:expcmp}

\begin{figure}[t] \centering
 \centering 
 \includegraphics[scale=1]{figures/test-opt-exp.eps}

 \caption{A fit to an exponential disk using three Gaussians.}
 \label{fig:expcmp}

\end{figure}



\end{document}

