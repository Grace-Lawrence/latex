\documentclass[preprint]{aastex}

\begin{document}

\title{maxBCG}
\author{Erin Sheldon, based on discussions with Jim Annis}

\section{Cluster Likelihood}

We assume that clusters of galaxies have a special galaxy with a known
distribution of colors as a function of redshift, and ``member'' galaxies on
the ES0 ridgeline, nearby in space.  We denote the special galaxy as a
Brightest Cluster Galaxy (BCG). 

We go to each galaxy in the survey and, for a given redshift and metric
aperture, ask how likely the galaxy is to be a BCG and how likely the 
neighbors in the aperture are to be in the ES0 ridgeline. Using Bayes'
theorem:

\begin{equation} \label{eq:posterior}
P(z | m_{BCG}, m_i) = \frac{ P(m_{BCG}, m_i | z) P(z) }{P(m_{BCG}, m_i)},
\end{equation}
where $z$ is the
cluster redshift, $m_{BCG}$ is the measured magnitudes for the BCG candidate,
and $m_i$ is the set of magnitudes for the neighbors in the aperture.

We will take a flat prior on $z$, so all we need to do is maximize the
likelihood function.  We will assume that the BCG likelihood separates from the
member likelihood
\begin{equation} \label{eq:likelihood}
P(m_{BCG}, m_i | z) = P(m_{BCG}|z)P(m_i | z)
\end{equation}
The first term is measured from known BCG's by examining their magnitudes
as a function of redshift.

The second term in equation \ref{eq:likelihood} is the probability that the
collection of galaxies in the aperture are cluster members at that redshift.
Similarly to $P(m_{BCG} | z)$, we can define the likelihood that a given
neighbor galaxy is in the ridgeline, $P(m_{neigh} | z)$.  If there were only
one neighbor, we could define the overall likelihood as $P(m_{BCG} |
z)P(m_{neigh} | z)$.  If there are many neighbors, we define the total neighbor
likelihood as the sum of the individual neighbor likelihoods, normalized such
that the total likelihood is unity if all the neighbors in the aperture are
right on the ridgeline:
\begin{equation} \label{eq:sumlike}
P(m_{BCG}, m_i | z) = P(m_{BCG}|z)\frac{1}{Norm}\sum_i P(m_i | z)
\end{equation}
where $Norm$ is the sum of probabilities evaluated at the peak of the 
ridgeline likelihood.  This is essentially the mean likelihood of the 
neighbors.  

If we define the ridgeline likelihood as simply a box in color-color-magnitude
space, such that $P(m_i | z) = 0$ outside the box and 1 inside, then equation
\ref{eq:sumlike} reduces to
\begin{equation} \label{eq:fraclike}
P(m_{BCG}, m_i | z) = \frac{n_{gals}}{N_{tot}}P(m_{BCG}|z)
\end{equation}
where $N_{tot}$ is the total number of galaxies in the aperture, and $n_{gals}$
is the total within the aperture and also within the box.  Thus, the $n_{gals}$
is not actually fit for, but is a byproduct of maximizing the likelihood. Note
that there will be, in general, a random component to this $n_{gals}$ which is
measurable and should be subtracted out.

If there are a large fraction of galaxies in the aperture randomly, but are not
associated with the cluster, then this could bias the redshift low, since
enlarging the angular aperture will increase this fraction.

This differs only slightly from Jim's likelihood function:
\begin{equation}
P(m_{BCG}, N_{ap} | z, n_{gals}) = n_{gals}(z)P(m_{BCG}|z)
\end{equation}
I don't think that maximizing this function gives the most likely $z$ in
general. This is because, effectively, the normalization of the neighbor
likelihood changes with redshift.  Choosing a lower $z$ results in larger
$n_{gals}$ just because the aperture, in angular space, has grown.  Comparisons
with spectroscopic redshifts do not reveal this bias, perhaps because the BCG
likelihood function is very narrow and because any biases are dominated by
biases in the BCG photoz.  However, its a simple enough change to convert the
likelihood to that in equation \ref{eq:fraclike}.

\section{Discussion}

Maximizing the likelihood in equation \ref{eq:fraclike} should give the best
$z$ for the BCG candidate.  However, $n_{gals} = 2$ objects, even if they are
very likely, must be distinguished from larger clusters, especially because
they may often lie near, or inside, the larger systems.  We need some kind of
ranking system for this.  Probably some combination of $n_{gals}$, the
likelihood from equation \ref{eq:likelihood}, and a measure of the spatial
distribution of members would be useful.  The difficulty is picking the right
way to combine these, both functionally and in terms of relative weight.  

\end{document}
