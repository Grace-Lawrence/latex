% AASGUIDE.TEX -- AASTeX user manual. 
% v5.0.2 03 Nov 99
 
\documentclass[preprint2]{aastex} 
 
\usepackage{aastexug}   % User guide style customizations 
 
\begin{document} 
 
\title{The \aastex\ Package\\for Manuscript Preparation} 
 
\tableofcontents 
 
\section{Introduction} 
 
The American Astronomical Society (AAS) has developed an author 
markup package called \aastex\ that assists authors in preparing 
manuscripts that are intended for submission to the AAS-affiliated 
journals.  \aastex\ can be used for other journals as well, 
some of which accept \aastex\ manuscripts. 
 
The most important aspect of the \aastex\ 
package is that it defines the set of commands (called \emph{markup}) 
that can be used to identify the structural elements of a paper. 
When articles are marked up using this set of standard commands, 
the papers may then be submitted electronically to the editorial 
offices and fed into electronic production of the journals. 
 
This guide contains basic instructions for creating 
manuscripts using the \aastex\ markup package. 
Authors are expected to be familiar with the editorial 
requirements of the journals so that they can make 
appropriate submissions; they should also have at least 
a rudimentary knowledge of \LaTeX\ (for instance, knowing 
how to set up equations using \LaTeX\ commands). 
A number of useful publications about \LaTeX\ (and \TeX) are listed in 
the reference section of this guide. 
 
{\em Authors who wish to submit papers electronically to the ApJ, AJ, or 
PASP are strongly encouraged to use the 
\aastex\ markup package as described in this guide.} 
 
\section{\aastex\ article markup} 
 
This section describes the commands in the \aastex\ 
package that an author might enter in a manuscript that is being 
prepared for 
electronic submission to one of the journals. 
The commands are described in roughly the same order as they 
would appear in a manuscript. 
The reader will also find it helpful to examine the 
sample file (\verb"sample.tex") that is 
distributed with the package. 
Authors are reminded to check the electronic submission procedures and 
the instructions to authors for the journals to which they submit their 
papers. 
 
\subsection{Preamble}\label{sec-preamble} 
 
In \LaTeX\ manuscripts, the preamble is that portion of the file before 
the \verb"\begin{document}"\  command. 
 
\subsubsection{Getting started} 
 
The first piece of markup in the manuscript declares the 
overall style of the document.  Any commands that appear before this 
markup will be ignored. 
\begin{quote} 
\begin{verbatim} 
\documentclass{aastex} 
\end{verbatim} 
\end{quote} 
The above example specifies the document class as 
\texttt{aastex}, 
with the 
default style, |manuscript|, 
in eff\-ect. 
The paper copy produced by this style file will be double-spaced. 
Any tables included in the main body of the manuscript will also be 
double-spaced. 
 
Other styles are available; 
they are discussed in Section~\ref{styles}. 
 
\subsubsection{Defining new commands} 
 
\aastex\ allows authors to define their own commands with 
\LaTeX's \verb"\newcommand". 
(Do not use the plain \TeX\ \verb"\def" command.) 
Authors' \verb"\newcommand" definitions must be placed in the 
document preamble. 
 
In general, author-defined commands that are 
abbreviations or shorthands are acceptable and can be translated, e.g., 
\begin{quote} 
\begin{verbatim} 
\newcommand{\lte}{local thermodynamic 
	equilibrium} 
\end{verbatim} 
\end{quote} 
However, abbreviations that attempt to define new symbols by using 
\LaTeX\ commands for repositioning text or symbols tend to 
cause problems during the production of the article. 
 
Specifically, author-defined commands that use any of the commands 
listed below are apt to be problematic in the conversion 
procedures after the manuscript is submitted. 
\begin{quote} 
\verb"\hskip", \verb"\vskip", 
\verb"\raise", \verb"\raisebox", 
\verb"\lower", \verb"\rlap", 
\verb"\kern", \verb"\lineskip", 
\verb"\char", \verb"\mathchar", 
\verb"\mathcode", \verb"\buildref", 
\verb"\mathrel", \verb"\baselineskip" 
\end{quote} 
Consequently, authors are strongly discouraged from 
employing them. 
 
Extra symbols are defined for \aastex, 
some specifically for an astronomical context, and others more 
broadly used in math and physics. 
In particular, the AMS has additional symbol fonts that are 
available in a standard \LaTeX\ package (\verb"amssymb"). 
All of these symbols are depicted in the additional tables 
supplied with this manual. 
 
Authors are advised to consult these tables to see whether a 
symbol already exists.  If it does, you should just use the 
corresponding markup command; these commands are reserved. 
You should \emph{not} redefine an existing command name. 
When one of these commands is encountered in an electronic 
manuscript, authors' redefinitions are ignored when the 
manuscript is translated. 
 
\subsubsection{Editorial information} 
 
A number of markup commands are available for the 
editorial offices to record the publication history and slug-line data 
for each manuscript. 
These commands are for use in the editorial and production offices only. 
\begin{quote} 
\begin{verbatim} 
\received{<receipt date>} 
\revised{<revision date>} 
\accepted{<acceptance date>} 
 
\ccc{<code>} 
\cpright{<type>}{<year>} 
\end{verbatim} 
\end{quote} 
 
Copyright information is specified through the commands \verb"\cpright"\ 
and \verb"\ccc".  The ``type'' of copyright and the corresponding year 
are given in \verb"\cpright"; valid copyright types are as follows. 
\begin{quote} 
\begin{tabular}{l@{\quad}p{2.8in}} 
\tt AAS & Copyright has been assigned to the AAS\\[.5ex] 
\tt ASP & Copyright has been assigned to the ASP\\[.5ex] 
\tt PD & The article is in the public domain\\[.5ex] 
\tt none & No copyright is claimed for the article 
\end{tabular} 
\end{quote} 
The copyright type is case sensitive, so the type string must be 
entered exactly as given above. 
The Copyright Clearing Center code may be given in the \verb"\ccc"\ 
command; the code is taken as regular text, so any special characters, 
notably `\$', must be properly specified. 
 
\subsubsection{Short comment on title page} 
 
Authors who wish to include a short remark on the title page, 
such as the name and date of the journal in which an article 
has been scheduled, may do so with the following command. 
\begin{quote} 
\begin{verbatim} 
\slugcomment{<text>} 
\end{verbatim} 
\end{quote} 
In the 
\texttt{manuscript} style, 
such comments appear on the title page after the title and authors; 
in the 
\texttt{preprint} style, 
they are placed at the upper right corner 
of the title page. 
 
\subsubsection{Running heads}\label{shorthead} 
 
Authors are invited to supply running head information. 
There are generally two different kinds of data in running heads; 
the left head contains an author list (last 
names, possibly truncated as ``et al.''), while the right head 
is an abbreviated form of the paper title.  This running head 
information will not appear on the \LaTeX-printed page but will be 
transmitted to the copy editor for inclusion in the published version. 
\begin{quote} 
\begin{verbatim} 
\shorttitle{<text>} 
\shortauthors{<text>} 
\end{verbatim} 
\end{quote} 
Editors and publishers impose varying requirements 
on the brevity of these data.  A good rule of thumb is to limit the list 
of authors to three, or use ``et al.'', and to limit the 
short form of the title to 40--45 characters. 
The editors may choose to modify the author-supplied 
running heads. 
 
\subsection{Starting the main body} 
 
None of the markup that appears in the preamble actually typesets 
anything; it is only a control section. 
The author must include a 
\begin{quote} 
\begin{verbatim} 
\begin{document} 
\end{verbatim} 
\end{quote} 
command to identify the beginning of the main textual 
portion of the manuscript. 
 
\subsection{Title and author information} \label{titlepage} 
 
Use the 
\verb"\title"\ 
 and \verb"\author"\  commands to specify title 
and author information. 
Specify the author's principle affiliation with 
the command \verb"\affil". 
Each \verb"\author"\ 
 command 
should be followed by a corresponding \verb"\affil"\ 
and optional \verb"\email"\  command. 
\begin{quote} 
\begin{verbatim} 
\title{<text>} 
\author{<name(s)>} 
\affil{<affiliation>} 
\affil{<address>} 
\email{<email address>} 
\and 
\end{verbatim} 
\end{quote} 
 
Line breaks are permitted in the title if the author wishes 
to specify them with the \verb"\\" command.  Long titles will 
be broken automatically, so the \verb"\\" markup is not required. 
If the title is explicitly broken over several lines, the 
preferred style for titles in AAS and ASP journals is the so-called 
``inverted pyramid'' style.  In this style, the longest line 
is the first (or top) line, and each succeeding line is shorter. 
The text of the title should be entered in mixed case; 
it will be published in upper case or mixed case according to 
the style of the publication to which the paper is submitted. 
Footnotes are permissible in titles; be careful to ensure that 
alternate affiliations (see below) are properly numbered if a 
footnote to the title is specified. 
 
Authors' names are given in \verb"\author"\ 
 commands, 
and should be entered in mixed case. 
Names that appear together in the author list for authors who 
have the same primary affiliation should be specified in a single 
\verb"\author"\  command. 
Each author group (\verb"\author"\  command) 
should be followed by an \verb"\affil"\  command, giving the principle 
affiliation of that author.  Physical and postal address information 
for the specified institution  may be included with the \verb"\affil". 
The address can be broken over several lines using the 
\verb"\\" command to indicate 
the line breaks. 
Usually, however, postal information will fit on one line. 
When there is more than one \verb"\author"\  command, the last 
one should be preceded by the \verb"\and"\  command. 
 
When there is a lengthy author list, all authors' names may be 
specified in a single \verb"\author"\ 
 command, with affiliations 
specified using the \verb"\altaffilmark"\   me\-chan\-ism described below. 
In this case, no \verb"\affil"\ 
 commands would be used, and in print the 
affiliations would all be listed in a footnote block at the bottom 
of the title page. 
 
Authors often have affiliations in addition to their principle employer. 
These alternate affiliations are specified with the \verb"\altaffilmark"\ 
 and \verb"\altaffiltext"\ 
 commands. 
These behave like the \verb"\footnotemark"\ 
and |\footnotetext|  commands of \LaTeX, except there are no 
optional arguments in \aastex. 
\verb"\altaffilmark"\ 
 is appended to authors' names in the \verb"\author"\ 
 list and generates superscript identification numbers. 
The text for the individual alternate affiliations is generated by the 
\verb"\altaffiltext"\ 
 command. 
\begin{quote} 
\begin{verbatim} 
\altaffilmark{<key number(s)>} 
\altaffiltext{<numerical key>}{<text>} 
\end{verbatim} 
\end{quote} 
It is up to the author to make sure that 
 each \m{key number}  in his or her \verb"\altaffilmark"\ 
 matches the \m{numerical key} for 
the corresponding \verb"\altaffiltext". 
 
\subsection{Abstract} 
 
The paper abstract should be enclosed in the 
\texttt{abstract} environment. 
\begin{quote} 
\begin{verbatim} 
\begin{abstract} 
<abstract text> 
\end{abstract} 
\end{verbatim} 
\end{quote} 
 
\subsection{Keywords} 
 
Keywords, subject headings, etc., are accommodated 
as a single piece of text. 
\begin{quote} 
\begin{verbatim} 
\keywords{<text>} 
\end{verbatim} 
\end{quote} 
If authors supply keywords, they must be delimited by whatever 
punctuation is required by the journal. 
They should be specified in alphabetical order. 
The \verb"\keywords"\ 
 command 
supplies the proper leading text (``Keywords:'', ``Subject headings:'', 
etc.), according to journal style. 
 
\subsection{Comments to editors} 
 
Authors may make notes or comments to the copy editor with the 
\verb"\notetoeditor" command. 
\begin{quote} 
\begin{verbatim} 
\notetoeditor{<text>} 
\end{verbatim} 
\end{quote} 
This command behaves like a 
footnote and appears on the bottom of the page. 
\notetoeditor{Notes to the editor only appear 
in the \texttt{manuscript} style.} 
Output to the printed page is produced only in the 
\texttt{manuscript} style. 
 
\subsection{Sections} 
 
The \texttt{manuscript} style for \aastex\ manuscripts supports four levels 
of section headings. 
\begin{quote} 
\begin{verbatim} 
\section{<heading>} 
\subsection{<heading>} 
\subsubsection{<heading>} 
\paragraph{<heading>} 
\end{verbatim} 
\end{quote} 
Section headings should be given in upper case or mixed case depending 
on the style of the journal. 
Note that these commands delimit sections by marking the 
\emph{beginning} of each section; 
there are not separate commands to identify the ends. 
 
\subsection{Figure and table placement}   \label{place} 
 
Generally figures and tables are not ``placed'' in the text of the 
document where an author would like them to appear physically. 
However, authors may indicate to the editors the preferred placement of 
these items using  the \verb"\place*"\  commands. 
\begin{quote} 
\begin{verbatim} 
\placetable{<key>} 
\placefigure{<key>} 
\end{verbatim} 
\end{quote} 
The \verb"\place*{"\m{key}\verb"}" commands are similar to the 
\verb"\ref" 
 command in La\TeX\ 
and require corresponding \verb"\label"\ 
 commands to link them to the 
proper elements. 
 
When used in the 
\texttt{manuscript} style, the \verb"\place*"\  commands will print a short 
message to the editor about figure and table placements. 
In the other styles, nothing is printed. 
 
\subsection{Acknowledgments} 
 
\aastex\ manuscript styles support an acknowledgments section. 
\begin{quote} 
\begin{verbatim} 
\acknowledgments 
\end{verbatim} 
\end{quote} 
In the \aastex\ styles, acknowledgments are set off from the 
concluding main body text by extra vertical space, 
with no heading or type size change. 
 
\subsection{Appendices} 
 
When one or more appendices are needed in a paper, the point where the 
main body text ends and the appendix begins must be marked 
with the \verb"\appendix"\ 
 command. 
\begin{quote} 
\begin{verbatim} 
\section{<body section>} 
\appendix 
\section{<appendix section>} 
\end{verbatim} 
\end{quote} 
The \verb"\appendix"\ 
 command takes care of a number of internal 
housekeeping concerns, such as identifying sections with letters 
instead of numerals, resetting the equation counter, etc. 
Note that the \verb"\appendix"\ 
 command takes no arguments. 
Sections in the appendix must be headed with \verb"\section"\ 
 commands, as described earlier. 
 
\subsection{Equations} 
 
Display equations can be typeset in \LaTeX\ in many ways; 
the following three are probably of greatest use in AASTeX. 
\begin{quote} 
\begin{verbatim} 
\begin{displaymath} 
\end{displaymath} 
 
\begin{equation} 
\end{equation} 
 
\begin{eqnarray} 
\end{eqnarray} 
\end{verbatim} 
\end{quote} 
The 
\texttt{displaymath}  environment 
will break out a single, 
unnumbered formula.  The \texttt{equation} environment does the same 
thing, except that the equation is 
autonumbered by \LaTeX. 
To set several formul\ae\ in which vertical alignment 
is required or to display a long equation across multiple lines, use the 
\texttt{eqnarray} environment. Each line of the \texttt{eqnarray} 
will be numbered 
unless a \verb"\nonumber"\ 
 command is inserted 
before the equation delimiter 
(\verb"\\"). 
\LaTeX's equation counter is \emph{not} incremented when 
\verb"\nonumber"\ 
 is used. 
 
Authors may occasionally wish to group related equations together and 
identify them with letters appended to the same equation number. 
When this is desired, such related equations should still be set 
in \texttt{equation} or \texttt{eqnarray} environments, whichever is 
appropriate, and the grouping should then be placed within 
the \texttt{mathletters} environment. 
\begin{quote} 
\begin{verbatim} 
\begin{mathletters} 
<equation> or <eqnarray> 
\end{mathletters} 
\end{verbatim} 
\end{quote} 
 
It is possible to override \LaTeX's automatic numbering within the 
\texttt{equation} or \texttt{eqnarray} environments using 
\begin{quote} 
\begin{verbatim} 
\eqnum{<text>} 
\end{verbatim} 
\end{quote} 
When \verb"\eqnum"\ 
 is specified inside an \texttt{equation} environment, 
or on a particular equation within an \texttt{eqnarray}, the text 
supplied as an argument to \verb"\eqnum"\ 
 is used as the equation 
identifier. 
\LaTeX's equation counter is \emph{not} incremented when \verb"\eqnum"\ 
 is used. 
\verb"\eqnum"\ 
 must be used \emph{inside} the environment. 
 
If, as a consequence of the use of \verb"\eqnum"\ 
 or \verb"\nonumber", 
\LaTeX's equation counter gets out of synch with the author's 
intended sequence, 
the counter may be reset to a particular value. 
\begin{quote} 
\begin{verbatim} 
\setcounter{equation}{<number>} 
\end{verbatim} 
\end{quote} 
The equation counter should be set to the number corresponding to the 
last equation that was formatted; therefore, it is most appropriate for this 
command to appear immediately after an \texttt{equation} or 
\texttt{eqnarray} environment 
ends. 
The command must be used 
\emph{outside} the math environments. 
 
The \texttt{eqsecnum} style file can also be used to modify the way equations 
are numbered. See Section~\ref{styles} for details. 
 
 
\subsection{Citations and bibliography}   \label{bibliography} 
 
Two options are available for symbolically marking citations 
and formatting reference lists: the standard \LaTeX\ \verb"thebibliography"\ 
environment, and the \aastex\ \verb"references"\  environment. 
Authors are strongly encouraged to use \verb"thebibliography"\ 
in electronic submissions. 
 
Please note that the bibliographic data supplied by the author in 
the reference list must conform to the standards of the journal. 
We have elected not to burden authors with tedious markup commands 
to delimit the bibliographic fields --- many of the journals that 
accept \aastex\ agreed to reduce typographic overhead 
(bold, italic, etc.) in reference lists \citep{Abt90}. 
It is the responsibility of the author to arrange these fields in the 
proper order with the correct punctuation; the information will be 
typeset in roman with no size or style changes. 
 
 
\subsubsection{The {\tt thebibliography} environment}   \label{bib} 
 
The preferred method for reference management is to use \LaTeX's 
\texttt{thebibliography} environment, marking citations in the body 
of the paper with 
\verb"\citep"\ or \verb"\citet"\ 
and associating references with them via \verb"\bibitem". 
The \verb"\cite"-\verb"\bibitem"\ 
mechanism associates citations and references symbolically 
while maintaining proper citation syntax within the paper. 
In the \verb"\bibitem"\ 
command, the author specifies citation data inside 
square brackets and a citation key in curly braces 
for each reference. 
\begin{quote} 
\begin{verbatim} 
\begin{thebibliography}{<dummy>} 
\bibitem[<cite data>]{<key>} <bibliographic data> 
   . 
   . 
\end{thebibliography} 
\end{verbatim} 
\end{quote} 
 
Note that the argument \m{dummy} to the environment is not 
used in the \aastex\ package, but is consistent with the 
syntax of standard \LaTeX. It is acceptable to simply insert an 
empty pair of curly braces after the \verb"\begin{thebibliography}"\ 
command. 
The |\bibitem| command is described in detail in the next section. 
 
 
\subsubsection{Specifying bibliographic and citation information}\label{spec-bib-data} 
 
The scheme used in \aastex\ for marking citations and for specifying 
bibliographic data allow us to utilize the features of the |natbib| 
package \citep{Daly98}. 
The |natbib| package re-implements \LaTeX's |\cite| command, 
and offers greater flexibility for managing citations in the 
author-year form. 
 
All of the bibliographic data are defined in |\bibitem| commands. 
\begin{quote} 
\begin{verbatim} 
\bibitem[<author>(<year>)]{<key>} <bibliographic data> 
\end{verbatim} 
\end{quote} 
The square-bracketed argument of |\bibitem| 
contains the \m{author} portion of the citation, 
followed by the \m{year} set off in parentheses. 
The parentheses are important --- please don't forget them. 
The curly-bracketed argument \m{key} is the code name 
by which the reference is known. 
 
When placing citations in the text, the author should use 
either a |\citep| or a |\citet| command. 
\begin{quote} 
\begin{verbatim} 
\citep[<extra>]{<key(s)>}\\ 
\citet[<extra>]{<key(s)>} 
\end{verbatim} 
\end{quote} 
The |\citep| command produces a citation that is entirely 
set off by parentheses, e.g.\ (Cox 1995), while |\citet| permits 
the author's name to form part of your text, e.g.\ Cox (1995). 
The classic \LaTeX\ |\cite| command behaves like |\citet|. 
 
The citation \m{key} corresponds to a \m{key} in a |\bibitem| command. 
During processing, information from the \m{cite data} is inserted in 
the text at the location of the \verb"\cite"\ command. 
Multiple citation \m{keys} are separated by commas, e.g., 
|\citep|\allowbreak 
|{knuth84,|\allowbreak 
|cox95,|\allowbreak 
|lamport94}|. 
|\citep| and |\citet| each take optional arguments that specify 
\m{extra} text that is appended to the citation label, 
per standard \LaTeX\ behavior. 
 
\aastex\ uses a full |natbib| implementation. 
There are more syntax options than we have described here, 
and these support the whole range of arcane possibilities 
for composing citations; the price is complexity. 
Authors who need more flexibility should become familiar 
with the |natbib| documentation. 
The syntax discussed above should be sufficient for the 
vast majority of cases. 
 
It is not possible to use \verb"\bibitem" 
within \aastex's \texttt{references} environment 
(section~\ref{refenv}), 
nor will \verb"\cite"\ commands work properly in the main body 
if \verb"\bibitem" commands are absent. 
 
 
\subsubsection{The {\tt references} environment}\label{refenv} 
 
Some authors might prefer to enter citations directly 
into the body of an article. If so, the |references| environment may be 
used to format the reference list. The 
\texttt{references}  environment 
simply sets off 
the list of references and adjusts spacing parameters. 
\begin{quote} 
\begin{verbatim} 
\begin{references} 
\reference{<key>}<bibliographic data> 
   . 
   . 
\end{references} 
\end{verbatim} 
\end{quote} 
 
While the 
\texttt{references}  environment 
remains supported in v5.0, 
we anticipate that authors will prefer the stronger capabilities of 
the standard \LaTeX\ |thebibliography| commands as extended by |natbib|. 
 
\subsubsection{Abbreviations for journal names} 
 
There are markup commands for many of the most frequently-referenced journals so  that authors may use the markup  rather than having 
to look up a particular journal's abbreviation. 
In principle, all the journals should be using the 
same abbreviations, but it is fair to anticipate some changes in 
the specific abbreviations before a system is finally settled on.  As 
long as these commands are kept up to date, authors need not be 
concerned about such editorial preferences and changes. 
A listing of the current abbreviations appears in Table~\ref{journame}. 
\begin{table} 
\begin{center} 
\caption{Abbreviations for Journal Names}\label{journame} 
\begin{tabular}{ll} 
\verb"\araa" & Annual Review of Astronomy\\ 
  & \hspace*{1em} and Astrophysics\\ 
\verb"\ao" & Applied Optics\\ 
\verb"\aj" & Astronomical Journal\\ 
\verb"\azh" & Astronomicheskii Zhurnal\\ 
\verb"\aap" & Astronomy and Astrophysics\\ 
\verb"\aapr" & Astronomy and Astrophysics Reviews\\ 
\verb"\apj" & Astrophysical Journal\\ 
\verb"\apjl" & \rule[.5ex]{2em}{.4pt}, Letters to the Editor\\ 
\verb"\apjs" & \rule[.5ex]{2em}{.4pt}, Supplement Series\\ 
\verb"\aplett" & Astrophysics Letters\\ 
\verb"\apspr" & Astrophysics Space Physics Research\\ 
\verb"\apss" & Astrophysics and Space Science\\ 
\verb"\aaps" & \rule[.5ex]{2em}{.4pt}, Supplement Series\\ 
\verb"\baas" & Bulletin of the AAS\\ 
\verb"\bain" & Bulletin Astronomical Inst. Netherlands\\ 
\verb"\gca" & Geochimica Cosmochimica Acta\\ 
\verb"\grl" & Geophysics Research Letters\\ 
\verb"\iaucirc" & IAU Circular\\ 
\verb"\jcp" & Journal of Chemical Physics\\ 
\verb"\jgr" & Journal of Geophysics Research\\ 
\verb"\jrasc" & Journal of the RAS of Canada\\ 
\verb"\memras" & Memoirs of the RAS\\ 
\verb"\mnras" & Monthly Notices of the RAS\\ 
\verb"\nat" & Nature\\ 
\verb"\nphysa" & Nuclear Physics A\\ 
\verb"\physscr" & Physica Scripta\\ 
\verb"\pra" & Physical Review A\\ 
\verb"\prb" & Physical Review B\\ 
\verb"\prc" & Physical Review C\\ 
\verb"\prd" & Physical Review D\\ 
\verb"\pre" & Physical Review E\\ 
\verb"\prl" & Physical Review Letters\\ 
\verb"\physrep" & Physics Reports\\ 
\verb"\planss" & Planetary Space Science\\ 
\verb"\procspie" & Proceedings of the SPIE\\ 
\verb"\pasj" & Publications of the ASJ\\ 
\verb"\pasp" & Publications of the ASP\\ 
\verb"\qjras" & Quarterly Journal of the RAS\\ 
\verb"\skytel" & Sky and Telescope\\ 
\verb"\solphys" & Solar Physics\\ 
\end{tabular} 
\end{center} 
\end{table} 
 
\subsection{Figures} 
 
\subsubsection{Electronic art} 
 
For manuscripts that are submitted electronically, the electronic figures 
themselves are generally sent as individual files to the editors 
and are not included as part of the main text, although the 
legends or captions for these files are included using the markup described 
in Section~\ref{legends}. 
Detailed information on preparing and submitting electronic art is 
available in the submissions instructions for the various journals. 
 
At this time, the most widely used means of including figures or other 
types of nontextual data in electronic manuscripts is to generate such 
files as Encapsulated PostScript files.\footnote{% 
 PostScript is a registered trademark of Adobe Systems Incorporated.% 
}% 
 
If an author wishes PostScript figures to be included in the 
pages being produced, it is necessary that 
the graphics files being included conform 
to the Encapsulated PostScript standard \citep{PLRM}. The 
author must also have an appropriate DVI translator, 
one that targets PostScript output devices. 
 
There are several commands to use when including EPS files 
in \aastex\ manuscripts. They should be placed within  the 
 \texttt{figure} environment. 
\begin{quote} 
\begin{verbatim} 
\begin{figure} 
\figurenum{<text>} 
\epsscale{<num>} 
\plotone{<epsfile>} 
\plottwo{<epsfile>}{<epsfile>} 
\caption{<text>} 
\end{figure} 
\end{verbatim} 
\end{quote} 
When \verb"\figurenum" is specified inside a {\tt figure} environment, 
the text supplied as an argument to \verb"\figurenum" is used as the 
figure identifier. 
\LaTeX's figure counter is \emph{not} incremented when \verb"\figurenum" 
is used. 
\verb"\figurenum" must be used \emph{inside} the {\tt figure} environment. 
 
\verb"\plotone" inserts the graphic in the named EPS file, 
scaled (in both dimensions) so that the horizontal 
dimension fits in the body text width; 
the vertical dimension is scaled to maintain the aspect ratio. 
\verb"\plottwo" inserts two plots next to each other. 
Scale factors are determined automatically from information in the 
EPS file. 
 
The automatic scaling may be overridden with the command 
\verb"\epsscale{<num>}", where \m{num} 
is in decimal units, i.e., 0.80. 
 
%See  ``Using Encapsulated PostScript with AAS\TeX'' on 
%the AAS\TeX\ V5.0 home page, \url{|http://\allowbreak 
%www.aas.\allowbreak 
%org/\allowbreak 
%publications/\allowbreak 
%aastex|}, 
%for information on use of the commands, 
%as well as where to obtain the DVI package. 
 
\subsubsection{Figure captions} \label{legends} 
 
Figure captions, or legends, should be included as a group at the end 
of the manuscript using either the \verb"\figcaption"\  command 
or a \texttt{figure} environment for each figure in 
the paper, whether or not electronic art is included. 
For electronic submissions, use of \verb"\figcaption"\ is preferred. 
 
\begin{quote} 
\begin{verbatim} 
\figcaption[<filename>]{<text>\label{<key>}} 
\end{verbatim} 
\end{quote} 
 
The optional argument, \m{filename}, can be used to iden\-ti\-fy 
the file for the corresponding figure; 
\m{text} refers to the caption for that figure.  The author 
may provide a \verb"\label"\ with a unique \m{key} for cross-referencing 
purposes. 
 
When the \verb"\figcaption"\  command is used, the figure 
identification, 
e.g., ``Figure 1,'' is generated automatically by the command itself, so 
there is no need to key this information. 
 There is an upper limit of seven figure captions per page. 
Footnotes are \emph{not} supported for figures. 
 
\subsection{Tables}  \label{tables} 
 
There is support in the \aastex\ package for tables via two mechanisms: 
\LaTeX's standard \texttt{table} environment 
and the 
\texttt{deluxetable} environment, which facilitates the formatting 
of lengthy tabular material.  Tables may be 
marked up using either mechanism, although use of 
\texttt{deluxetable} is preferred. 
Authors should \emph{not} use the \LaTeX\ 
\texttt{tabbing} environment 
or the \verb"\hline"\ 
 command when preparing electronic 
submissions. 
 
\LaTeX\ permits the preparation of fairly complex tables with 
arbitrary spacing, straddle heads, rules, and the like. 
Authors who need to specify complicated column headings and 
so forth are advised to consult the \LaTeX\ manual \citep{Lamport}. 
Most of the capabilities outlined there are applicable to  the \aastex\ 
 table preparation environments. 
 
\subsubsection{The {\tt deluxetable} environment}  \label{dte} 
 
Authors are encouraged to use the 
\texttt{deluxetable} environment to format their tables since it automatically 
handles many formatting tasks, including table numbering and insertion 
of horizontal rules. It also provides mechanisms for breaking tables and 
controlling width and vertical spacing that are unavailable in the \LaTeX\ 
\texttt{tabular} environment. 
 
The \texttt{deluxetable} environment 
is delimited by \LaTeX's familiar 
\verb"\begin"\  and \verb"\end"\  constructs. 
The content consists of \m{preamble} commands and \m{table data}, 
the latter delimited by \verb"\startdata"\ 
 and \verb"\enddata". 
\begin{quote} 
\begin{verbatim} 
\begin{deluxetable}{<cols>} 
<preamble commands> 
\startdata 
<table data> 
\enddata 
\end{deluxetable} 
\end{verbatim} 
\end{quote} 
where \m{cols} specifies the justification for each col\-umn. 
One of the letters `l', `c', `r', or `p' is given for each column, 
indicating flush left, centered, flush right, or justified text. 
Authors are 
referred to the \LaTeX\ manual \citep{Lamport} for further information. 
 
\subsubsection{Preamble to the {\tt deluxetable}} 
 
There are several items in the 
\texttt{deluxetable} environment 
that 
must be given before the data for the table; 
they constitute the \m{preamble} to the environment. 
\begin{quote} 
\begin{verbatim} 
\tabletypesize{<font size command>}
\rotate 
\tablewidth{<dimen>} 
\tablenum{<text>} 
\tablecolumns{<num>} 
\tableheadfrac{<num>} 
\tablecaption{<text>\label{<key>}} 
\tablehead{<text>} 
\end{verbatim} 
\end{quote} 
 
If a table is too wide for the printed page, it is permissible to change 
the font size within the 
\texttt{deluxetable} environment 
using the \verb"\tabletypesize" command with the font size change commands
\verb"\small"\ 
 (11pt), \verb"\footnotesize"\ 
 (10pt), or  
\verb"\scriptsize"\ 
 (8pt). 
 
You can force a table to be set in landscape orientation 
using the \verb"\rotate"\ 
\label{cmd-rotate} command. 
The default is for \aastex\ itself to determine if landscape orientation is required. 
 
The width of a deluxetable is defined by \verb"\tablewidth"; 
the default width, if this command is omitted, is the width of the body 
text. 
The table can be set to its natural width by specifying 
a \m{dimen} of \verb"0pt". 
Long tables may have a natural width that is 
different for each page.  The natural width for each page will be 
printed to the log file during processing; 
authors may then define a fixed table width based 
on this information, giving the 
tables a more uniform appearance across the pages. 
 
It is possible to override \LaTeX's automatic numbering within the 
\texttt{deluxetable} environment. 
When \verb"\tablenum"\ 
 is specified inside a \texttt{deluxetable} 
preamble, 
the text supplied as an argument to \verb"\tablenum"\ 
 is used as the 
table identifier. 
\LaTeX's equation counter is \emph{not} incremented when 
\verb"\tablenum"\ 
 is used. 
 
The caption (actually, the title) of the table is specified 
in \verb"\tablecaption". 
The text of \verb"\tablecaption"\ 
 should be brief; 
explanatory notes should be specified in the end notes to the table 
(see \verb"\tablecomments"\ 
 below).  If the caption 
does not appear 
centered above the table after processing, then specify the width of 
the table explicitly in the \verb"\tablewidth"\ 
 command and rerun 
\LaTeX\ on the file.   The author 
may provide a \verb"\label"\ in the caption with a unique \m{key} 
for cross-referencing 
purposes. 
 
  Column headings are specified with \verb"\tablehead". 
Within \verb"\tablehead", each column heading can be given 
in a \verb"\colhead", which will ensure that the heading 
is centered 
on the natural width of the column; this is the typical disposition 
of column headings, and the use of \verb"\colhead"\ 
 is encouraged. 
There should be a heading for each column, so that there are as 
many \verb"\colhead"\ 
 commands in the \verb"\tablehead"\ 
 as there 
are data columns. 
 
\begin{quote} 
\begin{verbatim} 
\tablehead{ 
\colhead{<heading>} & \colhead{<heading>}} 
\end{verbatim} 
\end{quote} 
 
If more complicated column headings are required, 
any valid \texttt{tabular} command that constitutes a proper 
head line in a \LaTeX\ table may be used. 
Consult the \LaTeX\ manual \citep{Lamport} for details. 
 
The \verb"\tablecolumns{"\m{num}\verb"}" command 
is necessary if the author has 
multi-line column headings produced by \verb"\tablehead"\ 
 or other \LaTeX\ 
commands and is using either the \verb"\cutinhead"\ 
 or \verb"\sidehead"\ 
markup (see below).  The \m{num} is 
set to the true number of columns in the 
table.  The command must come before the \verb"\startdata"\ 
 command. 
 
It is possible that a complicated table heading will overflow 
the vertical space allotted for the table heading. 
The fraction of the page allocated 
for the table heading may be changed with \verb"\tableheadfrac". 
The \m{num} argument to \verb"\tableheadfrac"\ 
 should be the 
decimal fraction of the page used for heading information. 
The default value is 0.1, meaning that 10\% of the page height 
is reserved for the table heading.  It should rarely be necessary 
to change this value. 
 
\subsubsection{Content of the {\tt deluxetable}} 
 
After the table title and column headings are specified, 
data rows can be entered. 
Data rows are delimited with the \verb"\startdata"\ 
 and \verb"\enddata"\ 
 commands. 
The end of each row is indicated with the standard \LaTeX\ \verb"\\" command. 
Data cells within a row of the table are separated with \& (ampersand) characters. 
\begin{quote} 
\begin{verbatim} 
\startdata 
<data line>\\ 
<cell>&<cell>&<cell>\\ 
<more data lines> 
\enddata 
\end{verbatim} 
\end{quote} 
 
Column alignment within the data columns can be adjusted with the \TeX\ 
\verb"\phantom{"\m{string}\verb"}" command, 
where \m{string} can be any character, e.g., \verb"\phantom{$\arcmin$}". 
A blank character of width \m{string} is then inserted in the table. 
Four commands have been predefined for this purpose. 
\begin{quote} 
\begin{tabular}{l@{\quad}p{2in}} 
{\verb"\phn"% 
} & {phantom numeral 0-9}\\ 
{\verb"\phd"% 
} & {phantom decimal point}\\ 
{\verb"\phs"% 
} & {phantom $\pm$ sign}\\ 
{\verb"\phm{"\m{string}\verb"}"} & {generic phantom}\\ 
\end{tabular} 
\end{quote} 
 
Extra vertical space can be inserted between rows with an optional argument to the 
\verb"\\" command; the argument is a dimension 
and may be specified in any units that are legitimate in \LaTeX. 
\begin{quote} 
\begin{verbatim} 
\\[<dimen>] 
\end{verbatim} 
\end{quote} 
 
In a table, it may happen that several rows of data are 
associated with a single object or item; 
such logical groupings should not be broken across pages. 
In such cases, the *-form of the |\\| command can be used to 
keep the current line with the next line (``Keep with Next''). 
\begin{quote} 
\begin{verbatim} 
<table row>\\* 
<next table row> 
\end{verbatim} 
\end{quote} 
An unlimited number of table rows can be held together in a block in this way. 
 
The journals often require that table cells that contain  no data 
be explicitly marked.  This is to differentiate such cells from 
blank cells, which are frequently interpreted as implicitly 
repeating the entry in the corresponding cell in the row preceding. 
Data elements for which there are no data should contain 
a \verb"\nodata"\ 
 command; an appropriate symbol will be placed in 
that data element. 
\begin{quote} 
\begin{verbatim} 
\nodata 
\end{verbatim} 
\end{quote} 
 
Within the deluxetable body, two kinds of ``specialty'' heads are 
recognized.  A cut-in head is a piece of text that is centered on the 
table width; 
it is spaced above and below from the data rows that precede and 
follow it; there may be rules associated with it, depending on the 
journal or manuscript style.  All of these formatting particulars 
are managed by the style files.  The author need only specify that the 
text be centered with a \verb"\cutinhead"\ 
 command. 
Similarly, a side head is a piece of text that is left-justified. 
 
\begin{quote} 
\begin{verbatim} 
\cutinhead{<text>} 
\sidehead{<text>} 
\end{verbatim} 
\end{quote} 
 
Table footnotes (more properly, table \emph{endnotes}) 
may be used in the \texttt{deluxetable} environment; 
their use is described in detail in the Section 
 \ref{tabfoot} below. 
 
\subsubsection{The {\tt table} environment} 
 
Tables may also be marked and composed using the \texttt{table} environment. 
\begin{quote} 
\begin{verbatim} 
\begin{table} 
\end{table} 
\end{verbatim} 
\end{quote} 
The 
\texttt{table}\ environment 
encloses not only the tabular 
material but also any title (caption) or footnote information 
associated with the table. 
 
Titles or captions for tables are indicated with 
\begin{quote} 
\begin{verbatim} 
\caption{<text>\label{<key>}} 
\end{verbatim} 
\end{quote} 
Tables will be identified with arabic numerals, e.g., ``Table 2''; 
the identifying text, including the number, is generated automatically 
by \verb"\caption". 
 The author 
may provide a \verb"\label"\ in the caption with a unique \m{key} 
for cross-referencing 
purposes. 
 
Tabular information is typeset within the 
\texttt{tabular} environment: 
\begin{quote} 
\begin{verbatim} 
\begin{tabular}{<cols>} 
\end{tabular} 
\end{verbatim} 
\end{quote} 
where \m{cols} specifies the justification for each column. 
One of the letters `l', `c', or `r' is given for each column, 
indicating left, center, or right justification. 
Consult the \LaTeX\ manual \citep{Lamport} for details about using 
the 
\texttt{tabular} environment 
to prepare tables. 
Each \texttt{tabular} table must appear within a \texttt{table} 
environment. There should be only one \texttt{tabular} table per 
\texttt{table} 
environment. 
If the journal requests manuscripts with only one table per page, 
the author may need to insert a \verb"\clearpage"\ 
 command after 
especially short tables. 
 
There is a table line command for use in \texttt{tabular} 
environments. 
\begin{quote} 
\verb"\tableline" 
 \end{quote} 
This command produces the horizontal rule(s) between the column headings 
and the body of the table. 
Authors are discouraged from using \verb"\hline". 
In addition, the use of vertical rules should be avoided. 
 
As with the \texttt{deluxetable} environment, it 
 is possible to override \LaTeX's automatic numbering within the 
\texttt{table} environment using \verb"\tablenum". 
\verb"\tablenum"\ 
 must be used \emph{inside} the \texttt{table} 
environment. 
 
 
 
\subsubsection{Table footnotes} \label{tabfoot} 
 
\aastex\ supports footnotes (endnotes) that are associated with tables; 
this support applies to both the 
\texttt{deluxetable}  environment 
and the standard \LaTeX\ \texttt{table} environment. 
Footnotes for tables are usually identified by lowercase letters 
rather than num\-bers. 
Marking and assigning associated text is achieved with 
the \verb"\tablenotemark"\ 
 and \verb"\tablenotetext"\ 
 commands, which function like the  \verb"\altaffilmark"\ 
 and \verb"\altaffiltext"\ 
 (see Section~\ref{titlepage} above). 
The \verb"\tablenote"\-|text| 
 must be specified 
\emph{after} the \verb"\end{tabular}"\  or \verb"\enddata"\  and before 
the enclosing \verb"\end{table}"\  or \verb"\end{deluxetable}". 
\begin{quote} 
\begin{verbatim} 
\tablenotemark{<alpha key>} 
\tablenotetext{<alpha key>}{<text>} 
\end{verbatim} 
\end{quote} 
Note that the \m{alpha key} in the \verb"\tablenotemark"\ 
should be the same as the 
\m{alpha key} for the corresponding \m{text}. 
It is the responsibility of the author to make the correspondence 
correct. 
 
Sometimes authors tabulate materials which have corresponding 
references, 
and it may be desirable to associate these references with the table 
rather than (or in addition to) the formal reference list. 
Also, authors may occasionally wish to append a 
short paragraph of explanatory 
notes that pertain to the entire table, but which are different than 
the caption. \aastex\ offers two special kinds of endnotes for these 
purposes. 
\begin{quote} 
\begin{verbatim} 
\tablerefs{<reference list>} 
\tablecomments{<text>} 
\end{verbatim} 
\end{quote} 
 
Like the \verb"\tablenotetext" command, these commands should be 
 specified after the \verb"\end{tabular}" or \verb"\enddata" and before 
the enclosing \verb"\end{table}" or \verb"\end{deluxetable}". 
 
\subsection{Miscellaneous} 
 
\subsubsection{Ionic species and chemical bonds} 
 
When discussing atomic species, ionization levels can be indicated 
with the following command. 
\begin{quote} 
\begin{verbatim} 
\ion{<element>}{<level>} 
\end{verbatim} 
\end{quote} 
The ionization state is specified as the second argument, 
and should be given as a numeral. 
For example, \ion{Ca}{3} is specified by typing \verb"\ion{Ca}{3}". 
 
For single, double, and triple chemical bonds, use the following macros. 
\begin{quote} 
\begin{verbatim} 
\sbond 
\dbond 
\tbond 
\end{verbatim} 
\end{quote} 
 
 
 
 
\subsubsection{Fractions} 
 
\aastex\ contains commands that permit authors to specify alternate 
forms for fractions. 
\LaTeX\ will set fractions in displayed math as built-up fractions; 
however, it is sometimes desirable to use case fractions in 
displayed equations. 
In such instances, one should use \verb"\case"\ 
rather than \verb"\frac", although authors submitting 
manuscripts electronically 
will generally find it unnecessary to use any markup other than the 
standard \LaTeX\ \verb"\frac". 
 
\begin{center} 
\renewcommand{\arraystretch}{1.4} 
\begin{tabular}{@{}llc@{}} 
Built-up & \verb"\frac{1}{2}" & $\displaystyle\frac{1}{2}$ \\[.5ex] 
Case     & \verb"\case{1}{2}" & $\case{1}{2}$ \\ 
Shilled  & \verb"1/2" & $1/2$ \\ 
\end{tabular} 
\end{center} 
 
The \aastex\ package also contains a collection of assorted macros 
for symbols and abbreviations specific to an astronomical context. 
These are commonly useful and also somewhat difficult for authors 
to produce themselves because fussy kerning is required. 
See the \anchor{symbolapp.pdf}{symbols pages} provided as 
a separate attachment with this manual. 
Most of these commands can be used in both running text and math. 
\verb"\lesssim"\  and \verb"\gtrsim"\  can only be used in math mode. 
 
\subsubsection{Celestial objects} 
 
Celestial objects should be marked by the author to facilitate 
post-processing of articles for the astronomical catalog databases. 
\begin{quote} 
\begin{verbatim} 
\objectname[<catalog ID>]{<text>} 
\end{verbatim} 
\end{quote} 
This command has no effect in printed manuscripts. 
When your article contains |\objectname| commands, 
the publisher can pass along a list of ``objects in article'' 
to database personnel, and software can construct queries 
to those databases in the online editions. 
 
In the future, it may be useful for the author to specify a 
particular catalog with an identifier. 
At this time, however, the optional \verb"<catalog ID>" 
argument is ignored, and need not be supplied. 
 
\subsubsection{Hypertext constructs} 
 
The \verb"\anchor"\ \label{cmd-anchor} command is a general-purpose 
hypertext link tag, associating \verb"<text>" in the manuscript with 
the specified resource (\verb"<href>"). 
\begin{quote} 
\begin{verbatim} 
\anchor{<href>}{<text>} 
\end{verbatim} 
\end{quote} 
\verb"<href>"\ is specified as a \emph{full} URI, including the 
\verb"scheme:"\  designator (|http:|, |ftp:|, etc.). 
 
The \verb"\url" command supports the special case where an author 
wishes to express a URL in the text. 
\begin{quote} 
\begin{verbatim} 
\url{<text>} 
\end{verbatim} 
\end{quote} 
 
The \verb"\email"\  command is used to identify email addresses 
anywhere in the manuscript. 
The argument is the pertinent email address; 
please do \emph{not} prepend the \verb|mailto:| part. 
\begin{quote} 
\begin{verbatim} 
\email{<address>} 
\end{verbatim} 
\end{quote} 
This command should be used to indicate authors' email addresses 
in author lists at the beginning of manuscripts. 
The \aastex\ v4.0 command, \verb"\authoremail", is superseded 
by \verb"\email". 
 
\subsection{Concluding the file} 
 
The last command in the electronic manuscript file should be the 
\begin{quote} 
\begin{verbatim} 
\end{document} 
\end{verbatim} 
\end{quote} 
command, which appears after all the material in the paper. 
This command directs the formatter to perform assorted termination 
activities and finish processing. 
 
\section{Style options}   \label{styles} 
 
The default style option is the |manuscript| style. This style will 
produce double-spaced pages printed in a single column at the width 
of the page. 
 
Two preprint styles are also available. 
The \texttt{preprint} style offers a single-spaced alternative for 
manuscripts. 
This style is similar to the |manuscript| style, 
but it is single-spaced, and it is possible to use a smaller typeface. 
Sometimes, it is desirable for preprints 
to be set in two columns with running heads. 
The \texttt{preprint2} style option achieves this. 
These two style options are discussed further below. 
 
The \texttt{eqsecnum} style file can be used to modify the way equations 
are numbered. 
Normally, equations are numbered sequentially through the 
entire paper, starting over at ``(A1)'' if there is an appendix. 
If \texttt{eqsecnum} appears in the \texttt{documentclass} command, 
equation num\-bers will be sequential through each section and will be 
formatted ``(sec-eqn),'' where ``sec'' is the current section number and 
``eqn'' is the number of the equation within that section. 
\begin{quote} 
\begin{verbatim} 
\documentclass[eqsecnum]{aastex} 
\end{verbatim} 
\end{quote} 
 
A \verb"flushrt.sty"\  file option is available for authors that prefer to 
have their margins left and right justified. 
The 
\texttt{preprint2} style 
is flush right by default, 
but the 
\texttt{manuscript} and 
\texttt{preprint} styles are ragged right by default. 
\begin{quote} 
\begin{verbatim} 
\documentclass[eqsecnum,flushrt]{aastex} 
\end{verbatim} 
\end{quote} 
 
\subsection{Preprint format} 
 
A single-column preprint format can be specified with the 
\texttt{preprint} style option: 
\begin{quote} 
\begin{verbatim} 
\documentclass[preprint]{aastex} 
\end{verbatim} 
\end{quote} 
The size of the typeface used is under author control by way of 
\LaTeX's \m{nn}\texttt{pt} class options 
(where \m{nn} is 10, 11, or 12). 
Use of 10 point type is not recommended with the 
\texttt{preprint} style. 
 
Authors may wish to adjust vertical spacing within a preprint, 
for instance, double-spacing text while single-spacing tables. 
Authors who want to alternate between single and double 
spacing in the manuscript may use the following commands. 
\begin{quote} 
\begin{verbatim} 
\singlespace 
\doublespace 
\end{verbatim} 
\end{quote} 
\verb"\singlespace"\  sets the vertical spacing to a smaller value, 
while \verb"\doublespace"\  invokes double-spacing. 
 
\subsection{Two-column format} 
 
The 
\texttt{preprint2} style 
has the principle function of providing two-column formatting. 
\begin{quote} 
\begin{verbatim} 
\documentclass[preprint2]{aastex} 
\end{verbatim} 
\end{quote} 
It is important to remember 
that text lines are considerably shorter when two of them are typeset 
side by side on a page.  Long equations, wide tables and figures, and 
the like, may not typeset in this format without some adjustments. 
The expenditure of great effort to adapt copy and markup for 
two-column pages is counterproductive. 
Remember that the main 
goal of this package is to produce draft 
(or referee) format pages; 
it is the responsibility of the editors and publishers to 
produce publication-format papers for the journals. 
 
The 
\texttt{preprint2} style 
sets the article's front matter---the title, author, abstract, 
and keyword material---on a separate page at full text width. 
The body of the article is set in a two-column page grid, 
the appendices in a one-column page grid, 
and the bibliography in a two-column page grid. 
This manual was prepared using the \texttt{preprint2} style. 
 
The author may supply \LaTeX's \verb"\twocolumn"\  or |\onecolumn| 
commands whenever desired. 
Be aware, however, that using explicit column-switching commands can 
cause formatting problems. 
 
\section{Additional documentation}  \label{docs} 
 
The preceding explanation of the markup commands in the 
\aastex\ package has merit for defining syntax, but many 
authors will prefer to examine the sample papers that are 
included with the style files. 
The files of interest are described below. 
 
A comprehensive example employing nearly all of the capabilities 
of the package (in terms of markup as well as formatting) 
is in \verb"sample.tex". 
This file is annotated with comments that describe 
the purpose of most of the markup. 
\verb"sample.tex"\ has three tables: two marked up using the 
\texttt{deluxetable} environment 
and another table using the \LaTeX\ \texttt{table} 
environment. 
 
In \verb"table.tex", a complex but short example of the 
\texttt{deluxetable} environment 
demonstrates some of the techniques 
that can be used to generate complex column headings and to align 
variable-width columns. 
Here the \LaTeX\ \verb"\multicolumn" command is used to span a heading 
over several columns.  When \verb"\multicolumn" is used along with the 
\verb"\cutinhead" command, the \verb"\tablecolumns" command must be used to specify 
the number of columns in the table---otherwise the \verb"\cutinhead" command 
will not work properly. 
This table also makes use of the \verb"\phn" command to better align some of the 
columns. 
 
This user guide (\verb"aasguide.tex") 
is also marked up with the \aastex\ package, 
although it is not exemplary as a scientific paper. 
 
Many of the markup commands described in the preceding 
sections are standard \LaTeX\ commands. The reader who is 
unfamiliar with their syntax is referred to the \LaTeX\ 
manual \citep{Lamport}, or any of several other guides 
by \citet{Kopka99}, \citet{Goossens94}, and \citet{Hahn93}. 
 
Authors who wish to know the ins and outs of \TeX\ itself 
should read the \emph{\TeX book} \citep{Knuth}, probably more than once. 
There is a good deal of information about typography in general 
in this resource.  Many details of mathematical typography are 
discussed in \emph{Mathematics into Type} by \citet{Swanson}. 
 
\section{Acknowledgments} 
 
\aastex\ was designed and written by Chris Biemesderfer in 1988. 
Substantial revisions were made by Lee Brotzman and Pierre 
Landau when the package was updated to v4.0. 
\aastex\ was rewritten as a \LaTeXe\ class by Arthur Ogawa 
when v5.0 was created. 
The documentation has benefited from revisions by 
Jeannette Barnes and Sara Zimmerman. 
 
\begin{thebibliography}{} 
 
\bibitem[Abt(1990)]{Abt90} 
  Abt, H.  1990, \apj, 357, 1 (editorial) 
 
\bibitem[Adobe(1999)]{PLRM} 
  Adobe Systems, Inc.  1999, 
  \anchor{http://www.adobe.com/prodindex/postscript/} 
  {PostScript Language Reference Manual} (Reading, MA: Addison-Wesley) 
 
\bibitem[Daly(1998)]{Daly98} 
  Daly, P. 1998, \emph{Natural Sciences Citations and References} 
  (\texttt{natbib} package documentation) 
 
\bibitem[Goossens, Mittelbach, and Samarin(1994)]{Goossens94} 
  Goossens, M., Mittelbach, F., and Samarin, A. 1994, 
  \emph{The \LaTeX\ Companion} (Reading, MA: Addison-Wesley) 
 
\bibitem[Hahn(1993)]{Hahn93} 
  Hahn, J. 1993,  \emph{\LaTeX\ for Everyone} 
  (Englewood Cliffs, NJ: Prentice-Hall) 
 
\bibitem[Knuth(1984)]{Knuth} 
  Knuth, D. 1984, \emph{The \TeX book} (Reading, MA: Addison-Wesley). 
  \newblock Revised to cover \TeX3, 1991. 
 
\bibitem[Kopka and Daly(1999)]{Kopka99} 
  Kopka, H. and Daly, P. 1999,  \emph{A Guide to \LaTeX}, 
  3rd edition, (Reading, MA: Addison-Wesley) 
 
\bibitem[Lamport(1994)]{Lamport} 
  Lamport, L. 1994, \emph{\LaTeX: A Document Preparation System\/}, 
  2nd edition, (Reading, MA: Addison-Wesley) 
 
\bibitem[Swanson(1979)]{Swanson} 
  Swanson, E. 1979, \emph{Mathematics into Type} 
  (Providence, RI: American Mathematical Society) 
 
\end{thebibliography} 
 
\end{document} 
