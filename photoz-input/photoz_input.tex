%\documentclass{emulateapj}

\documentclass[preprint]{aastex}
%\documentclass{aastex}

\usepackage{verbatim}

\slugcomment{Last revision \today}

\def \epssmall{0.7}

\shortauthors{Sheldon}
\shorttitle{Photoz Input Catalogs}




\begin{document}

\title{Inputs to the UChicago Photoz Pipeline}

\author{
Erin S. Sheldon\altaffilmark{1}}

\altaffiltext{1}{Center for Cosmology and Particle Physics,
New York University, New York, NY 10003, USA; erin.sheldon@gmail.com}

\begin{abstract}

\end{abstract}
The catalog was drawn from the object catalogs produced at FNAL with a few
additional quantities calculated to improve the star galaxy separation.  
These include the PSF probability \texttt{objc\_prob\_psf} and the lensing 
smear polarizability \texttt{m\_r\_h}.  The probabilities were cut
at a galaxy probability greater than 0.8 which is very stringent, and
smear polarizability less than 0.8. Further cuts on magnitude were made.
For the list of runs and reruns see 
\newline
http://cheops1.uchicago.edu/weaklens/photoz/photoz\_input\_runlist.dat

\begin{verbatim}
The cuts:

  counts_model[2] != -9999    &&
  objc_prob_psf   >=  0.0     &&
  objc_prob_psf   <   0.2     &&
  (
    (m_r[1] > 0.0 && m_r[1] < 0.8) || 
    (m_r[2] > 0.0 && m_r[2] < 0.8) || 
    (m_r[3] > 0.0 && m_r[3] < 0.8) || 
  )                           &&
  cmodel_counts[2] < 22.0     &&
  cmodel_counts[2] > 14.0     &&
  counts_model[2]  < 22.5     &&
  counts_model[2]  > 13.0

The tags in the input catalogs:
  taglist = ['run','rerun','camcol','field','id','ra','dec', 'primtarget', 'objc_type', 
             'counts_model', 'counts_modelerr', 'reddening', 'objc_prob_psf']
\end{verbatim}

\end{document}
