\documentclass[12pt]{article}

\usepackage[margin=1in,dvips]{geometry}

% for comments
\usepackage{verbatim}

\begin{document}

\title{Measuring Dark Energy with Gravitational Lensing in the Dark Energy Survey}
\author{Erin Sheldon, Assistant Physicist\\
Brookhaven National Laboratory\\
(631) 344-3117, erin.sheldon@gmail.com\\
Year Doctorate Awarded: 2002\\
Program Announcement Number: LAB 12-751}

\date{}
\maketitle

\section{Introduction}
I propose to analyze data from the Dark Energy Survey (DES) to constrain the
properties of Dark Energy.  My primary focus will be on measuring gravitational
lensing effects to probe the expansion history and growth rate of massive
structures in our universe.  I am a DES member with data rights and am co-leading 
development of the DES lensing software pipeline.

Dark Energy accelerates the expansion of the universe, dramatically increasing
the volume in comparison to a matter-only universe.  Dark Energy also inhibits
the growth of massive structures under gravitational collapse.  Thus the number
density of massive objects such as galaxy clusters as a function of cosmic time
is directly related to the properties of Dark Energy, in particular the
equation of state parameter $w=$pressure/density.  Critical to using the number
density to constrain cosmology is knowing the masses of the clusters. We will
measure these masses using gravitational lensing.  Using cluster counts,
lensing and other complimentary probes, the DES will measure $w$ to $\sim$3\%.
This program is a natural continuation of my earlier measurements of lensing in
the Sloan Digital Sky Survey (SDSS), which are most sensitive such measurements
to date.

DES will see first light in the fall of 2012.  I will spend the intervening
time finalizing our data reduction pipelines, which we strenuously test using
realistic simulations.  I am developing a new code to measure shear using
gaussian mixtures which I have tested on simulations and will apply to the
data.  In collaboration with Mike Jarvis at the University of Pennsylvania, I
am also developing a second code using the method of ``shapelets''.  These
codes run in a processing framework and queuing system I have developed. 

After first light, DES will take data for five years, during which I will
process the data as it arrives and perform analysis to extract Dark Energy
parameters.  These analyses will most likely produce results in two stages: an
early set of results from the first year data and a second set of results from
the full data set after year five.

To achieve these goals, I need the assistance of postdocs and significant
computing infrastructure.

\section{Resources}

The DES survey will produce of order a petabyte of data which will arrive over
the course of five years.  I am co-leading the development of the lensing
software pipelines.  These pipelines will process all the images produced by
the DES to measure the detailed shapes of galaxies for use in lensing analyses.
Fortunately, the basic reduced images are a relatively small fraction of the
total data set, about 130 terabytes.

Processing this data data as it arrives will require significant manpower and
computing resources.  The NSF funded DES data management (DESDM) is committed
to produce a single processing of the data per year.  Thus, although our
lensing pipelines will be incorporated into DESDM, all development and testing
must occur outside of DESDM.  

In order to test our pipelines and analysis codes we must process the full data
set multiple times.  This is because many systematics tests require essentially
the full data set to explore.  And only from analyzing the full data set to
extract the lensing signal will we be able to feed back what we have learned
into the pipelines.  Working with a single processing would be highly
restrictive.  The processing requires very high memory machines and large
permanent data storage, which precludes use of existing computing resources
such as NERSC. Finally, computing from other experiments at Brookhaven is not
available for use.  Thus we must have separate, although quite moderate,
computing resources to facilitate the data processing.

Based on current hardware and codes, it would take about 2.5 weeks to process
the final DES data on a moderate 70 node cluster with 8 cores per node.  This
is approximately the desired processing time given that we want to reprocess
the data with several iterations with two or more independent shape measurement
algorithms in order to obtain a robust result.

We will also require significant resources for the science analysis, which is
computationally intensive when performed optimally, although this should be
sub-dominant to the image processing.

A moderate compute cluster of 70 nodes the required disk will cost about
\$55,000 per year over the course of this five year award.  This number
includes a compounded Moore's law, and is entirely dominated by CPU rather than
storage costs.  The power, cooling, and maintenence, which is significant, will
be provided free of charge by the RHIC Computing Facility at BNL.

\section{Personnel}

As required for applicants at national laboratories, this award will cover my
own salary, including overhead.  I will hire a postdoc to work on testing and
development of the pipelines and DES data, as well as analysis of the results
to constrain the properties of Dark Energy.  These analyses will focus on
measuring lensing effects to constrain the masses of galaxy clusters and
interpretation in terms of cosmology.  This is a primary goal of the DES survey
with high scientific impact:  most of the constraining power on Dark Energy
comes from lensing and the counts of galaxy clusters at a given mass threshold.  

\end{document}
