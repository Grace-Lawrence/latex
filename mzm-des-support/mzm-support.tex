\documentclass[12pt]{letter}

\begin{document}


\hfill August 23,  2011
\newline

Dear Committee -

I am writing in support of Zhaomin Ma's application for personal DES data
rights.  

Zhaoming joined me as a postdoc at Brookhaven National Laboratory in the Fall
of 2009.  Prior to arriving at BNL, Zhaoming worked on DES forecasts, amounting
to about two months full time work.  He has spent the last fourteen months,
since June 2010, working exclusively on DES infrastructure.  This work has been
focused on two major areas, the first is optimization of the DES observing
strategy, and the second is development of a nearly optimal method to
interpolate the point spread function of the sky+optical system.  Zhaoming has
already detailed his work in his proposal.  Here I will try to address the
importance of his contributions, and how the DES collaboration will benefit
from Zhaoming's continued involvement.

Zhaoming's work on cosmological parameter forecasts were included in the DES
science proposal.  Although part of a large effort, the parameter forecasts are
an important focus of that document.  Indeed the forecasts are essential to
making the science case and justifying the effort as a whole.

Zhaoming's work on survey strategy focuses on finding a set of observing
strategies that maximize the dark energy figure of merit.  Zhaoming has worked
primarily with Jim Annis and Carlos Cunha on this.  Jim has developed
monte-carlo simulations of the survey that account for weather, moon
brightness, seeing, etc. as they vary throughout the year, essentially all the
external conditions.  An ansats is chosen for dealing with various choices that
arise, such as when to change filters, when to slew the telescope, how long to
expose etc.  Zhaoming's contribution is to calculate the impact these decisions
have on the weak lensing and BAO figure of merit.  For example, he questions if
it is better to expose all filters equally to improve photometric redshift
accuracy, or if it is better to favor the i-band in hopes to increase the
sensitivity of shear measurements from that band.   This work is essential to
optimizing the DES survey for dark energy constraints.  Zhaoming will continue
to work on this problem through commissioning, but I anticipate more work will
be needed after the survey proper starts, as we take data and discover the most
efficient modes of operation.  Zhaoming has spent approximately 2 months full
time work on this.

Zhoaming has spent about a year full time working on a near-optimal method to
interpolate the point spread function (PSF).  This is quite a technical
subject, but is of great importance to DES science, so I will give some
detailed description here.

The effect of the atmosphere, optics, ccd, etc. can be mathematically expresses
as a convolution of the image with a kernel, the PSF.  This convolution alters
galaxy shapes causing spurious signals in measurements such as gravitational
lensing.  This effect must be removed with high accuracy if we are to meet our
science goals.  In order to remove the effect we must estimate the PSF at the
location of all galaxies.

The PSF can only be measured by using true point sources, the most abundant of
which are stars.  The stars chosen must be high signal-to-noise ratio (S/N);
this is because, in the convolution, high order moments in the PSF affect low
order moments in the galaxy.  High S/N is needed to extract this information.
Furthermore, the PSF varies with position in the focal plane.  But these
sources are fairly low density in comparison to the galaxies.   There is not an
abundance of good reference PSF stars near every galaxy that could be used for
determining a correction.  Instead the PSF must be interpolated to the location
of the galaxy.  But even so, the density of stars is so low that the PSF cannot
be interpolated accurately on the small scales, or high frequencies, at which the
PSF varies. 

But many of the PSF patterns repeat throughout the survey.  For example, PSF
patterns due to defocus or mirror flexure will repeat, and stars will most
likely be located at different points in the focal plane each time.  As
proposed by Jarvis \& Jain, the repeats can be used to essentially provide a
higher density stellar sample.  The ensemble of exposures, sampling each
physical configuration many times, can be used to extract the principle
components of these PSF patterns.  The required number of these components is
not large, of order 20.  There are easily enough stars on each exposure to
determine these coefficients.  

Zhaoming has developed a novel, model independent method for performing this
PCA analysis.  Rather than fitting a model such as a polynomial the PSF as a
function of position in the focal plane, he grids the focal plane and uses data
from stars in each grid cell as his data vector.  He then finds the
eigen-vectors in this space.  This allows for any form of spatial variation,
and is limited only by the grid spacing, the scale of which is only limited by
the number of exposures available for PCA determination.  This is facilitated
by Zhaoming's algorithm, which deals robustly with missing data (empty cells),
and so quite small cells, and thus high spatial frequencies, can be recovered.
Using simulations, Zhaoming has shown this method is an order of magnitude more
accurate that the existing polynomial-based interpolation scheme, and should be
accurate enough to meet our survey goals.  Furthermore, the code is fully
parallelized, and can scale to the large data set of DES.

Zhaoming's work on PCA PSF interpolation directly facilitates the overall
lensing pipeline.  Without the PCA the pipeline could not achieve the desired
accuracy. As work on the lensing pipeline is itself infrastructure, I think
this should translate directly to the PCA work.  

If Zhaoming receives data rights, he will continue developing and perfecting
this PCA code throughout the survey.  He will be responsible for running this
code as data arrives to update his PCA solution to be used in the lensing
analysis.  In other words, Zhaoming is committed to continuing his
infrastructure contribution for the next five years.

Zhaoming's own scientific interests, which he has explained in detail, are
primarily lensing and cosmological parameter estimation.  A successful pursuit
of these interests depends directly on the accuracy of the shear estimates and
thus on his PSF interpolations.  So so he has a strong motivation to continue
this infrastructure work as he moves on in his career.

Thank you for your consideration,

{\noindent Erin Sheldon}
\newline
{\noindent Brookhaven National Laboratory}


\end{document}
