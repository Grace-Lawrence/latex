\documentclass[12pt]{letter}

\addtolength{\textwidth}{0.5in}
\addtolength{\hoffset}{-0.5in}
\addtolength{\textheight}{0.5in}
\addtolength{\voffset}{-0.5in}

\begin{document}


\hfill September 22,  2011
\newline

To Whom It May Concern-

I am writing to give my full support to Dr. Zhaoming Ma's application for
permanent residency in the United States of America.

Dr. Ma joined me as a Postdoctoral Researcher at Brookhaven National Laboratory
in the Fall of 2009.  Since that time we have worked on a new astronomical
instrument to study the phenomena of Dark Matter and Dark Energy in our
universe.  These phenomena are perhaps the greatest mysteries of our time, and
Dr. Ma has dedicated his career to their investigation.

I have a PhD in astrophysics from the University of Michigan, and am currently
an associate physicist at Brookhaven National Laboratory.  I have worked my
entire career to illuminate the Dark Matter and Dark Energy mysteries.  I
brought Dr. Ma to Brookhaven to collaborate in the study of these phenomena,
and to help build the Dark Energy Survey, of which I will say more below.

I will first say a bit about Dark Matter and Dark Energy, in order to clarify
my description of Dr. Ma's work.  Dark Matter is essentially a name for the
enormous amount of matter in our universe that is unseen.  We don't know that
this matter is, but we know it exists because of its gravitational influence on
our galaxy and other distant galaxies.  Just as we know the mass of the Sun by
studying the orbits of the planets, we can infer the mass of galaxies from the
orbits of stars in those galaxies.  There is far more unseen mass than visible
mass, and we call this mass ``Dark''.  Similarly, Dark Energy is a name for a
phenomenon, in this case the inexplicable acceleration of our universe. We have
known for almost a century that the galaxies are all moving apart, but we
expect gravity to pull inward and slow them down.  Instead, some unknown energy
is driving galaxies in our universe apart at an ever increasing rate.  Again
we cannot see the cause, so we name it ``Dark'' energy.

The Dark Energy Survey is a new experiment to study these phenomena, funded
primarily by the United States.  We have built a new camera, the largest CCD
camera in the world, to survey the sky and look for more evidence.  The camera
is being installed as I write this letter, and Dr. Ma and I will work
throughout the winter and spring to test this camera, and prepare for survey
operations next year.

Dr. Ma's work on the Dark Energy Survey has fallen into two broad categories:
so-called ``gravitational lensing'' measurements and survey design, which I
will describe in turn.  Dr. Ma has a also done much other work outside
of this survey, please see his own letter for details.

Dr. Ma has spent most of his time on gravitational lensing measurements.
Gravitational lensing is a beautiful phenomenon where the light from distant
background galaxies gets ``lensed'' by closer foreground galaxies as it makes
its way to us.  Much as the image of a word in a book is enlarged and distorted
by a magnifying glass, the image of a background galaxy is magnified and
distorted by the presence of another foreground galaxy.  This magnification and
distortion occurs because the light is deflected as it passes through the
gravity well of the foreground galaxy.  And much like the magnifying glass, the
amount of distortion depends on how strong the lens is and how close or far it
is from your eye.  In gravitational lensing, the ``strength'' of the lens is
just the mass of the foreground galaxy.

Gravitational lensing is the most powerful method for the study of Dark Matter
and Dark Energy.  Because the strength of a gravitational lens is just its
mass, we use lensing to weigh the Dark Matter in galaxies.  Similarly, because
the lens effect depends on how far away the lens is, we can use the inferred
lens magnification or distortion to tell us how far apart galaxies are over the
history of the universe, and thus see how much Dark Energy has driven galaxies
apart.

The difficulty is that the telescope has its own astigmatism, and this also
distorts images.  This effect must be understood if we are to learn about Dark
Energy.  To make matters worse, it turns out there is not enough information in
an image to determine this effect.  But Dr. Ma has developed the best method in
the world to deal with this lack of information.  He characterizes the patterns
of the effect by studying all the images in a survey, which reduces greatly the
range of possibilities, and makes determination of the effect possible in a
single image.  He just finds which pattern matches.  This is similar to how the
best analysts study the stock market; only by studying the entire history can
one understand why the market acts as it does on a particular day.  Dr. Ma's
method is crucial to the success of the survey, and he will employ it
continuously as the survey progresses to ensure that we can learn as much as
possible about Dark Energy and Dark Matter.

Dr. Ma also studied which survey design would give the most useful information
about Dark Energy.  For example, he calculated the trade-off between exposing
images for a very long time or taking many more shorter exposures; the
trade-off between taking color images as opposed to monochrome; and how and
when to look at certain parts of the sky as opposed to others.  These decisions
critically impact the quality of the survey, and Dr. Ma's contribution is
recognized within the Dark Energy Survey as crucial for the success of this
\$35M project.

Finally, when the data is in hand, Dr. Ma will analyze the lensing effects
to determine the properties of Dark Energy and Dark Matter.

I expect Dr. Ma to continue his important work and contribute equally to other
large future US experiments.  For example, he has already begun to apply his
methods to the new Large Synoptic Survey Telescope (\$800M). 

In summary, Dr. Ma has made critical contributions to the Dark Energy Survey, a
new, primarily US experiment to study the mysterious Dark Energy and Dark
Matter.  In addition to his numerous other contributions to science, he has
developed the worlds best technique for characterizing certain spurious
patterns in these data, and he will employ these techniques in the coming years
to guarantee we get the most out of this and other experiments.     I think Dr.
Ma will greatly enrich the future of US science, and I strongly support his
application for permanent residency. 

{\noindent Thank you for your consideration,}
\newline
\newline
{\noindent Erin Sheldon}
\newline
{\noindent Brookhaven National Laboratory}


\end{document}
