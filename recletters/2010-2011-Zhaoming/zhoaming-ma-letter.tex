\documentclass[12pt]{letter}
\usepackage{nopageno}

\address{
Erin Sheldon\\
Bldg 510\\
Brookhaven National Laboratory\\
Upton NY, 11973\\
}

\signature{Erin Sheldon}

\begin{document}
\begin{letter}{}

\opening{Dear Committee,}

I am writing to recommend Zhaoming Ma for your position in Cosmology.

Zhoaming's interests lie primarily in Cosmology, with a particular interest in
gravitational lensing and large scale structure.  He has worked in a number of
areas, including analysis of quasar spectra, numerical n-body simulations,
cosmological phenomenology, and recently characterization of detailed
systematic effects in astronomical instruments.

Zhaoming's technical expertise is numerical analysis. He is very good with
algorithms and relishes digging into the details.  Much of his work has
involved applying numerical techniques to astronomical data and simulations.

I'll go briefly into Zhaoming's previous work before talking about our work
together. For Zhaoming's thesis work, he modified an n-body code to allow
arbitrary power spectra and a dark energy equation of state, allowing detailed
studies and predictions in our modern view of the universe. This is highly
relevant now as cosmological probes become more precise and we can begin to
probe these fundamental physical aspects of our universe.  It was also a
technical challenge that he met readily. 

Zhaoming developed a code to find the eigenspectra of quasars using principle
component analysis.  These spectra are highly complex, and not entirely
uniform, but PCA gives useful information.  I single out this study because
Zhaoming's knowledge of this technique has been important for our work.

Zhaoming has performed a number of useful phenomenological studies related to
gravitational lensing, in particular the prediction of systematic errors and
their impact on the inference of cosmological parameters.  The largest
uncertainty in interpreting lensing measurements is the distances to the lensed
background source galaxies.  Zhaoming has led a number of studies of how these
uncertainties translate into errors in cosmological parameter estimation, and
isolated the effect of different sorts of errors, such as so called
``catastrophic'' errors.  He then wrote a paper detailing just how much prior
information we need to characterize and overcome these systematics.  These
studies have been highly useful in planning for upcoming surveys such as the
Dark Energy Survey (DES). For DES we we will need large spectroscopic follow-up
surveys to measure the distances to a fair sample of ``training set'' galaxies.
These training sets will be used to estimate, in a statistical sense, the
distances to the source galaxies and correct for these systematics.

The second most important systematic for lensing is the smearing of images by
the point spread function (PSF) of the telescope and instrument.  In weak
lensing measurements, we use the shapes of objects to infer the gravitational
shear, so any alteration of shapes by the instrument must be accounted for.
Zhaoming studies how well this function can be determined and corrected for in
space based telescopes.  He found that, in a well designed space based mission,
the PSF can be characterized to sufficient accuracy.  As I will describe below,
Zhaoming's interest in PSF characterization is now being applied to real world
situations in DES.


Zhoaming Zhaoming has been working with me as a postdoc at Brookhaven since the
fall of 2009.  His research focus has been in two areas, which I will describe
in turn.  Zhoaming's first project he began while at UPenn. He is developing a
new technique to find massive objects in shear data.  His approach is to look
at the population of such objects in the data as a while rather than just one
at a time.  The algorithm is iterative and uses the multi-dimensional data at
each step and the bayesian evidence at each step to develop an internal
understanding of the full population of objects and produce statistically
meaningful detection thresholds.  This has been a sobering exercise; we have
found that robust detections are extremely difficult even in the absence of
systematics, and that perhaps thinking of ``objects'' in these raw shear maps
may not be the right approach at all.  This work is very nearly complete and
will appear shortly.

Zhaoming's other large project is to develop a code to characterize the point
spread function (PSF) in astronomical images.  This work is directed mainly at
DES.  I am a member of DES and am developing the software pipeline to measure
the shapes of objects for lensing.  Zhaoming's earlier study of space missions
showed that measuring the PSF was quite tractable, but on the ground the
problem is much more difficult.  The PSF is less well resolved by the detectors
and the physical conditions are highly variable, e.g. the mirror can flex under
gravity as the telescope moves around.  It turns out there are two few point
like objects (stars) in the images to properly characterize the PSF across the
focal plane.  Zhaoming is using is knowledge ofQ principle component analysis
to attack this problem.  Most of the causes for the PSF will repeat over time:
every time the camera is out of focus a certain amount, or the telescope points
in a certain direction, the pattern of the PSF will repeat. We can use all the
exposures from the survey to characterize the principle components of these
patterns.  Then only a few parameters per image need be fit in order to
reconstruct the PSF pattern.  This knowledge can then be used to correct
the shape measurements before use in measuring the lensing effects.

Zhaoming began this project in the summer and it has progressed rapidly, he now
has a working code.  Zhaoming will implement this on a large set of simulated
DES data for testing, and apply it to real DES data as it arrives this fall. 

The project to measure the PSF in DES is of utmost importance to the project,
as most science goals rely on accurate lensing measurements.  For this core
infrastructure work, we expect Zhaoming will be elected as an associate member
and carry DES data rights wherever he works.

I have found Zhaoming to be an excellent researcher and a good colleague.  He
is very methodical in his approach, and has consistently produced meaningful
results in important areas of research.  He has also followed his own research
program since he came to BNL.  His lensing cluster finder was his own project,
not guided by me or others.  And, although we both wanted him to work on DES at
BNL, he chose the PSF characterization project, and has implemented it almost
entirely on his own.  I have advised mainly from a high level, and to give
technical details of DES data and image processing.  Zhaoming will implement
this on the DES data in coming years and use the results to produce science.
Zhaoming is also planning to be involved in the spectroscopic follow-up 
to characterize the redshfit distributions of our lensing source galaxies,
as this is directly informed by his previous studies.  He will play a lead
role in that planning.



\closing{Sincerely, }

\end{letter}
\end{document}

