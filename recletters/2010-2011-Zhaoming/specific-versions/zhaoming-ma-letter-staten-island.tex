\documentclass[12pt]{letter}
\usepackage{nopageno}
\usepackage{verbatim}

\address{
Erin Scott Sheldon\\
Bldg 510\\
Brookhaven National Laboratory\\
Upton NY, 11973\\
}

\signature{Erin Scott Sheldon}

\begin{document}
\begin{letter}{}

\opening{Dear Committee,}

I am writing to recommend Zhaoming Ma for your position in Cosmology. Zhaoming
has been a postdoc working with me at Brookhaven National Laboratory since the
Fall of 2009.

Zhoaming's scientific interests lie primarily in Cosmology, with a particular
focus on gravitational lensing and large scale structure.  Zhaoming's technical
expertise is numerical analysis. He is very good with algorithms and relishes
digging into the details.  Much of his work has involved applying numerical
techniques to astronomical data and simulations.

He has worked in a number of areas, including analysis of quasar spectra,
numerical n-body simulations, cosmological phenomenology, and recently
characterization of detailed systematic effects in astronomical instruments.  I
will describe briefly his work related to gravitaional lensing, as it is most
closely related to our work together.

\begin{comment}

I'll go briefly into Zhaoming's previous work before talking about our work
together. For Zhaoming's thesis work, he modified an n-body code to use
arbitrary power spectra and a dark energy equation of state, facilitating
detailed studies and predictions in our modern view of the universe. This is
highly relevant now as cosmological probes become more precise and we can begin
to probe these fundamental physical aspects of our universe.  It was also a
technical challenge that he met readily. 

Zhaoming developed a code to find the eigenspectra of quasars using principle
component analysis.  These spectra are highly complex, and not entirely
uniform, but PCA gives useful classification information.  I single out this
study because Zhaoming's knowledge of this technique has been important for our
work.
\end{comment}

Zhaoming has performed a number of useful phenomenological studies related to
gravitational lensing, in particular the prediction of systematic errors and
their impact on the inference of cosmological parameters.  The largest
uncertainty in interpreting lensing measurements is the distances to the lensed
background source galaxies. Just as with ordinary lenses, interpretation of
lens strength requires knowledge of the geometry.  Zhaoming has led a number of
studies of how these uncertainties translate into errors in cosmological
parameter estimation. He also isolated the effect of different sorts of errors,
such as so called ``catastrophic'' errors.  He then wrote a paper detailing
just how much prior information we need to characterize and overcome these
systematics.  These studies have been highly useful in planning for upcoming
surveys such as the Dark Energy Survey (DES). For DES we will need large
spectroscopic follow-up surveys to measure the distances to a fair sample of
``training set'' galaxies.  These training sets will be used to estimate, in a
statistical sense, the distances to the source galaxies. This statistical
estimation is called ``photometric redshifts''.

The second most important systematic for lensing is the blurring of images by
the point spread function (PSF) of the telescope and instrument.  In weak
lensing measurements, we use the shapes of background sources to infer the
gravitational shear induced by a foreground lens, so any alteration of shapes
by the instrument must be accounted for.  Zhaoming studied how well this
function can be determined and corrected for in space based telescopes.  He
found that, in a well designed space based mission, the PSF can be
characterized to sufficient accuracy.  As I will describe below, Zhaoming's
interest in PSF characterization is now being applied to real world situations
in DES.


At BNL, Zhaoming has researched two areas, which I will describe in turn.
Zhoaming began the first project while at UPenn. He is evaluating techniques to
find massive objects solely by their gravitational lensing signal.  He has
implemented a number of different algorithms, including traditional model
fitting. He has also implemented a technique called MultiNest which uses the
multi-dimensional data and the bayesian evidence at each step to develop an
internal understanding of the full population of objects. This allows Zhaoming
to use statistically meaningful detection thresholds.  Although his algorithms
work correctly, this has been a sobering exercise. We have found that robust
detections are extremely difficult even in the absence of systematics, and that
perhaps thinking of ``objects'' in these data may not be the right approach at
all.  This work is very nearly complete and will appear shortly.

Zhaoming's other large project is to develop a code to characterize the point
spread function (PSF) in astronomical images.  This work is directed mainly at
DES.  I am a member of DES and am developing the software pipeline to measure
the shapes of objects for lensing.  Zhaoming's earlier study of space missions
showed that measuring the PSF was quite tractable, but on the ground the
problem is much more difficult.  The PSF is less well resolved by the detectors
and the physical conditions are highly variable, e.g. the mirror can flex under
gravity as the telescope moves around.  It turns out there are two few point
like objects (stars) in the images to properly characterize the PSF across the
focal plane.  Zhaoming is using his knowledge of principle component analysis
to attack this problem.  Most of the physical conditions that produce the PSF
will repeat over time: every time the camera is out of focus a certain amount,
or the telescope points in a certain direction, the pattern of the PSF will
repeat. We can use all the exposures from the survey to characterize the
principle components of these patterns.  Then only a few parameters per image
need be fit in order to reconstruct the PSF pattern.  This knowledge can then
be used to correct the shapes before use in measuring the lensing effects.

Zhaoming has a working code and will implement this on a large set of realistic
simulated DES data for further testing. He will apply it to real DES data as it
arrives this fall.  He will then participate in the interpretation of the
lensing measurements to deduce cosmological parameters.

The characterization of the PSF in DES is of utmost importance to the project,
as most science goals rely on accurate lensing measurements.  For this core
infrastructure work, we expect Zhaoming will be elected as an associate member
and carry DES data rights for himself, students and a postdoc wherever his
career takes him.

I have found Zhaoming to be an excellent researcher and a good colleague.  He
is very methodical in his approach, and has consistently produced meaningful
results in important areas of research.  He has followed his own research
program since he came to BNL.  His lensing cluster finder was his own project,
not guided by me or others.  And, although we both wanted him to work on DES at
BNL, he chose the PSF characterization project, and has implemented it almost
entirely on his own.  I have advised mainly from a high level, and to give
technical details of DES data and image processing.  Zhaoming will implement
this on the DES data in coming years and use the results to produce science.
Zhaoming is also planning to be involved in the spectroscopic follow-up to
characterize the redshfit distributions of our lensing source galaxies, as this
is directly informed by his previous studies on photometric redshifts.  He will
play a lead role in that planning.

Zhaoming also has an interest in teaching.  He has been involved in outreach at
BNL working separately on these projects on his own time.  The first was a
lecture given to local elementary students on interesting topics in cosmology.
The second was a talk about our cosmology program to a group of community
college students.  I will leave him to describe this work in more detail if he
wishes.

In short, I think Zhoaming will make an excellent colleague and will have a
productive career as a researcher and teacher.  By becoming a member of DES, he
has secured a steady flow of data for analysis over the next five years. And
his membership brings data rights for students and a postdoc, which provides an
excellent start to his research program.

\closing{Sincerely, }

\end{letter}
\end{document}

