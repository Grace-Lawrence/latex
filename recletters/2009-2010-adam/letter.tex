\documentclass[12pt]{letter}
\usepackage{nopageno}

\address{
Erin Sheldon\\
Bldg 510\\
Brookhaven National Laboratory\\
Upton NY, 11973\\
}

\signature{Erin Sheldon}

\begin{document}
\begin{letter}{}

\opening{Dear Committee,}

I'm writing to recommend Adam Myers for your position.  

I know Adam through our work over the last year on the Baryon Oscillation
Spectroscopic Survey (BOSS).  The primary mission of BOSS is to study the
baryon acoustic feature in the distribution of galaxies and gas in the
universe.  This feature is ``frozen'' at the time of recombination, and is
easily identifiable in the Cosmic Microwave Background.  The relative size of
this ``standard ruler'' at the time of recombination in comparison to any later
time is directly related to the expansion history of the universe, and thus the
basic parameters of our cosmological model.  This feature was first found in
the distribution of luminous red galaxies in Eisenstein et al. 2005, and BOSS
is an extension of that study to higher redshift, and thus back in time.

Using imaging from the Sloan Digital Sky Survey (SDSS), we identify interesting
objects that we target for follow-up study using spectroscopy.  I'm in charge
of this target selection.  The main focus is on two types of objects, luminous
red galaxies as were used in the Eisenstein study, and quasars, which are
Adam's specialty.  For galaxies, the primary aim of the spectroscopy is to get
redshifts.  The idea with quasars is to look at the absorption lines from gas
along the line of sight between the quasar and us.  Due to the
redshift-distance relation, the distribution of these absorption lines in
wavelength is related to spatial distribution of the underlying density field.
These absorption lines thus retain the impression of the acoustic feature.  The
information content is similar to that in the spatial distribution of the
galaxies, but quasars probe a much higher redshift, significantly extending our
``lever arm''.

I am not an expert on quasars, so when I took over the target selection I
needed an expert to help integrate the quasar algorithms into my pipeline.
Adam filled that role.  

There are a number of quasar groups within BOSS, and much independent work was
done on quasar selection algorithms.  In the end four different primary
algorithms were identified for finding quasars.  At first, Adam worked with
Gordon Richards on a kernel density estimation (KDE) technique.  The use of
more than one algorithm may seem odd at first, but this is a non-trivial
problem because, at the redshifts of interest, the colors of quasars are nearly
indistinguishable from that of stars.  Different algorithms seem to have
different perspectives into the nature of quasars, and are able find different
subsets of the quasar sample among the background of stars.  

When I took over, none of these algorithms were integrated into a pipeline to
dig through the hundreds of millions of galaxies and stars and root out the
likely quasars.  I needed to have a code that could incorporate all the varied
algorithms in a sensible way.  Adam first started working to help me
incorporate the KDE algorithm, but quickly became fluent in the other
algorithms as well.  He recognized the complimentarity of KDE with the simpler
chi-squared method and combined them into a single method to define what became
known as the ``core'' sample.  

As the BOSS commissioning approached last August, there were many last-minute
tasks that had to be completed. This was partly because the imaging reductions
for the relevant region of sky were just becoming ready for use in target
selection.  Also, it became clear that quasar target selection was more
complicated than anticipated, and was going to need much fine tuning in order
to reduce stellar contamination in the sample.  During the last week before
commissioning, Adam and I spent many 20-hour days working to get quasar target
selection ready, trying to balance the need to meet the deadline with the need
for improved control over the number density of targets.  Without Adam, the
quasar target selection for commissioning would have been much less optimal.

This process has been repeated twice since that time:  a ``second
commissioning'' run in October, and the first of the ``main survey'' runs that
we just finished last week.  

The main survey run was the most difficult yet.  We were just starting to get
results from the previous commissioning spectroscopic runs, and we were trying
to incorporate this new knowledge into the pipeline in order to improve the
quasar selection algorithms.  The relevant imaging data were, again, only
slowly becoming available.  This effort would not have been nearly as
successful without Adam.  There was a barrage of new ideas and changes to the
algorithms as the spectroscopic data was being internalized by the group, and
there simply would not have been enough time to do all the work myself.  Only
Adam stepped up and worked the long days and nights necessary to meet the
drop-dead deadline.

I'm not that familiar with Adam's scientific work outside of BOSS, so I won't
try to speak to that.  I imagine that his other letter writers can address that
better.   What I can say is very complimentary to that, however.  Adam showed
me that he is willing to do whatever is necessary to make a project work.  He
seems to feed off the pressure, and do his best work under difficult
conditions.  He has that quality I think is just as important as brilliance: he
{\it cares} about the work he does.

\closing{Sincerely, }

\end{letter}
\end{document}

