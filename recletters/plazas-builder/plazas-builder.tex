\documentclass[12pt]{letter}
\usepackage{nopageno}
\usepackage{verbatim}

\usepackage[absolute]{textpos}
\usepackage{graphicx}

\addtolength{\textwidth}{1.0in}
\addtolength{\hoffset}{-0.5in}
\addtolength{\textheight}{1.0in}
\addtolength{\voffset}{-0.5in}


% to make bibliography work in letter
\makeatletter
\newenvironment{thebibliography}[1]
     {\list{\@biblabel{\@arabic\c@enumiv}}%
           {\settowidth\labelwidth{\@biblabel{#1}}%
            \leftmargin\labelwidth
            \advance\leftmargin\labelsep
            \usecounter{enumiv}%
            \let\p@enumiv\@empty
            \renewcommand\theenumiv{\@arabic\c@enumiv}}%
      \sloppy
      \clubpenalty4000
      \@clubpenalty \clubpenalty
      \widowpenalty4000%
      \sfcode`\.\@m}
     {\def\@noitemerr
       {\@latex@warning{Empty `thebibliography' environment}}%
      \endlist}
\newcommand\newblock{\hskip .11em\@plus.33em\@minus.07em}

\makeatother

\address{
Erin Scott Sheldon\\
Bldg 510\\
Brookhaven National Laboratory\\
Upton NY, 11973\\
}

\signature{Erin Scott Sheldon\\Brookhaven National Laboratory}

\begin{document}
\begin{letter}{}

\opening{Dear Committee,}

I am writing to nominate Andres Plazas for builder status.  Andres has spent
more than four years working on DES infrastructure tasks as specified in document
docdb: 5247. His work is of particular importance, improving our understanding
of the CCDs and the signatures they imprint on the images.  Indeed his work
can be used to correct for some of the most important CCD-related effects,
and help to meet our basic requirements for astrometric and photometric accuracy.

I have summarized his infrastructure work in following table.

\begin{tabular}{|l|l|l|l|}
    \hline
    Dates              & Location & Activity                         & Time (FTE-year) \\ \hline \hline
    Jun.-Oct. 2006     & FNAL     & CCD Testing \& Characterization  & 0.375 \\
    Jun.-Aug. 2007     & FNAL     & CCD Testing \& Characterization  & 0.25  \\
    Jun.-Aug. 2008     & FNAL     & CCD Testing \& Characterization  & 0.25  \\
    \hline
    2009-2011          & UPenn    & Weak Lensing Pipeline Testing    & 1.5 \\
    \hline
    2011-2012          & UPenn    & Astrometry \& Photometry Testing & 0.4 \\
    2012-2014          & BNL      & Astrometry \& Photometry Testing & 1.0 \\
    \hline
    Sep. 2013          & CTIO     & DES Observer \& Run Manager      & 0.4 \\
    \hline 
    Sep. 2013          & BNL      & Eyeball Squad                    & ? \\
    \hline 
    Fall 2013          & BNL      & Exposure Checker                 & ? \\
    \hline
    \hline
    Total              &          &                                  & 4.175 \\
    \hline
\end{tabular}

I intentionally left empty the time field for Andres' work on the Eyeball Squad
and Exposure checker since I do not know how to convert his time to FTE-year.
Andres spent a week working with the Eyeball Squad and a approximately a week
with the Exposure Checker, in total.

Andres' work on CCD testing was performed at FNAL under the supervision of J.
Estrada, T. Diehl, B. Flaugher, D. Kubik, H. Cease., T. Shaw (FNAL), G.
Bernstein, B. Jain (Penn), and J. P. Negret (Universidad de los Andes, Bogot ́).
His work involved the measurement and validation of crosstalk, full-well level,
non-linearity, persistence (validation of the ERASE mechanism), diffusion, and
charge transfer inefficiency.  Andres has contributed to a number of DES
related publications and internal DES documents regarding these measurements
\cite{Diehl2008,Flaugher2010,Flaugher2012}; Doc-db: 207, 795, 1921, 1968, 2043.

Andres' work at U. Pennsylvania was supervised by Gary Bernstein.  This work
involved testing of the ``shapelets'' pipeline developed by Jarvis and Sheldon.
In particular Andres tested the recovery of the point spread function (PSF) and
shear estimation in the presence of various systematic effects.  Andres mapped
out the dependence of the shear measurement accuracy as a function of
signal-to-noise ratio and galaxy type.  He discovered a number of interesting
failure modes that can plague any method, such as dependence on initial guess
for the fitter. His work has been important for demonstrating the efficacy of
using shapelets for PSF measurement, as well as indicating that shapelets are
not a good choice for shear measurements (the lensing working group has since
moved to alternative methods).  He also began his work measuring astrometric
residuals while at UPenn.  Andres' work on shear measurement testing
comprised the majority of his PhD thesis.

Andres' work at Brookhaven National Laboratory (BNL) is supervised by Erin
Sheldon.  Andres characterized the astrometric and photometric residuals in the
DES science verification data. The most important of these features are the
``tree-rings'' and ``glowing edges''. He has successfully modeled the tree-ring
effect assuming they are caused by lateral electric fields associated with
non-uniform doping during the manufacturing
process\cite{PlazasProceedings2014,Plazas2014}.  These models imply the
features are due to pixel size variations rather than quantum efficiency
variations, and thus the flat fielding procedure must be altered.  He
demonstrated that these features can be predicted from ``templates'' derived
from the dome flats.  These templates will be used to refine the astrometric
solution in DES images, helping to meet the DES specifications for photometric
and astrometric accuracy.

In summary, Andres has contributed greatly to DES through a number of different
channels.  His work has clarified our understanding of our CCDs and will be
used to improve the most basic data products of our survey:  the location and
brightness of the astronomical objects we study.

\closing{Sincerely, }


\bibliographystyle{unsrt}
\bibliography{astroref}


\end{letter}
\end{document}

