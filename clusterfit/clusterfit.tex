
\documentclass[preprint]{aastex}

\begin{document}

\title{Fits to Weak Lensing Shear Profiles}
\author{Erin Scott Sheldon}
\affil{Department of Physics, University of Michigan, 500 East University, Ann
Arbor, Mi 48109}
\email{esheldon@umich.edu}

\begin{abstract}
\end{abstract}

\keywords{}

\section{Introduction}

One of the ultimate goals of weak lensing analyses is to infer the masses of
objects, such as galaxies or clusters of galaxies. This can be accomplished
directly by producing a mass map of the regions surrounding the object.  
However, this map is extremely noisy unless it is smoothed \citep{KS93}.  The
introduction of a smoothing function complicates the inferred density profile.

Due to these limitations, masses are often inferred from the shear profile
induced in background galaxies.  Because the shear measures a density contrast
(see \S \ref{shearmeas}), these masses can only be inferred by fitting a 
model to the shear.  The fitting of such models is the topic of this paper.

\section{Shear Measurements} \label{shearmeas}

The shear due to a mass distribution with projected surface density
$\Sigma$ is given by:
\begin{equation} \label{sheareq1}
\gamma_T(r) = \overline{\kappa}(\leqslant r) - \overline{\kappa}(r),
\end{equation}
where $\gamma_T(r)$ is the tangential shear at radius $r$ relative 
to the position of the lens \citep{Escude91}.  This tangential shear is 
related to the 
distortion of an image by the lens.  $\kappa = \Sigma/\Sigma_{crit}$ is the 
scaled surface density, where the scale factor $\Sigma_{crit}$ is the critical
density for multiple lensing.  The critical density 
depends only on the geometry of the source-lens-observer system:
\begin{equation} \label{sigcriteq}
\Sigma_{crit} = { c^2 \over {4\pi G} } {D_S \over {D_L D_{LS}} } \approx
0.35\ \left( {D \over {1 \textrm{Gpc}}}\right)^{-1}\textrm{g\ cm$^{-2}$}
 = 1660 \ \left( {D \over {1 Gpc}}\right)^{-1} \textrm{M$_{\sun}$ \ pc$^{-2}$}
\end{equation}
where $D_S$ and $D_L$ are the angular diameter distances to the source and 
lens respectively, and $D_{LS}$ is the angular diameter distance between 
source and lens.  D is known as the effective distance 
and $D = D_L D_{LS}/D_S$.  Multiple images typically occur when 
$\kappa=\Sigma/\Sigma_{crit}$ approaches unity.  The production of multiple
images takes either very special alignment of the source with the center of 
the mass distribution (where $\kappa$ is large) or extremely high densities.
We will only discuss the weak regime here, when $\kappa << 1$ and the 
shear $\gamma_T(r) \approx e_T/2.$ where $e_T$ is the induce ellipticity.

One can multiply both sides of Eq. \ref{sheareq1} by $\Sigma_{crit}$ and get
the following relationship:
\begin{equation} \label{densecont}
\overline{\Sigma}(\leqslant r) - \overline{\Sigma}(r) = \Sigma_{crit} \gamma_T(r)
\end{equation}
Thus, if one knows the redshift of the lens and the redshift distribution of
the background galaxies, a density contrast can be inferred from the shear.
One can then choose an appropriate model for $\Sigma$ and infer the mass.

\section{Models}
\subsection{Singular Isothermal Sphere} \label{SIS}
This is a bit hand wavy, but here goes.  We model the stars in a galaxy
or the galaxies in a cluster of galaxies as ``particles'' in an ideal gas.
This gas has the equation of state
\begin{equation}\label{siseqofstate}
P = {{\rho kT} \over m }
\end{equation}
If the gas is in equilibrium, then the temperature T is related to the velocity
dispersion by
\begin{equation}
m\sigma_v^2 = kT
\end{equation}
where the velocity dispersion could depend on radius, but we will assume that
the gas is isothermal implying that T and $\sigma_v$ are constant.
For hydrostatic equilibrium we have
\begin{eqnarray}
{ {\partial P} \over {\partial r} } & = & -{ {G M(r)\rho } \over r^2 } \\
{ {\partial M} \over {\partial r} } & = & 4 \pi r^2 \rho
\end{eqnarray}
A simple solution to these equations is 
\begin{equation}
\rho(r) = { \sigma_v^2 \over {2 \pi G} } {1 \over r^2}
\end{equation}
which is known as the \it{singular isothermal sphere} \normalfont(SIS).  In
lensing all we can see is the projected mass density:
\begin{equation}
\Sigma(R)_{SIS} = { \sigma_v^2 \over 2 G } {1 \over R} = 1.16\times10^{14}\left( \sigma_v
\over 1000 \textrm{km/s} \right)^2 \left(R \over 1\textrm{Mpc}\right)^{-1}
\textrm{M$_{\sun}$ Mpc$^{-2}$} 
\end{equation}
where R will now represent the two dimensional radius. Note this also implies a
temperature:
\begin{equation}
T = 1.14\times10^8 \left( \sigma_v \over 1000 \textrm{km/s} \right)^2 K
= 9.82 \left( \sigma_v \over 1000 \textrm{km/s} \right)^2 \textrm{kev}
\end{equation} 

What we measure with Eq. \ref{densecont} is a density contrast; the average
density within a radius $r$ minus the azimuthally averaged density at that
radius. For this model we get:

\begin{eqnarray}
\overline{\Sigma}(\leqslant R) - \overline{\Sigma}(R) & = & 
  { { \intop_0^R \Sigma(r) 2 \pi r dr } \over 
  { \intop_0^R 2 \pi r dr} } - \Sigma(R) \\ \nonumber
 & = & { \sigma_v^2 \over {2 G} } \left( {1 \over \pi R^2 } \intop_0^R 
          2 \pi dr - {1 \over R} \right)  \\ \nonumber
 & = & {\sigma_v^2 \over {2 G}} {1 \over R} \\ \nonumber
    \\
 & = & \Sigma(R)_{SIS} \nonumber
\end{eqnarray}

Thus, for a true isothermal we can directly measure the density profile with weak
lensing.  This is usually done by fitting for the velocity dispersion
$\sigma_v$.  The mass can then be inferred from integrating the surface 
density:
\begin{equation}
M(\leqslant R)_{SIS} = 7.31 \times 10^{14}\left( \sigma_v \over 1000 \textrm{km/s} \right)^2 
\left(R \over 1\textrm{Mpc} \right) \textrm{M$_{\sun}$}
\end{equation}
 
\subsection{Truncated Isothermal Sphere} \label{SIStrunc}
Perhaps we can measure the shear to large enough radii that we can consider
seeing a possible ``edge'' to our mass distribution. We can write a truncated form of the SIS:
\begin{equation}
\Sigma(R) = { \sigma_v^2 \over {2 G} } 
\left( {1 \over R} - {1 \over \sqrt{ R^2 + s^2 }} \right)
\end{equation}
This density profile is isothermal when $R << s$ but goes more rapidly to zero in
the regime $R >> s$.  Thus we can think of s as the truncation radius of the
density profile.

Following the same analysis as we used in \S \ref{SIS} we get
\begin{equation}
\overline{\Sigma}(\leqslant R) - \overline{\Sigma}(R) = \Sigma(R)_{SIS}
 \left( 1 + 2 X - { {1 + 2 X^2}  \over \sqrt{1+ X^2}  } \right)
\end{equation}
where $X = s/R$ and
\begin{equation}
M(\leqslant R) = M(\leqslant R)_{SIS} \left(1 + X - \sqrt{1 + X^2}\right)
\end{equation}
One then fits for the velocity dispersion and cutoff radius.  Note that, unlike
the SIS, this model has a finite mass at $R = \infty$.
\begin{equation}
M = 7.31\times10^{14}\left( \sigma_v \over 1000 \textrm{km/s} \right)^2 
\left(s \over 1\textrm{Mpc} \right) \textrm{M$_{\sun}$}
\end{equation}

\subsection{Power Laws}

What if your data doesn't fit well to an isothermal?  You could try an power
law, for example I have tried fitting this function to our RASS clusters:
\begin{equation}
\Sigma(R) = \Sigma_0 \left( R \over {1 \textrm{Mpc} } \right)^{-\alpha}
\textrm{M$_{\sun}$ Mpc$^{-2}$}
\end{equation}
One can then fit Eq. \ref{densecont} to a power law:
\begin{equation}
\overline{\Sigma}(\leqslant R) - \overline{\Sigma}(R) = 
A \left( R \over {1 \textrm{Mpc} } \right)^{-\alpha} \textrm{M$_{\sun}$ Mpc$^{-2}$}
\end{equation}
And it can be shown that
\begin{eqnarray}
\Sigma_0 & = & \left( {2-\alpha} \over \alpha \right) A \\
M(\leqslant R) & = & {2 \over {2-\alpha}} \Sigma(R) 
\pi \left(R \over \textrm{1 Mpc} \right)^2 \textrm{M$_{\sun}$}
\end{eqnarray}

These models have the undesirable property of negative density for power law
indices greater than 2. For an isothermal, $\alpha$ = 1 and we
have $M \propto R$ and $\Sigma_0 = {\sigma_v^2 \over {2 G}}$.

\begin{thebibliography}{}

\bibitem[Kaiser \& Squires (1993)]{KS93} Kaiser, N.. Squires, G., 1993, \apj,
404, 441
\bibitem[Miralda-Escud\'{e} (1991)]{Escude91} Miralda-Escud\'{e}, J. 1991, 
ApJ, 370, 1.

\end{thebibliography}{}
\end{document}
