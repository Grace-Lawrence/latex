\documentclass{beamer}

\usepackage{beamerthemesplit}
\usepackage{verbatim}

\usepackage{xcolor}
\definecolor{mygray}{RGB}{200,200,200}
\usetheme{default}
%\usetheme{Pittsburgh}
%\usecolortheme{seagull}
%\usecolortheme{seahorse}
%\usecolortheme{beaver}
\usecolortheme{mule}

\usefonttheme{serif}

%\DeclareGraphicsExtensions{.pdf,.png,.jpg}

\newcommand{\snT}{$(S/N)_{\textrm{size}}$}
%\newcommand{\snT}{$\left( \frac{S}{N}\right)_{\textrm{size}}$}
\newcommand{\snflux}{$(S/N)_{\textrm{flux}}$}
%\newcommand{\snflux}{$\left( \frac{S}{N}\right)_{\textrm{flux}}$}

\newcommand{\lensfit}{\texttt{LENSFIT}}
\newcommand{\numba}{\texttt{Numba}}
\newcommand{\python}{\texttt{Python}}
\newcommand{\ngmix}{\texttt{ngmix}}
\newcommand{\shear}{{\bf g}}
\newcommand{\redmapper}{redMaPPer}

\newcommand{\prelim}{{\bf{\it Preliminary}}}

\definecolor{gold}{rgb}{1.,0.84,0.}

\definecolor{brightred}{rgb}{1.,0.4,0.4}


\title{Metacalibration}
\author{Erin Sheldon}
\institute{Brookhaven National Laboratory}

% http://texblog.net/latex-archive/plaintex/beamer-footline-frame-number/
% to add the page (frame ) number and not screw up the bottom line
% works for split themes?
\expandafter\def\expandafter\insertshorttitle\expandafter{%
      \insertshorttitle\hfill%
        \insertframenumber\,/\,\inserttotalframenumber}

% suppress navigation bar
\beamertemplatenavigationsymbolsempty
\setbeamertemplate{footline}{}

\begin{document}

\frame{\titlepage}


\setbeamertemplate{background canvas}[vertical shading][bottom=mgray,top=mblack]

\frame
{
    \frametitle{Outline}

    \setbeamerfont*{itemize/enumerate body}{size=\Large}
    \setbeamerfont*{itemize/enumerate subbody}{parent=itemize/enumerate body}
    \setbeamerfont*{itemize/enumerate subsubbody}{parent=itemize/enumerate body}
 
    \begin{itemize}

        %\item The Primary Goal is to Study Dark Energy
        \item Metacalibration

        \item My take on it

        \item Simulations using models

        \item Simulations using real galaxy images

    \end{itemize}

}

\frame
{
    \frametitle{Shear Accuracy Requirements}

    \setbeamerfont*{itemize/enumerate body}{size=\Large}
    \setbeamerfont*{itemize/enumerate subbody}{parent=itemize/enumerate body}
    \setbeamerfont*{itemize/enumerate subsubbody}{parent=itemize/enumerate body}
 
    \begin{itemize}

        \item In order to measure the Dark Energy equation of state
            to our desired accuracy for DES/LSST, we must measure
            shear with exquisite accuracy.

        \item Shear calibration errors
            \begin{itemize}
            
                \item {\color{cyan} DES}:~ $\Delta \gamma/\gamma \lesssim 0.004$
                \item {\color{green} LSST}: $\Delta \gamma/\gamma \lesssim 0.001$

            \end{itemize}

        \item There are other sources of error besides the shear
            measurement error, so we would like to be well under
            these numbers

    \end{itemize}

}



\frame
{
    \frametitle{Metacalibration Idea from Eric Huff}

    \setbeamerfont*{itemize/enumerate body}{size=\normalsize}
    \setbeamerfont*{itemize/enumerate subbody}{parent=itemize/enumerate body}
    \setbeamerfont*{itemize/enumerate subsubbody}{parent=itemize/enumerate body}
 
    \begin{itemize}

        \item Say we have a biased shear estimator {\color{gold} $E$}.  Then we can write
            {\color{gold}
                \begin{eqnarray}
                    E & = & E(\gamma=0) + \frac{\partial E}{\partial \gamma} \gamma + ... \nonumber \\
                      & \sim &  \frac{\partial E}{\partial \gamma} \gamma \equiv R \gamma \nonumber 
                \end{eqnarray}
            } 
        \item Use image manipulation to estimate the derivative of the
            estimator with respect to shear
            {\color{gold}
                \begin{equation}
                    R = \frac{E(+\Delta\gamma) - E(-\Delta\gamma)}{2 \Delta \gamma} \nonumber 
                \end{equation}
            }
            \begin{itemize}
                \item Deconvolve the PSF
                \item Shear the image by a small amount
                \item Reconvolve by the PSF.  Use a slightly larger PSF to suppress
                    the noise amplification
            \end{itemize}


    \end{itemize}

}

\frame
{
    \frametitle{Metacalibration Idea from Eric Huff}

    \setbeamerfont*{itemize/enumerate body}{size=\Large}
    \setbeamerfont*{itemize/enumerate subbody}{parent=itemize/enumerate body}
    \setbeamerfont*{itemize/enumerate subsubbody}{parent=itemize/enumerate body}
 
    \begin{itemize}
        
        \item Should correct for modeling biases

        \item Should correct for noise-related biases

        \item Can also derive terms for PSF leakage {\color{gold} $R_{\mathrm{PSF}}$ }

        \item Works well at high shear.

    \end{itemize}

}



\frame
{
    \frametitle{My Take on Metacalibration}

    \setbeamerfont*{itemize/enumerate body}{size=\large}
    \setbeamerfont*{itemize/enumerate subbody}{parent=itemize/enumerate body}
    \setbeamerfont*{itemize/enumerate subsubbody}{parent=itemize/enumerate body}
 
    \begin{itemize}

        \item These {\color{gold} $R$} are very noisy, cannot be used naively.
            \begin{itemize}

                \item Slightly different noise correlations in the + and -
                    sheared images.
                \item Amplified by {\color{gold} $1/\Delta \gamma$}
            \end{itemize}

        \item Eric Huff has a method to bring in prior information (paper
            to appear shortly)

        \item My take is to use deep image data.
            \begin{itemize}
                \item Perform Deconv/Reconv on deep data
                \item Then degrade images to noise level of shallow data.
                \item Use exactly same noise image for 
                    {\color{gold} $E(+\Delta\gamma), E(-\Delta\gamma)$}.
            \end{itemize}

        \item If needed, weight deep data to match properties of shallow data.

    \end{itemize}

}

\frame
{
    \frametitle{Simulations Using Models}

    \setbeamerfont*{itemize/enumerate body}{size=\large}
    \setbeamerfont*{itemize/enumerate subbody}{parent=itemize/enumerate body}
    \setbeamerfont*{itemize/enumerate subsubbody}{parent=itemize/enumerate body}
 
    \begin{itemize}

        \item Challenging simulation similar to recent paper Bernstein et al. 2015

            \begin{itemize}
                \item {\color{cyan} PSF} Slightly undersampled Moffat: $r_{50}=1.5$ pixels

                \item {\color{cyan} Galaxies}: Bulge+Disk each with $r_{50} \in [1.5, 3.0]$ pixels.  Bulge
                    offset randomly up to distance of r50

                \item {\color{cyan} S/N}: Realistic flux distribution.  $S/N \gtrsim 8$
            \end{itemize}

        \item The shear is recovered to the accuracy required for future
            surveys

            {\color{gold}
                \begin{equation}
                    \Delta \gamma/\gamma = (2.16 \pm 3.19) \times 10^{-4} \nonumber
                \end{equation}
            }

    \end{itemize}

}

\frame
{
    \frametitle{Simulations Using Real Galaxy Images and Realistic PSF}

    \setbeamerfont*{itemize/enumerate body}{size=\normalsize}
    \setbeamerfont*{itemize/enumerate subbody}{parent=itemize/enumerate body}
    \setbeamerfont*{itemize/enumerate subsubbody}{parent=itemize/enumerate body}
 
    \begin{itemize}

        \item Matt Becker (Stanford) created a modified version of 
            the GREAT3 image simulation code. Matt's version is
            a re-usable library

        \begin{itemize}
            \item {\color{cyan} PSF}: Realistic, modeled after DES.  Optical aberrations derived
                by Aaron Roodman and Chris Davis (SLAC)

            \item {\color{cyan} Galaxies}: Images drawn from COSMOS HST images.
        \end{itemize}

        \item First short run

            {\color{gold}
                \begin{equation}
                    \Delta \gamma/\gamma = (2.9 \pm 2.2) \times 10^{-3} \nonumber
                \end{equation}
            }
        \item Matt is creating a large simulation now

        \item My Gaussian mixture modeling of the PSF may be a limiting
            factor

    \end{itemize}

}

\frame
{
    \frametitle{Summary}

    \setbeamerfont*{itemize/enumerate body}{size=\large}
    \setbeamerfont*{itemize/enumerate subbody}{parent=itemize/enumerate body}
    \setbeamerfont*{itemize/enumerate subsubbody}{parent=itemize/enumerate body}
 
    \begin{itemize}
        \item Metacalibration is a new idea for shear recovery from
            Eric Huff

        \item My version is an alternative to Eric Huff's, using
            deep imaging
        
        \item In initial simulations the shear is recovered within errors

        \item {\color{brightred}
                \textbf{
                    Please do not distribute these results yet, I'm waiting for Eric's paper to come out
                }
        }

    \end{itemize}

}

\end{document}
