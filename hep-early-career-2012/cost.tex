\begin{deluxetable}{lcc}
    \tabletypesize{\small}
    \tablecaption{Projected Computing Purchases\label{table:computing}}
    \tablewidth{0pt}
    \tablehead{
        \multicolumn{1}{l}{Fiscal Year} &
        \colhead{Compute Servers with 3 GPUs Each}   & 
        \colhead{Total Storage} \\
        &
        &
        [TB]
    }
    \startdata
2013 & 3 & 30 \\
2014 & 3 & 30 \\
2015 & 3 & 30 \\
2016 & 2 & 30 \\
2017 & 2 & 30 \\
\hline
\relax\\[-1.7ex]
Total in 5 years & 13 & 130 TB \\\\[-2.7ex]
\enddata

    \tablecomments{The number of compute nodes purchased is based on the
        assumption that each node and GPU (Intel 12 cores, 2.7GHz, 32GB ram,
        two nvidia 2050 GPUs) would stay at the performance level of a node
        purchased in 2012.  It is not clear how the performance per GPU will
        increase over time so this table is conservative.  Disk is likely to
        get cheaper, so the disk per machine could be expanded at fixed cost.
        The purchase pattern above was chosen to force the overall budget to be
        flat each year, but in fact carryover is allowed and we will most
        likely make purchase evenly over the five year period.  Power,
        cooling and maintenance will be provided at no extra cost to this
        experiment, but overhead is included in the budget.}

    \end{deluxetable}
    
