\begin{deluxetable}{lcccc}
    \tabletypesize{\small}
    \tablecaption{Projected Computing Purchases\label{table:computing}}
    \tablewidth{0pt}
    \tablehead{
        \multicolumn{1}{l}{Fiscal Year} &
        \colhead{Disk Storage}       & 
        \colhead{\$ for Storage}    & 
        \colhead{Compute Servers}   & 
        \colhead{\$ for CPU} \\
        &
        [TB] &
        &
        2010 Equivalent &
    }
    \startdata
2012 & 14 & 4000 & 13 & 27000 \\
2013 & 18 & 4000 & 17 & 27000 \\
2014 & 23 & 4000 & 21 & 27000 \\
2015 & 30 & 4000 & 28 & 27000 \\
2016 & 46 & 4000 & 41 & 27000 \\
\hline
\relax\\[-1.7ex]
Total in 4 years & 131 & 20000 & 120 & 135000 \\\\[-2.7ex]
\enddata

    \tablecomments{The number of compute nodes purchased is based on
    the assumption that each node (26kSI2k, 104 HEP-SPEC 2006) would stay at the
    performance level of a node purchased in 2010. As the performance per node will
    increase over time the actual number of compute nodes after 3 years will be
    significantly smaller (probably O(70)), providing a combined performance of
    O(120) 2010 equivalent nodes. Prices include 40\% bulk discounts from
    purchasing through the RHIC ATLAS Computing Facility at BNL. {\bf Power,
    cooling and maintence will be provided at no extra cost to this experiment.}}

    \end{deluxetable}
    
