\documentclass[12pt]{article}

\begin{document}

This currently addressed through section three. I've glanced at the
rest and things look pretty good.  I have extensive comments and
questions for the ridgeline likelihood section.

Overall notes:

There sometimes seems to be mismatching between the referenced object
and the actual number of the object. For example, just after equation
15, you refer to figure 5, but it is actually figure 6 you are referring
to.  Are you not using labels?


\section{Introduction}

\subsection*{p.1}

Some of these statements are not yet proven, such as inflation. 

Cosmology isn't an entity, so it cannot contribute :) I would say something
about how the evolution of structure is predicted by the current models, which
of course are parametrized by some numbers (which is probably what you meant by
cosmology).

I would generally rework this first paragraph.  This is the toughest paragraph
to write and is also very important.

\subsection*{p.2}

Individual galaxies are also observable counterparts of dark matter halos.

\subsection*{p.3}

``However they suffer....''  I think you mean the method suffers. 
``....which can mimic in projection...''  Could be reworded.
The last sentence needs to be reworded and expanded, since it isn't
clear from context how you would be estimating the mass.

\subsection*{p.4}

``X--ray methods are...''  Maybe you should be more clear about what is not as
sensitive.  I assume you mean ``the identification of potential wells through
their X--ray emmision...'' or something.  Note, there are are point sources, so
``..the hot glow...'' can come from other things.  You need an extended source
in the right energy range, etc.  Its enough to say the catalogs are flux
limited without saying they are flux limited at low redshift. Generally people
use capital X and two dashes in X--ray.

\subsection*{p.5}

``...across wide redshifts..'' no plurality on redshift ranges.  Your
comment about flux preservation is a little confusing.  It implies that
somehow preservation of flux is a problem for the other methods, when really
the point is that the energy boost of the CMB photons just depends on the
physics of the gas not the redshift.  Note, there is still that angular
size -redshift relation to contend with, so small clusters are still limited.
It allows a construction of volume, pressure, and size limited samples,
the last two being correlated.

\subsection*{p.6}

Weak lensing doesn't really give you a mass limited sample, since it is highly
dependend on redshift.  Its a shear-size limited sample.  The light isn't
distorted; you might say the shapes are distorted.  

\subsection*{p.7}

Here is a good place to say what you will do in the article in addition
to whose work you will build upon.

\section{Algorithm}

\subsection{Outline}

\subsubsection*{p.1}

No need to say it is an alternative, its just a method.  You might call
it a new method, since this is the first paper on the algorithm! You say
it labels peaks in the galaxy density field, but then talk about color too.

You should say 1/r in projection.  CMD, is this the first time it's used?
Maybe you should also first explain what the E/S0 ridgeline is. 

Not all the galaxies on a given E/S0 ridgeline are very luminous, since it has
been seen to extend to a fraction of L*.

\subsubsection*{p.2}

Might want to talk about the evidence for this, and the fact that it isn't
always the case that the brightest galaxy lives at the center.

\subsubsection*{p.3}

``Individual objects...''  This is a very difficult sentence.  Needs to
be made clear.

\subsubsection*{p.5}

You wrote S0 (S-zero) when abbreviating Spherical Overdensity.

\subsection{Likelihood Framework}

\subsubsection*{p.1}

The sentence ``..center of cluster...'' should be ``..center of a cluster...''.

This first sentence is a little misleading.  You are really using the galaxies
as discrete tracers of the ``density field'' in ra/dec/z, then looking at the
maximum likelihood of the objects being a cluster center in three dimensions.
Varying the redshift tells you nothing about its proximity to the center
in ra/dec.

\subsection{Ridgeline Likelihood}

\subsubsection*{p.2 right after ``Spatial Filter''}

Do you mean truncated at x=0.1 or given a core? The galaxies are still counted
right?  I think you should show the plot as a log-log.

This is a long paragraph.
\subsubsection*{p.6}

LRG hasn't been defined I don't think.  

``...LRGs in this range must be...'' I think you mean LRG's with $z < 0.15$
must be selected in alternative ways.  

Define eclass and fracDev.

Another long paragraph.

\subsubsection*{p.7 right after ``Luminosity Filter''}

First sentence needs reworking.  I find this paragrph confusingly worded.  You
want a cutoff in absolute magnitude that results in a volume limited catalog to
moderate redshift with high S/N detections of the members at that redshift
limit.

\subsubsection*{p.12}

I don't understand equation 8.  Let me try to reverse engineer it.

I think you are comparing a model plus background $M + b$ to the measured
counts $D$. A simple model for the measured counts, ignoring filters and
redshift,
\begin{equation}
N(<R) = N_{bg} + P_{bg} + \int_{0}^{R} d^{2}R \times NFW(R)   + P_{NFW}
\end{equation}
where $N(<R)$ is the total number in the aperture of radius $R$, 
$N_{bg}$ is the mean number expected from the field, $P_{bg}$ is the
a poisson random number for the background, the integral is the
integral over an NFW profile assuming the galaxies follow an NFW, 
and $P_{NFW}$ is the poisson noise on this term.  The noise is not background 
dominated beyond some intermediate redshift, but I'll need to run
some numbers to know: later.

We normalize the NFW w.r.t to the maximum radius which will correspond to our
aperture:
\begin{eqnarray}
NFW(R) & = & N_{gal} F(R) \nonumber \\
\int_{0}^{Rmax} d^{2} R F(R) & = & 1 \\
\end{eqnarray}
where $F(R)$ is just the normalized, projected NFW profile. This give's a 
definition to $N_{gal}$, which can therefore be fit for.  

I presume your model is essentially this, but with the filters added.  The new
normalization condition involves integrals over color as well as space.  The
filters must then be included in the model, but the fit parameters are still
just $N_{gal}$ and z. The model including the color filters will suppress the
background by integrating the field color distribution with the filters.
\begin{equation}
N^{eff}_{bg} = N_{bg} \int_{0}^{Rmax}d^{2} R F(R) \int_{-\infty}^{\infty} d(g-r) n_{bg}(g-r) G_{gr} \int_{-\infty}^{\infty} d(r-i) n_{bg}(r-i) G_{ri}
\end{equation}
Which can be measured, along with the noise term, from data.  

The NFW model term then becomes an integral over colors
and an integral of the spatial distribution $n_{cl}(r)$ times the NFW. 
\begin{equation}
NFW(<R) = \int_{0}^{Rmax} d^{2} R~n_{cl}(R) F(R) \int_{-\infty}^{\infty} d(g-r) G_{gr} n_{cl}(g-r) \int_{-\infty}^{\infty} d(r-i)  G_{ri} n_{cl}(r-i) 
\end{equation}
In the case where we are sitting at the center of a cluster at the right
redshift this becomes
\begin{eqnarray}
NFW(<R) & = & N_{gal} \int_{0}^{Rmax} d^{2} R ~F^{2}(R) \int_{-\infty}^{\infty} d(g-r) G^{2}(g-r) \int_{-\infty}^{\infty} d(r-i) G^{2}(r-i) \nonumber \\
  & = & \frac{N_{gal}}{2} \int_{0}^{Rmax} d^{2} R ~F^{2}(R) \equiv N_{gal} \frac{F_2}{2}
\end{eqnarray}
you will need to keep the factor of 1/2 in mind.  I don't know the answer
to the integral $F_2$  but it must also be accounted for in order to
recover the correct $N_{gal}$

One can then test the $\chi^{2}$ for a grid of $N_{gal}$ and $z$ to find the
maximum likelihood.  I'm not sure where equation 12 comes from, however.

Also, you are assuming background noise dominated, but this won't be true
at all redshifts. What do we know about the validity of this?

\end{document}
