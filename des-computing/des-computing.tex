\documentclass[12pt]{article}

\usepackage{graphicx}

\usepackage[margin=1in,dvips]{geometry}

\usepackage{color}
\usepackage{deluxetable}


% for comments
\usepackage{verbatim}


% bibliography stuff
\usepackage{natbib}
% Stripped some defs out of aastex
% plotting
\def\eps@scaling{1.0}% 
\newcommand\epsscale[1]{\gdef\eps@scaling{#1}}% 

\newcommand\plotone[1]{% 
 \centering 
 \leavevmode 
 \includegraphics[width={\eps@scaling\columnwidth}]{#1}% 
}% 
\newcommand\plottwo[2]{% 
 \centering 
 \leavevmode 
 \columnwidth=.45\columnwidth 
 \includegraphics[width={\eps@scaling\columnwidth}]{#1}% 
 \hfil 
 \includegraphics[width={\eps@scaling\columnwidth}]{#2}% 
}% 
\newcommand\plotfiddle[7]{% 
 \centering 
 \leavevmode 
 \vbox\@to#2{\rule{\z@}{#2}}% 
 \includegraphics[% 
  scale=#4, 
  angle=#3, 
  origin=c 
 ]{#1}% 
}% 


%% journal definitions
\newcommand\aj{\rmfamily{AJ}}% 
          % Astronomical Journal 
\newcommand\araa{\rmfamily{ARA\&A}}% 
          % Annual Review of Astron and Astrophys 
\newcommand\apj{\rmfamily{ApJ}}% 
          % Astrophysical Journal 
\newcommand\apjl{\rmfamily{ApJ}}% 
          % Astrophysical Journal, Letters 
\newcommand\apjs{\rmfamily{ApJS}}% 
          % Astrophysical Journal, Supplement 
\newcommand\ao{\rmfamily{Appl.~Opt.}}% 
          % Applied Optics 
\newcommand\apss{\rmfamily{Ap\&SS}}% 
          % Astrophysics and Space Science 
\newcommand\aap{\rmfamily{A\&A}}% 
          % Astronomy and Astrophysics 
\newcommand\aapr{\rmfamily{A\&A~Rev.}}% 
          % Astronomy and Astrophysics Reviews 
\newcommand\aaps{\rmfamily{A\&AS}}% 
          % Astronomy and Astrophysics, Supplement 
\newcommand\azh{\rmfamily{AZh}}% 
          % Astronomicheskii Zhurnal 
\newcommand\baas{\rmfamily{BAAS}}% 
          % Bulletin of the AAS 
\newcommand\jrasc{\rmfamily{JRASC}}% 
          % Journal of the RAS of Canada 
\newcommand\memras{\rmfamily{MmRAS}}% 
          % Memoirs of the RAS 
\newcommand\mnras{\rmfamily{MNRAS}}% 
          % Monthly Notices of the RAS 
\newcommand\pra{\rmfamily{Phys.~Rev.~A}}% 
          % Physical Review A: General Physics 
\newcommand\prb{\rmfamily{Phys.~Rev.~B}}% 
          % Physical Review B: Solid State 
\newcommand\prc{\rmfamily{Phys.~Rev.~C}}% 
          % Physical Review C 
\newcommand\prd{\rmfamily{Phys.~Rev.~D}}% 
          % Physical Review D 
\newcommand\pre{\rmfamily{Phys.~Rev.~E}}% 
          % Physical Review E 
\newcommand\prl{\rmfamily{Phys.~Rev.~Lett.}}% 
          % Physical Review Letters 
\newcommand\pasp{\rmfamily{PASP}}% 
          % Publications of the ASP 
\newcommand\pasj{\rmfamily{PASJ}}% 
          % Publications of the ASJ 
\newcommand\qjras{\rmfamily{QJRAS}}% 
          % Quarterly Journal of the RAS 
\newcommand\skytel{\rmfamily{S\&T}}% 
          % Sky and Telescope 
\newcommand\solphys{\rmfamily{Sol.~Phys.}}% 
          % Solar Physics 
\newcommand\sovast{\rmfamily{Soviet~Ast.}}% 
          % Soviet Astronomy 
\newcommand\ssr{\rmfamily{Space~Sci.~Rev.}}% 
          % Space Science Reviews 
\newcommand\zap{\rmfamily{ZAp}}% 
          % Zeitschrift fuer Astrophysik 
\newcommand\nat{\rmfamily{Nature}}% 
          % Nature 
\newcommand\iaucirc{\rmfamily{IAU~Circ.}}% 
          % IAU Cirulars 
\newcommand\aplett{\rmfamily{Astrophys.~Lett.}}% 
          % Astrophysics Letters 
\newcommand\apspr{\rmfamily{Astrophys.~Space~Phys.~Res.}}% 
          % Astrophysics Space Physics Research 
\newcommand\bain{\rmfamily{Bull.~Astron.~Inst.~Netherlands}}% 
          % Bulletin Astronomical Institute of the Netherlands 
\newcommand\fcp{\rmfamily{Fund.~Cosmic~Phys.}}% 
          % Fundamental Cosmic Physics 
\newcommand\gca{\rmfamily{Geochim.~Cosmochim.~Acta}}% 
          % Geochimica Cosmochimica Acta 
\newcommand\grl{\rmfamily{Geophys.~Res.~Lett.}}% 
          % Geophysics Research Letters 
\newcommand\jcp{\rmfamily{J.~Chem.~Phys.}}% 
          % Journal of Chemical Physics 
\newcommand\jgr{\rmfamily{J.~Geophys.~Res.}}% 
          % Journal of Geophysics Research 
\newcommand\jqsrt{\rmfamily{J.~Quant.~Spec.~Radiat.~Transf.}}% 
          % Journal of Quantitiative Spectroscopy and Radiative Trasfer 
\newcommand\memsai{\rmfamily{Mem.~Soc.~Astron.~Italiana}}% 
          % Mem. Societa Astronomica Italiana 
\newcommand\nphysa{\rmfamily{Nucl.~Phys.~A}}% 
          % Nuclear Physics A 
\newcommand\physrep{\rmfamily{Phys.~Rep.}}% 
          % Physics Reports 
\newcommand\physscr{\rmfamily{Phys.~Scr}}% 
          % Physica Scripta 
\newcommand\planss{\rmfamily{Planet.~Space~Sci.}}% 
          % Planetary Space Science 
\newcommand\procspie{\rmfamily{Proc.~SPIE}}% 
          % Proceedings of the SPIE 


\let\astap=\aap 
\let\apjlett=\apjl 
\let\apjsupp=\apjs 
\let\applopt=\ao 
\newcommand\phn{\phantom{0}}% 
\newcommand\phd{\phantom{.}}% 
\newcommand\phs{\phantom{$-$}}% 
\newcommand\phm[1]{\phantom{#1}}% 
\let\la=\lesssim            % For Springer A&A compliance... 
\let\ga=\gtrsim 
\newcommand\sq{\mbox{\rlap{$\sqcap$}$\sqcup$}}% 
\newcommand\arcdeg{\mbox{$^\circ$}}% 
\newcommand\arcmin{\mbox{$^\prime$}}% 
\newcommand\arcsec{\mbox{$^{\prime\prime}$}}% 
\newcommand\fd{\mbox{$.\!\!^{\mathrm d}$}}% 
\newcommand\fh{\mbox{$.\!\!^{\mathrm h}$}}% 
\newcommand\fm{\mbox{$.\!\!^{\mathrm m}$}}% 
\newcommand\fs{\mbox{$.\!\!^{\mathrm s}$}}% 
\newcommand\fdg{\mbox{$.\!\!^\circ$}}% 
%\newcommand\farcm@mss{\mbox{$.\mkern-4mu^\prime$}}% 
%\let\farcm\farcm@mss 
%\newcommand\farcs@mss{\mbox{$.\!\!^{\prime\prime}$}}% 
%\let\farcs\farcs@mss 
\newcommand\fp{\mbox{$.\!\!^{\scriptscriptstyle\mathrm p}$}}% 
\newcommand\micron{\mbox{$\mu$m}}% 
\def\farcm@apj{% 
 \mbox{.\kern -0.7ex\raisebox{.9ex}{\scriptsize$\prime$}}% 
}% 
\def\farcs@apj{% 
 \mbox{% 
  \kern  0.13ex.% 
  \kern -0.95ex\raisebox{.9ex}{\scriptsize$\prime\prime$}% 
  \kern -0.1ex% 
 }% 
}% 
\newcommand\case[2]{\mbox{$\frac{#1}{#2}$}}% 
\newcommand\slantfrac{\case}% 
\newcommand\onehalf{\slantfrac{1}{2}}% 
\newcommand\onethird{\slantfrac{1}{3}}% 
\newcommand\twothirds{\slantfrac{2}{3}}% 
\newcommand\onequarter{\slantfrac{1}{4}}% 
\newcommand\threequarters{\slantfrac{3}{4}}% 
\newcommand\ubvr{\mbox{$U\!BV\!R$}}%% UBVR system 
\newcommand\ub{\mbox{$U\!-\!B$}}%   % U-B 
\newcommand\bv{\mbox{$B\!-\!V$}}%   % B-V 
\newcommand\vr{\mbox{$V\!-\!R$}}%   % V-R 
\newcommand\ur{\mbox{$U\!-\!R$}}%   % U-R 
\newcommand\ion[2]{#1$\;${\small\rmfamily\@Roman{#2}}\relax}% 
\newcommand\nodata{ ~$\cdots$~ }% 
\newcommand\diameter{\ooalign{\hfil/\hfil\crcr\mathhexbox20D}}% 
\newcommand\degr{\arcdeg}% 
\newcommand\Sun{\sun}% 
\newcommand\Sol{\sun}% 
\newcommand\sun{\odot}% 
\newcommand\Mercury{\astro{\char1}}% Mercury symbol, "1" 
\newcommand\Venus{\astro{\char2}}% Venus symbol, "2" 
\newcommand\Earth{\earth}% 
\newcommand\Terra{\earth}% 
\newcommand\earth{\oplus}% 
\newcommand\Mars{\astro{\char4}}% Mars symbol, "4" 
\newcommand\Jupiter{\astro{\char5}}% Jupiter symbol, "5" 
\newcommand\Saturn{\astro{\char6}}% Saturn symbol, "6" 
\newcommand\Uranus{\astro{\char7}}% Uranus symbol, "7" 
\newcommand\Neptune{\astro{\char8}}% Neptune symbol, "8" 
\newcommand\Pluto{\astro{\char9}}% Pluo symbol, "9" 
\newcommand\Moon{\astro{\char10}}% Moon symbol, "M" 
\newcommand\Luna{\Moon}% 
\newcommand\Aries{\astro{\char11}}% 
\newcommand\VEq{\Aries}% vernal equinox (Aries) 
\newcommand\Taurus{\astro{\char12}}% 
\newcommand\Gemini{\astro{\char13}}% 
\newcommand\Cancer{\astro{\char14}}% 
\newcommand\Leo{\astro{\char15}}% 
\newcommand\Virgo{\astro{\char16}}% 
\newcommand\Libra{\astro{\char17}}% 
\newcommand\AEq{\Libra}% autumnal equinox (Libra) 
\newcommand\Scorpius{\astro{\char18}}% 
\newcommand\Sagittarius{\astro{\char19}}% 
\newcommand\Capricornus{\astro{\char20}}% 
\newcommand\Aquarius{\astro{\char21}}% 
\newcommand\Pisces{\astro{\char22}}% 


\begin{document}


\title{Computing Resources for DES Weak Lensing}
%\author{Erin Sheldon\\{\normalsize Brookhaven National Laboratory}}
\author{DES Weak Lensing Working Group}
%\author{Erin Sheldon, Zhaoming Ma, Thomas Throwe\\
%{\normalsize Brookhaven National Laboratory}\\
%Michael Jarvis\\
%{\normalsize University of Pennsylvania}}

\date{}
\maketitle

%{\tiny
%\tableofcontents
%}

\section{Executive Summary}

We propose to support the DES weak lensing science effort by building a
computing base at Brookhaven National Laboratory (BNL).  

Lensing measurements are particularly computationally intensive, and make use
of essentially all of the basic data products, from pixels to catalogs.  The
lensing signal and systematic effects are both subtle enough that probing them
will require processing a significant fraction of the full DES data set.
However, effective development requires a reasonably tight feedback loop
between development and data processing.  The need to process large amounts
of data quickly enough to provide meaningful feedback to developers 
can only be met if significant computing is available.

On top of this, the lensing working group is developing multiple pipelines in
parallel, which requires correspondingly more computing power but notably not
more infrastructure.  It makes sense to share computer resources and local
expertise to aid all pipeline development.

Computer hardware purchases of \$30,000/year for five years will meet these
needs.  The computing will be hosted at the RHIC ATLAS Computing Facility at
BNL (RACF).  The RACF will provide power, cooling, installation, and
maintenance at zero cost, and we will receive $\sim$40\% bulk discounts by
purchasing alongside larger experiments.

A number of DES weak lensing group participants are already using a small
compute cluster at BNL for science code development.  Erin Sheldon of BNL and
Mike Jarvis of UPenn have been developing the primary DES WL pipeline at BNL
and all major tests runs of the code have been performed there.  Zhaoming Ma is
a postdoc at BNL and has developed and tested a new, highly computationally
intensive method for PSF interpolation using Principle Component Analysis
(PCA).  He will run this code as needed throughout the survey.  Mandeep Gill of
Ohio State has begun working on an alternative WL code at BNL.  Tim Eifler has
begun running is two-point lensing code on the cluster.  Also, Joerg Dietrich
of Michigan now has an account and will begin testing his WL code on the BNL
cluster in winter 2010.  Any other interested parties are encouraged to join
this effort at no cost to them.  Tom Throwe, Brookhaven physicist and computer
systems expert, will assist DES users in solving computing issues that arise.
Funding for computing resources at BNL will ensure this development can proceed
efficiently and at minimal cost.


\section{Outline of Goals}

The goal of the DES weak lensing working group (WLWG) is to support DES weak
lensing science.  This science relies on many data products from the survey,
including the images, calibrated fluxes, astrometry, as well as derived
products such as PSF characterization, galaxy shear estimates, and galaxy
photometric redshift distributions.

The WLWG will primarily support the science through development of pipelines to
derive accurate shear estimates.  These measurements require touching all the
pixels, and so are computationally intensive.  The group is also developing
multiple pipelines in parallel in order to converge on an optimal method and
for consistency checks.  

\begin{comment}
In addition to pipeline development and analysis, a small amount of computing
will be needed for lensing ray tracing simulations.  These will be used to aid
theoretical predictions and test the covariance matrices used in the lensing
analysis.  Calculations predicting the exact hardware needs were not available
at the time of this writing, but the expectation is about 5\% the level
required for the main pipeline processing.  These simulations will be run on
dedicated high memory machines at the University of Pennsylvania.
\end{comment}

\section{Current Algorithms and the Development Cycle}

The pipeline currently developed by Mike Jarvis and Erin Sheldon is
incorporated into the DESDM.  The pipeline is run on data produced by the DESDM
at earlier stages in the processing.  These are images that have instrumental
signatures removed and basic catalogs generated from those images.  The DESDM
will run the weak lensing pipeline on a time scale corresponding to the yearly
data releases. 

For efficient development of the algorithms, many more processings will be
required.  Weak lensing methods are still evolving, and there is much
work to be done in order to produce shear estimates sufficiently free of
systematics to reach our science goals.

Supporting this research will require significant computing resources. For DES
science we must measure one percent shear signals to an accuracy of $\approx 1$
part in 100.  But the noise per galaxy shear estimate due to the intrinsic
shape of the galaxy is of order 30\%, so one must reduce a large number of
galaxies in order to even test the pipeline to desired accuracy.  In order to
characterize the signals and systematics to the required level under a wide
variety of observational circumstances, a significant fraction of the total
data set must be processed, which in turn requires computing.

In addition, a tight feedback loop is required between data processing and
development.  The only way to process such huge amounts of data quickly enough
to provide sufficient feedback is to assemble a large amount of computing.

Once the reduction pipeline has run to create the galaxy shape catalog from the
pixel data, we must also analyze the catalog data to extract parameters of the
dark matter and dark energy.   Some of the most powerful of these analyses are
computationally intensive.  In the next section we will give some examples of
computation times in existing datasets.



\section{Example Timings for Weak Lensing Codes}

\subsection{Basic Pipeline Timings}

Table \ref{table:timing} gives some example timings on DC4 data using the
pipeline developed by Jarvis and Sheldon.  For these timings we used the small
astro cluster at BNL.  The astro cluster, at that time, had three compute
nodes, each with 8 Intel Xeon 3GHz cores and 32GB RAM.  These are timings for
the weak lensing pipeline alone: the images and catalogs used as input are
generated beforehand by the DESDM.


\begin{deluxetable}{lcccc}
\tabletypesize{\small}
\tablecaption{DC4 Timing Numbers on the Bach Cluster at BNL 
	\label{table:timing}}
\tablewidth{0pt}
\tablehead{
	\colhead{Data Set} &
	\colhead{\# images/tiles} & %\colhead{Size} &
	\colhead{Memory Per Job} &
	\colhead{CPU hours}
}
\startdata
%DC4 SE & 46,500/NA & 700G & 1G & 1728 \\
%DC4 ME & 22,402/224 & 355G & 15-48G & 576 \\
DC4 SE & 46,500 & 1G & 1728 \\
DC4 ME & 22,402/224 & 15-48G & 576 \\
\enddata

\tablecomments{Resources used for processing $i$-band DC4 images using the
astro cluster at BNL in 2009.  The Bach cluster at that time consisted of three
compute nodes with 8 cores and 32G RAM each.  DC4 SE is all the SE images (4k
by 2k chips) available at the time DC4 was released.  DC4 ME is the multi-epoch
data: Catalogs derived from the coadd tiles and all the corresponding SE images
that contributed to each tile. Only the unique images are reported in the
count.  Note the timings for coadd ME analysis would be significantly longer if
the astro machines did not have high memory. }

\end{deluxetable}

In DES each bit of sky will be imaged during multiple epochs.  Thus there are
generally two types of algorithms for measuring shear: those processing a
single epoch (SE) and those simultaneously processing multiple epochs (ME).  In
table \ref{table:timing}, the SE data set is every image we had available
at the time DC4 was released. This may include more images than the ``official''
DC4 release.  When multiple processings of the same image were available, we
used the newest. The ME analysis made use of a subset of these images, about
half.

One ``image'' here means an individual 4k by 2k CCD exposure.  ``Tiles'' are
combined ``coadd'' images of all exposures covering a predetermined area of the
sky.  There are a number of steps in the SE processing. We find bright stars,
characterize the PSF, interpolate the PSF to the location of all objects, and
finally determine a best shear for each object based on associated pixels and
PSF.  For our current code, this takes about 160 CPU-seconds per image.  The
processing of each image is entirely independent and is currently using 8 cores
in parallel for each image.  Further parallelization is trivial.

For ME processing, we select objects from the coadd catalogs.  We then
transform the coordinates back into the individual SE images that contributed
to the coadd.  For each of these SE images we reconstruct the PSF as determined
during the SE processing described above. We then perform a joint fit for the
shear across all images.  This takes about 2.5 CPU-hours per coadd tile. The
code uses all 8 cores in parallel. This works out to be about 90 CPU-seconds
per SE image contributing to the tile.  Processing each tile is independent but
depends upon the previous SE processing to get PSF information.

Note the memory usage for the ME processing is very high, ranging from 15G to
48G depending on the number of SE images that contribute to each coadd tile.
The average is about 20G.  The machines in the Bach cluster have 32Gb memory,
so all but a few tiles fit into memory.  The processing is significantly slower
when the machine has less than the required memory; e.g. 30-40\% slower if the
computers had 4G instead of 32G. 

\subsection{Analysis Timings}

Many lensing analyses are relatively quick, but some require significant
computing time.  For example, the cluster mass and luminosity analysis
presented in \cite{SheldonM2L07} took two weeks running on a 300 processor
cluster.  The computers used were about a factor of two slower than the CPUs in
the Bach cluster.  DES data set will be 50 times larger.  

The three-point shear function and the PCA PSF decomposition are also
computationally intensive, each taking of order a week to run on the fiducial
compute system we propose below.  But these analysis will need to run dozens of
times: For the PCA we must explore the quality of the interpolation for
different algorithmic models for the spatial variations and the number of
important principle components.  For the three-point function, we expect to
recalculate this for different combinations of shears from various redshift
bins.



\section{Extrapolation and Requested Resources}

\subsection{Timing Extrapolation}


\begin{deluxetable}{lcccc}
    \tabletypesize{\small}
    \tablecaption{Projected Computing Purchases\label{table:computing}}
    \tablewidth{0pt}
    \tablehead{
        \multicolumn{1}{l}{Fiscal Year} &
        \colhead{Disk Storage}       & 
        \colhead{\$ for Storage}    & 
        \colhead{Compute Servers}   & 
        \colhead{\$ for CPU} \\
        &
        [TB] &
        &
        2010 Equivalent &
    }
    \startdata
2012 & 14 & 4000 & 13 & 27000 \\
2013 & 18 & 4000 & 17 & 27000 \\
2014 & 23 & 4000 & 21 & 27000 \\
2015 & 30 & 4000 & 28 & 27000 \\
2016 & 46 & 4000 & 41 & 27000 \\
\hline
\relax\\[-1.7ex]
Total in 5 years & 131 & 20000 & 120 & 135000 \\\\[-2.7ex]
\enddata

    \tablecomments{The number of compute nodes purchased is based on
    the assumption that each node (26kSI2k, 104 HEP-SPEC 2006) would stay at the
    performance level of a node purchased in 2010. As the performance per node will
    increase over time the actual number of compute nodes after 5 years will be
    significantly smaller (probably O(70)), providing a combined performance of
    O(120) 2010 equivalent nodes. Prices include 40\% bulk discounts from
    purchasing through the RHIC ATLAS Computing Facility at BNL. {\bf Power,
    cooling and maintence will be provided at no extra cost to this experiment.}}

\end{deluxetable}
 

The full DES survey is expected to generate 80,000 exposures over five years.
Each exposure generates 62 CCD images, for a total of $5\times 10^6$ images.
The data volume on disk will be $\approx 100$~TB of disk space when stored in
compressed format, and including the data quality indicators for each pixel,
plus the coadded images and catalogs that are produced by the DESDM pipeline
and serve as inputs to the ME process.

Scaling from the hardware and software speeds used for table
\ref{table:timing}, the SE processing is simply (1728 cph)/(46000 im)*(80000
exp*62 im/exp) gives about 21 cpu-years.  The multi-epoch processing is more
difficult to scale, but assuming the relative number of images per tile is
similar in the future, we predict (576 cph)/(22402 im)*(80000 exp)*(60 im/exp)
gives 14 cpu-years.  On an 80 node cluster of equivalent 2010 machines this
could be processed in 2.5 weeks.  This is approximately the desired processing
time given that we will want to reprocess the data with several iterations on
two or more independent shape algorithms in order to obtain a robust result.

We can expect to speed up the current algorithms but also should be prepared
for future algorithms that require more computation.  We will also gain as
processing power follows Moore's law. This gain has traditionally been in
transistor density on a single device, but recently has been maintained by
increasing the number of cores.  The Jarvis and Sheldon code can make use of
multiple cores and thus fully utilize high memory and multiple cores optimally.
Future algorithms will be coded for multi-core performance also.

Note also, we plan to support shear algorithms from groups other than Jarvis \&
Sheldon.  It is difficult to predict the speed of these algorithms, but for
example first tests of the imcat based pipeline provided by Mandeep Gill is a
factor of five slower than the Jarvis \& Sheldon pipeline.  Certainly this can
be improved, but we may expect the multiplier for adding additional algorithms
to be greater than unity.

We must also store the 100 TB of data in order to efficiently process it
through multiple algorithms.  


\subsection{Purchasing Plan}

Taking the fiducial cluster of 80 nodes and the desired storage, we have
developed a purchasing plan that should be nearly optimal in the sense that
we can process data as it arrives but take advantage of increasing computing
power and storage per dollar.  This plan is outlined in table \ref{table:computing}.

To store and process the 100 TB of DES data, we propose to spend about
\$30,000/year for five years. The first year we will acquire $\sim$ 13 nodes
with at least 32G of memory each (26kSI2k, 104 HEP-SPEC 2006).  In following
years we will purchase more computing for the same price, and keep a trajectory
to our goal of about 70 new nodes.  Note, the table shows {\bf 2010 equivalent
nodes}; adjusting for a compounded Moore's law, we get 122 of today's nodes
corresponding to 70 actual nodes.  These 70 nodes will augment the  nodes we
currently have, but note the existing nodes are in heavy use for other
purposes.

Note in table \ref{table:computing} we have listed storage purchases per year.
We are also requesting and additional 30TB of disk to provide storage for the
lensing data products, including many re-processings.  This makes a total of
$\approx 130$TB of storage.

It is important to note these prices include bulk discounts of order 40\%.
This is due to purchasing along with other BNL experiments through the RHIC
ATLAS Computing Facility (RACF).  Because these resources are on a relatively
small scale for the RACF, power, cooling, and maintenance are provided at no
additional cost.  


\section{Brookhaven as a Host for the Computing Resources}

Brookhaven is well suited to hosting this computing initiative. 

Erin Sheldon of BNL is an expert at performing weak lensing analysis in
enormous datasets. He has performed many lensing analyses using data from the
SDSS, which is the largest lensing data set to date.   He has been a member of
DES since 2003 and has since helped to develop the current de facto
pipeline used for lensing. He has also developed a general framework for
processing single epoch and multi epoch DES data through any code.  This
framework will support various DES lensing algorithms.

BNL will support any DES weak lensing efforts as needed.  Current development
of the Jarvis/Sheldon pipeline is primarily occurring at BNL, and all major
recent tests and runs of the code have occurred there including DC6b.  Catalogs
from individual runs of the code are available on the BNL web site.

Mandeep Gill of Ohio State and Tim Eifler have already begun working at BNL and
Mandeep will be the first to incorporate an alternate lensing codes into the
framework.  These catalogs will be hosted at BNL with a possible future
release through the Brazil portal. 

%We are open to collaboration with anyone in DES who needs significant computing
%to process the pixel data, especially those in the WLWG.

The Brookhaven RHIC ATLAS Computing Facility is massive and world class.  In
comparison to our plan for $\sim$70 new machines, the other experiments sharing
the RACF support about 7500 equivalent cpus, many petabytes of storage, and
three supercomputers.  Our system will use power in the kilowatt range, whereas
currently RACF uses 2.5 megawatts continuously.  In preparation for the data
coming from ATLAS, the computing center will more than double in size and power
usage during the period we will purchase our computers.  Because our needs are
insignificant in comparison, they will provide us with complimentary power,
cooling, and maintenance, as well as bulk purchasing discounts of order 40\%.
The RACF is an excellent base upon which to build our computing initiative.


\bibliographystyle{unsrt}
\bibliography{astroref}


\end{document}
