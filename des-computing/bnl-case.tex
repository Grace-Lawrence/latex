\documentclass[12pt]{article}
\usepackage{amsmath}
\title{DES Computing}
\author{Erin S. Sheldon\\
Brookhaven National Laboratory}
\date{}

\begin{document}
\maketitle


In this document we present the case for separate computing for DES lensing
pipeline work at Brookhaven National Laboratory (BNL).  Here we present a high
level view; technical details can be found in the full DES analysis computing
document (ref to not compromised version of doc).

The lensing pipeline will be run in a ``production mode'': as data is gathered
it will be immediately processed through one or more lensing pipelines.  We
expect to approach 100\% utilization of the computing resources.  For
efficiency reasons, we do not wish to fight for computing resources in a shared
queue system with highly inhomogeneous demands from a large number of users.
This is made more acute by the fact that much of the lensing processing
requires very high memory for each job: this usage pattern requires that entire
nodes be used for a given job, not just a single core on a machine.

The technology required to run the pipelines in production mode has been
developed ``in house'' at BNL:  
\begin{itemize}

    \item The processing framework was written by Sheldon. This system would
    have to be transferred, requiring significant time from Sheldon.

    \item In order to get the required data throughput, we use a distributed
    disk model built on the Hadoop Distributed File System (HDFS).  HDFS
    provides the throughput for image processing which we were not able to get
    using NFS servers.  We at BNL set this up and understand it. We would have
    to transfer the knowledge to another site.  This would require significant
    time from Sheldon and experts at BNL.
    
    \item A simplified job queuing system was written by Sheldon which
    facilitates the lensing work.

\end{itemize}

In production mode, we must have a tight relationship with those who maintain
the machines and software stack.  Although HDFS provides high availability, we
still need tight integration with the maintainers to deal with hardware
failures and software upgrades in order to work at maximum efficiency.  

Because maintenance, power, and cooling for DES computing is provided free by
BNL, there is no direct cost to the DES analysis computing effort for these
services.

In summary, all the needed technology and expertise exists at BNL to facilitate
the lensing pipeline work.  We expect to work at close to 100\% efficiency.
Transferring this knowledge to another institution would require months of
effort from Sheldon and other experts at BNL.  Furthermore, the lack of tight
integration with the system maintainers will introduce latency and require more
effort from Sheldon and others.  These inefficiencies will introduce unneeded
cost to the project and reduce time available to Sheldon and others for
scientific investigation.

\end{document}
