
%\documentclass{article}
\documentclass[12pt]{article}


\usepackage{vmargin}
% can use \begin{singlespace} \begin{doublespace} \begin{onehalfspace} (1.5 spacing)

\setpapersize{USletter}
%\setmargrb{leftmargin}{topmargin}{rightmargin}{bottommargin}
\setmargrb{1.4in}{.5in}{1.4in}{1in}

\newcommand{\tsp}{\vspace{0.1cm}}
\newcommand{\isp}{\vspace{0.3cm}}
\newcommand{\ssp}{\vspace{0.4cm}}
\newcommand{\As}{A$_s$}
\newcommand{\Ap}{A$_p$}
\newcommand{\Ma}{M$_a$}
\newcommand{\Mb}{M$_b$}
\newcommand{\micros}{$\mu$s}
\newcommand{\es}{{\it e}}
\newcommand{\gs}{{\it g}}
\newcommand{\kettt}[3]{$|${#1},{#2},{#3}$\rangle$}
\newcommand{\kett}[2]{$|${#1},{#2}$\rangle$}
\newcommand{\ket}[1]{$|${#1}$\rangle$}

\begin{document}


\footnotesize
\noindent
{Rauschenbeutel, et al., ``Controlled Entanglement of Two Field } \hfill {Erin Sheldon}

{Modes in a Cavity Quantum Electrodynamics Experiment'',} \hfill {Final Project}

{Phys. Rev. A {\bf64}, 050301(R) (2001)} \hfill {Physics 644}

\normalsize

\section*{Summary}
The authors present the first measurement of entanglement of two field modes
in a cavity. They accomplish this by interacting a Rydberg atom with the two
modes in succession, producing a maximally entangled state. The state is copied
onto a second probe atom after an adjustable delay, the state of which is then
measured to reveal the entanglement.

As the authors point out, the experimental setup is very similar to that of 
earlier papers, \cite{rausch00,brune96,nogues99} to name a few. 
In these other works the demonstrated 
entanglement was between a cavity mode and the atoms, and between atoms. In this
work the entanglement is between two different field modes. The implications of this
new type of entanglement will be discussed later. This paper is very brief so 
some of the information I present here was gathered from other sources

Themal Rubidium atoms are relased from an oven and are then velociy selected
using laser pumping (v=503 $\pm$ 2 m/s). These atoms are then prepared in a 
Rydberg state in a time resolved way. 
The atoms are prepared in pairs, with each 
successive pair separated in time
by about 1.7 ms. 
The individual atoms in each pair 
are separated by an adjustable time T in the range 50-700 \micros.
They claim a spatial resolution of 1mm, corresponding to
a temporal resolution of about 0.2 \micros. 
The first atom, \As, is prepared in the Rydberg state \es\ (``excited'') and the 
second, \Ap,  in the lower state \gs\ (``ground''), 
corresponding to n=51 and 50 respectively. The atoms cross a cavity made
from two niobium superconducting mirrors. This cavity supports two orthogonally
polarized transverse
modes \Ma\ and \Mb\, with frequency 
separation \Ma$-$\Mb$\equiv \delta$. For this experiment
$\delta/2\pi$ = 128.3 $\pm$ 0.1 kHz as measured previously.
Initially each mode is in the
zero photon state. The energy difference between $e-g$ can be 
tuned to resonance with \Ma\ or \Mb\ via the Stark effect: i.e. the 
detuning $\Delta = (E_e-E_g)/\hbar - $
\Ma\ is set to 0 or $-\delta$. 
The atom-field interaction can also be frozen out by shifting
 the atomic line well below \Mb.

The first atom, \As, is initially in state \es\ and when it enters the cavity
the detuning $\Delta$ is zero, in resonance with mode \Ma. 
%In Brune et al.(1996) the point was to demonstrate the Rabi flopping. 
The atom
is allowed to interact for a $\pi/2$ pulse.  Note the tuning can be done in a
\micros\ as compared to the $\pi/2$ pulse which lasts $\pi/2\Omega$, or 
about 6 \micros\ for $\Omega/2\pi$ = 47 kHz. Thus the detuning takes a
non-negligible 
fraction of the $\pi/2$ pulse time; I assume this isn't a problem, but couldn't find
any documentation on the matter.
After the $\pi/2$ pulse the state has evolved
from \kettt{$e_s$}{$0_a$}{$0_b$} to an entangled state
%\ket{$\Psi_1$}  = 
%( \kett{$e_s$}{$0_a$} + \kett{$g_s$}{$1_a$} ) \ket{$0_b$}$/\sqrt{2}$.
\begin{equation}
\textrm{\ket{$\Psi_1$}}  = 
\frac{( \textrm{\kett{$e_s$}{$0_a$} + \kett{$g_s$}{$1_a$}} ) \textrm{\ket{$0_b$}}}{\sqrt{2}}.
\end{equation}
Where here the zero of energy is taken to be the \kettt{$g_s$}{$1_a$}{$0_b$} state.
Then the detuning is set to resonance with \Mb, $\Delta = -\delta$, and the
atom is allowed to interact for a $\pi$ pulse. The \kett{$g_s$}{$0_b$} 
remains unchanged, but
the \kett{$e_s$}{$0_b$} $\rightarrow$ $e^{i \phi}$ \kett{$g_s$}{$1_b$}.
The phase is $\phi = \pi/2 + \delta\pi/\Omega$. The $\pi/2$ phase comes from
the orthogonal polarization of 
\Ma\ relative to \Mb. The second comes from the 
energy difference $-\hbar \delta$, which accumulates phase over the
flopping time $\pi/\Omega$.
The atom ket factors out and we see that the two field modes are 
entangled. After time t, 
\begin{equation}
\textrm{\ket{$\Psi_2(t)$}} = 
\frac{ e^{i \phi + i \delta t} \textrm{\kett{$0_a$}{$1_b$} + 
\kett{$1_a$}{$0_b$}} }{\sqrt{2}}.
\end{equation}
The interaction is then frozen out and \As\ leaves the cavity.

After an adjustable time t=T the probe atom \Ap\ is sent through the cavity. The 
idea is to copy the cavity mode entanglement onto the probe atom. It is
initially in state \gs\ and is in resonance with \Ma.  \Ap\ undergoes a $\pi$
pulse, induced by the photon in \Ma\ after which \ket{$0_a$} factors out and
we have a superposition  i exp($i\delta\pi/2\Omega$)exp($i\delta T$)\kett{$g_p$}{$1_b$} $-$ \kett{$e_p$}{$0_b$}. Then, 
after setting $\Delta = -\delta$, \Ap\ 
undergoes a $\pi/2$ pulse in resonance with \Mb. 
\begin{equation}
\textrm{ \ket{$\Psi_4$} } = \frac{1}{2}[i\textrm{\kett{$g_p$}{$1_b$}}
             (1 - e^{i \delta T}e^{i \delta \pi/2 \Omega})
                        + \textrm{\kett{$e_p$}{$0_b$}} 
             (1 + e^{i \delta T}e^{i \delta \pi/2 \Omega}) ]
\end{equation}
The atom then has a probability
of being in \es\ or \gs\ that depends on the time interval T: 
\begin{equation}
P_e(T) = \frac{1}{2}[ 1 + \textrm{cos} (\delta T + \Phi) ]
\end{equation}
Thus P$_e$ oscillates with frequency $\delta$, the mode frequency difference.
This is a direct reflection of the beating between the modes.

The authors show a plot of this measurement in four different
time windows. These plots indicate an oscillation at the correct frequency 
$\delta$ (from the plot I calculated about 132 kHz,  the best fit value
is not given in the paper). 
The amplitude of the oscillation damps over time as the cavity comes
into thermal equilibrium and the atom absorbs these fields
(this thermal noise was ``erased'' by pulses of atoms in the \gs\ state 
before \As\ entered the cavity \cite{nogues99}).

The new feature of this measurement is the entanglement of two cavity
modes. Previous work using the same apparatus demonstrated other types
of entanglement. The first step
here, the Rabi flopping of a Rydberg atom in resonance with
a field mode, was the
basis of many of the papers cited in this work:
e.g. a quantum phase gate was demonstrated using that technique
\cite{rauch99}.
The difference here is the extra detuning step to bring the atom in resonance
with another cavity mode. The resulting Rabi oscillation for a $\pi$ pulse
entangles the two cavity modes. 

The authors mention a couple of uses for this entangled state. The state
of \Mb\ could be copied onto a second atom, which is then detuned
and interacts non resonantly with \Ma, after which the new 
state can be copied back onto \Mb. This can be used to realize the ``conditional
dynamics of a quantum phase gate.''
I think this could be done since the basic pieces of that measurement have
already been demonstrated.
They also mention that teleportation
could be achieved by entangling the modes from two different cavities. 
I'm not in a position to judge whether or not that is feasible.

I recommend this paper for publication
after the authors address the comments which follow.

\section*{Comments}
Because the experimental techniques and the theoretical concepts here 
are not novel, but the result is of immediate interest, this paper is well
suited for rapid communications. The authors still have plenty of
room for more text, however, and I think that there are issues that 
should be addressed in the remaining space.

The authors
should use a few sentences to address scalability in this system.
Coupling many atoms is difficult with
this setup because of the relatively fast photon decay and
thermalization timescales ($\sim$ 1ms), and higher
atom fluxes are not the answer because it becomes more difficult
to control the exact number of atoms in the cavity. 
Also, the atoms are high speed (500 m/s in this work, as low as 300 m/s in others) 
and cover a few millimeters during the interaction, 
so the suggestion
the authors made of coupling multiple cavities can only be scaled so far because of
space restrictions, at least with the current setup. 

When the techniques are listed the authors usually include a reference. 
This is good, but this reference is more often than not just a pointer to
one of their own papers, which itself does not explain the technique but also
references more of their papers. In this way the reader is led through a
series of references to the authors' 
PRL, Science, and Nature papers, none of which explains the technique
in any detail. Please include a few definitive references for the 
newcomer to this field.


\bibliographystyle{plain}

\begin{thebibliography}{}

\bibitem{rausch00} A. Rauschenbeutel, et al., Science {\bf 288}, 2024 (2000)
\bibitem{brune96} M. Brune et al., Phys. Rev. Lett. {\bf 76}, 1800 (1996)
\bibitem{nogues99} G. Nogues et al., Nature (London) {\bf 400}, 239 (1999)
\bibitem{rauch99} A. Rauschenbeutel, et al., Phys. Rev. Lett. {\bf 83}, 5166 (1999)

\end{thebibliography}


\end{document}
