\documentclass{aastex}
\usepackage{emulateapj5}

\def\sax{{\it BeppoSAX\,}}
\def\asca{{\it ASCA\,}}
\def\rosat{{\it ROSAT\,}}
\def\rxte{{\it RXTE\,}}
\def\uly{{\it Ulysses\,}}

\newcommand{\ave}[1]{\langle{#1}\rangle}
\newcommand{\PSbox}[3]{\mbox{\special{psfile=#1}\hspace{#2}\rule{0pt}{#3}}}
\newcommand{\pfg}{\vspace{0.3cm}}

\shorttitle{ASM GRB Light Curves}
\shortauthors{Smith et al.}

\slugcomment{Draft: \today}

\begin{document}
 
\title{X-Ray Light Curves of Gamma-ray Bursts Detected with the All-Sky
Monitor on \rxte}

\author{D. A. Smith\altaffilmark{1}, A. Levine\altaffilmark{2}, H. Bradt\altaffilmark{2}, K. Hurley\altaffilmark{3}, M. Feroci\altaffilmark{4}, P. Butterworth\altaffilmark{5}, T. Cline\altaffilmark{5}, G. Pendleton\altaffilmark{6}, \& S. Phengchamnan\altaffilmark{6,7}}
\altaffiltext{1}{The University of Michigan}
\altaffiltext{2}{Massachusetts Institute of Technology}
\altaffiltext{3}{The University of California, Berkeley}
\altaffiltext{4}{Istituto de Astrofisica Spaziale}
\altaffiltext{5}{NASA/Goddard Space Flight Center}
\altaffiltext{6}{NASA/Marshall Space Flight Center}
\altaffiltext{7}{University of Alabama at Huntsville}


\email{dasmith@space.mit.edu}

\begin{abstract}

We present X-ray light curves (2--12~keV) for sixteen gamma-ray bursts
(GRBs) detected by the All-Sky Monitor on the {\it Rossi X-ray Timing
Explorer}.  We discuss these light curves within the context of a
simple relativistic fireball and synchrotron shock paradigm, and we
address the possibility of observing the transition between a GRB and
its afterglow.  We compare these soft X-ray light curves with count
rate histories of the GRBs obtained by the high-energy experiments
($>~12$~keV) BATSE, the \sax~Gamma-Ray Burst Monitor, the burst
monitor on \uly, and Konus/WIND.  The sixteen light curves show a
diverse range of morphologies, with striking differences between
energy bands.  In several bursts, intervals of significant emission
are evident in the ASM energy range with little or no corresponding
emission apparent in the high-energy light curves.  For example, the
final peak of GRB~970815 is only detected in the softest BATSE energy
bands and may represent the beginning of the afterglow.  We also study
the duration of bursts as a function of energy.  Simple, singly-peaked
bursts seem consistent with the $E^{-0.5}$ power law expected from an
origin in synchrotron radiation, but durations of bursts that exhibit
complex temporal structure are not consistent with this prediction.
Bursts such as GRB~970828 that show many short spikes of emission at
high energies last significantly longer at low energies than the
synchrotron cooling law would predict, and this may be the result of
an emission region being shocked by internal shocks before the
electrons radiating in X-rays can cool, rather than the onset of an
X-ray afterglow.

\end{abstract}

\keywords{gamma rays: bursts}

\section{INTRODUCTION\label{sec:intro}}

\subsection{X-Rays from Gamma-Ray Bursts}

The first detection of a gamma-ray burst (GRB) at X-ray energies
($\sim$1--15~keV) was made in 1972 with two proportional counters on
the {\it OSO-7} satellite~\citep{wubde73}.  For the next 15 years,
there were few X-ray observations of GRBs, but they revealed many
interesting GRB properties.  Instruments on {\it Apollo-16} were used
to detect a GRB, also in 1972, and light curves in several energy
bands showed that the spectrum evolved 
over the burst's single peak~\citep{mpgpt74,tesam74}.  Four bursts
were detected with the Air Force satellite {\it P78-1} in 1979, and
the peak X-ray emission was found to lag the peak gamma-ray
emission~\citep{lefks84}.  

Efforts to identify X-ray counterparts to GRBs intensified in the late
1980s.  The {\it Ginga} satellite was equipped with co-aligned
wide-field detectors to cover the energy range from
2--400~keV~\citep{mfhin89}.  These instruments were used to detect
$\sim$120 bursts between 1987 and 1991~\citep{omnyf91}.  Analysis of
twenty-two of these observations confirmed that spectral softening is
common in the tails of bursts and showed that the X-ray band can
contain a large fraction of the energy emitted from a
GRB~\citep{sfmy98}.  X-ray precursors were observed in a few
cases~\citep{minpf91}.  Between 1989 and 1994, 95 bursts were detected
by the GRANAT/WATCH all-sky monitor in two energy bands, 8--20~keV and
20--60~keV, and thirteen of them were found to exhibit significant
emission in the lower energy band before and/or after the activity in
the higher energy band~\citep{sstlb98}.

The Wide Field Camera on \sax~was used to detect 34 GRBs in the
1.5--26.1~keV band between July 1996 and February 2000\footnote{See
also J. Greiner's archive at {\tt
http://www.aip.de:8080/$\sim$jcg/grbgen.html}}.  In addition to
providing broadband light curves for these GRBs~\citep{facmp00}, the
\sax~effort has proven fruitful for the study of GRB ``afterglows'',
the fading emission sometimes seen after a GRB, associated with
cooling radiation in the aftermath of the GRB explosion.  Only one of
those 34 bursts (GRB~980519) exhibited significant soft X-ray activity
before the onset of the GRB in gamma-rays~\citep{zhpf99}.

The general picture that has emerged is that X-ray light curves for
GRBs tend to track their gamma-ray counterparts.  Spectral evolution
may or may not be present.  Sometimes both forms of evolution are
present.  Most often the times of X-ray peak emission tend to lag
behind the peaks at higher energies.  Bursts tend to last longer in
X-rays, but this is not true of every burst.  A small fraction of
bursts exhibit X-ray activity with no detectable emission at gamma-ray
energies.  This activity sometimes precedes the GRB and sometimes
follows it.  In very rare cases, it does both.

\subsection{A Simplified Conventional Model\label{sec:grbmod}}

\citet{piran99} has reviewed the standard ``fireball'' model for GRBs
in great detail.  Here we present of brief summary of this model, to
which we shall refer when interpreting features of the GRB light
curves observed with the ASM.  In this model a large amount of energy
($\sim10^{51}$~ergs) is dumped into a small volume to create a very
hot, optically thick fireball~\citep{goodm86}.  The fireball expands
rapidly, and since it contains only a very small amount of baryonic
matter~\citep{shpi90,kps99}, the expansion becomes highly
relativistic~\citep{blmck76}.  By the end of the acceleration phase,
all the available energy has been transferred to the bulk kinetic
flow~\citep{cavree78}.  

The optical depth of an expanding fireball, $\tau$, can be related to
the bulk Lorentz factor of the expansion, $\Gamma$, and the minimum
observed timescale for variability during the burst, $\delta T$
(e.g.~\citet{piran99}).  This relation hinges on the observation that
high-energy GRB spectra are non-thermal, with a photon index
$\alpha$. In order for the observed spectrum to be non-thermal, the
optical depth must be less than unity.  The derived relation is
\begin{equation}
\label{eq:tauyy}
\tau \sim \frac{2\times10^{15}}{\Gamma^{4+2\alpha}}\ E_{52}\ 
\left( \frac{\delta T}{10\ {\rm ms}} \right)^{-2} \lesssim 1,
\end{equation}
where $E_{52}$ is the total energy in the fireball in units of
$10^{52}$~erg. This constraint demands a lower limit on the bulk
Lorentz factor, dependant on $\alpha$.  Early work found that the high
energy spectral index can vary from $\sim1.6$ to higher than 5, with
no particular preferred value~\citep{bmfsp93}.  A recent analysis of
bright bursts from the Fourth BATSE Catalog reports an asymmetric
distribution of high-energy spectral indices that peaks around 2.25
and extends beyond 4~\citep{pbmpp00}.  If $\alpha$ is 2.25, then
$\Gamma$ must be greater than $\sim60$, if the other terms in
Equation~\ref{eq:tauyy} are of order unity.

Once such a highly relativistic speed is reached, the ejecta coast
quietly until a shock forms.  If the ejecta sweep up enough mass from
the surrounding medium to decelerate and convert the bulk kinetic
energy to random motion and radiation~\citep{mr92,rm92}, the shock is
referred to as ``external''.  If the central engine of the GRB source
is erratic and emits multiple shells of ejecta at different speeds,
``internal'' shocks may form as these shells overtake each other, and
radiation will be emitted~\citep{kps97}.  The fading afterglow
observed after many GRBs is believed to originate in the cooling of
the ejecta behind the external shock front~\citep{mr97}, while the
complex temporal structure of the GRBs themselves has been attributed
to multiple internal shocks~\citep{npp92}.  In both internal and
external shocks, the dominant form of radiation is expected to be
synchrotron emission~\citep{snp96,piran99}.

In a model of post-shock synchrotron radiation, the frequency of peak
emission ($\nu_m$) increases as the bulk Lorentz factor of the shock
to the fourth power~\citep{snp96,spn98,wg99}.  The inverse of this
relationship is given by
\begin{equation}
\label{eq:peaklimgam}
\Gamma_{\rm sh} = 14\ \epsilon_B^{-1/8} \epsilon_e^{-1/2} n_1^{-1/8} 
\left( \frac{p-1}{p-2} \right)^{1/4}
\left( \frac{h \nu_m}{1\ {\rm keV}} \right)^{1/4}
\end{equation}
(See, e.g.,~\citet{piran99}, Equation~105), where $n_1$ is the density
of the external medium, $\epsilon_B$ and $\epsilon_e$ define the
fractional energy transferred to the post-shock magnetic field and
electron distribution, respectively, and $p$ is the index of the
number spectrum of the Lorentz factor distribution of the shocked
electrons ($N \propto \Gamma_{\rm sh}^{-p}$). 

In this paper we present X-ray light curves for the GRBs detected with
the All-Sky Monitor (ASM) on the {\it Rossi X-ray Timing Explorer}
({\it \rxte}) and interpret them in the context of this synchrotron
shock model.  We include light curves for each of the thirteen GRBs
described in~\citet{sblr99}, as well as GRB~990308 and GRB~000301C.
We also present a light curve for GRB~961216, which was detected but
could not be localized accurately.  All these events have been
confirmed as GRBs via their detection at higher energies by other
instruments (Table~\ref{flutab})\footnote{The IPN master list of burst
detections is maintained by K. Hurley at {\tt
http://ssl.berkeley.edu/ipn3/}}.  Section~\ref{sec:lcdat} describes
the data and our analysis techniques.  Section~\ref{sec:lcs} presents
the observations and the GRB light curves.  We compare the ASM light
curves with the equivalent high-energy light curves as recorded by
devices on other satellites.  Section~\ref{sec:lcsum} summarizes both
the common features and the striking differences among these light
curves and discusses them in the context of the model.

\section{INSTRUMENTATION AND ANALYSIS}\label{sec:lcdat}

The ASM consists of three Scanning Shadow Cameras (SSCs) mounted on a
motorized rotation drive~\citep{lbcjm96}.  The assembly holding the
three SSCs is generally held stationary for a 90-s ``dwell''.  The
drive then rotates the SSCs through $6\arcdeg$ between dwells, except
when it is necessary to rewind the assembly.  Each SSC contains a
proportional counter with eight resistive anodes.  Each event detected
on exactly one resistive anode is characterized by SSC and anode
numbers, total pulse height, and a one-dimensional position in the
coordinate parallel to the anode, determined via the charge-division
technique.  Each SSC views a $12\arcdeg \times 110\arcdeg$ (FWZI)
field through a mask perforated with pseudo-randomly spaced slits.
The long axes of the slits run perpendicular to the anodes.

During each dwell, the positions of incident photons are accumulated
in histograms for each anode for each of three pulse height channels.
These channels, denoted A, B, and C, nominally span the energy ranges
1.5--3, 3--5, and 5--12~keV, respectively.  The sum of the counts in
these three bands is referred to as the ``sum band'', or ``S''.  In
this position histogram mode, the arrival time of each photon is not
preserved.  The histogram contains the result of the illumination of
the anodes through the slit mask from each discrete X-ray source in
the field of view (FOV), as well as contributions from the diffuse
X-ray background and local particle events.  For each of the 3 energy
channels, we fit these data for the intensity of each known X-ray
source by constructing a collection of vectors that we treat as a
matrix.  Each vector corresponds to a source in the FOV, according to
a catalog with positions for known X-ray sources.  Each component of
that vector represents the fractional exposure of a particular anode
bin to the corresponding source, as determined by a detailed model of
the geometry of the SSC.  A reduced $\chi^2$ fit through a matrix
inversion technique determines the amplitude of each vector.

Once an amplitude for each source in the FOV is determined, it is
converted into a measurement of the source's intensity in ``ASM
units'' through division by the exposure time and the application of a
multiplicative correction factor.  This time-dependent factor ($\equiv
a$) is empirically-determined such that the corrected intensity of the
Crab Nebula is constant.  The corrected intensity derived from
observations of any of the three SSCs is expressed as the count rate
that would be obtained for the source, had it been at the center of
the FOV of SSC~1 at a fiducial time near the start of the mission.
This defines ``ASM units'' for any energy channel.  The corrected ASM
units for 1.0 Crab are 26.8, 23.3, 25.4, and 75.5 for the A, B, C, and
Sum channels respectively.

In addition to the position histogram data mode, the ASM records the
total number of good events detected from all sources in the FOV of
each SSC in 1/8-s bins in each of the same three energy channels.  In
this ``multiple time series'' (MTS) data mode, the physical location
of each incident photon is not preserved.

In combination, these two data modes can provide a light curve for a
highly variable source like a GRB.  We take advantage of the fact that
no known variable X-ray sources were in the FOV during the
observations of any of the GRBs presented here and assume that the
count rate from all sources other than the GRB does not vary during
each dwell. Should the FOV contain variable sources other than the GRB
or should the background rate vary, the method described here would be
invalid (see, e.g., the description of GRB~000301C, below).  The total
number of counts from the GRB source detected during one dwell of one
SSC is obtained from the standard fit to the position histogram data.
We then define the effective background level (for the GRB) in the MTS
count rate data such that the total number of counts above that level
for the appropriate 90-s interval is equal to the number of counts
inferred from the position histogram data to come from the GRB source.
The number of counts from the GRB recorded in a given time bin can be
estimated as:
\begin{equation}
c_j = n_j - t_j\ \left(\frac{N}{T} - \frac{fR}{a}\right),
\label{eq:bgcounts}
\end{equation}
where $N$ is the total number of counts detected in a given SSC energy
band during an observation with total exposure time $T$, $R$ is the
time-averaged source intensity in ASM units derived from the fit to
the position-histogram data, $f$ is the transmission fraction for the
location of the GRB in the FOV, $t_j$ is the time bin size in seconds,
and $n_j$ is the number of counts in the $j$th time bin.  One can then
convert $c_j$ into ASM units through multiplication by $a/f$.  For
additional details see~\citet{smithe99}.

The ASM count rates are compared with available contemporaneous BATSE,
\uly, Konus, or \sax~GRBM count rates.  The BATSE count rates 

\end{multicols}
\begin{deluxetable}{lccrcc}
\tablecaption{Properties of 16 ASM-detected GRBs \label{flutab}}
\tablehead{
Date of GRB &
Time of GRB &
Confirming &
2-12 keV Fluence & 
R.A. &
Decl. \\
(yymmdd) & 
(hh:mm:ss) &
Satellite\tablenotemark{a}  &
($10^{-7}$ ergs cm$^{-2}$) & 
(J2000) &
(J2000) 
} 

\startdata
960416 & 04:09:00 & ubk  & $  6.0\pm0.3$ & 04h15m27s & $+77\arcdeg10\arcmin$ \\
960529 & 05:34:34 &  k   & $>17.5\pm0.6$ & 02h21m50s & $+83\arcdeg24\arcmin$ \\
960727 & 11:57:36 & uk   & $  9.5\pm0.5$ & 03h36m36s & $+27\arcdeg26\arcmin$ \\
961002 & 20:53:55 & uk   & $  9.2\pm0.5$ & 05h34m46s & $-16\arcdeg44\arcmin$ \\
961019 & 21:08:11 & ubk  & $  4.6\pm0.6$ & 22h49m00s & $-80\arcdeg08\arcmin$ \\
961029 & 19:05:10 & k    & $  3.3\pm0.4$ & 06h29m27s & $-41\arcdeg32\arcmin$ \\
961216 & 16:29:02 & bk   & \nodata & \nodata & \nodata \\
961230 & 02:04:52 & u    & $  1.5\pm0.3$ & 20h36m45s & $-69\arcdeg06\arcmin$ \\
970815 & 12:07:04 & ubks & $>33.3\pm0.8$ & 16h08m33s & $+81\arcdeg30\arcmin$ \\
970828 & 17:44:37 & ubk  & $>14.9\pm0.6$ & 18h08m23s & $+59\arcdeg19\arcmin$ \\
971024 & 11:33:32 & bk   & $  1.1\pm0.3$ & 18h25m00s & $+49\arcdeg27\arcmin$ \\
971214 & 23:20:41 & ubks & $  3.4\pm0.3$ & 12h04m56s & $+64\arcdeg43\arcmin$ \\
980703 & 04:22:45 & ubk  & $>18.3\pm0.8$ & 23h59m04s & $+08\arcdeg33\arcmin$ \\
981220 & 21:52:21 & uks  & $ 12.6\pm0.5$ & 03h43m38s & $+17\arcdeg13\arcmin$ \\
990308 & 05:15:07 & ubk  & $ 5.8\pm0.4$ & 12h23m11s & $+06\arcdeg44\arcmin$ \\
000301C & 06:51:39 & unk  & $ 3.8\pm0.5$ & 16h20m19s & $+29\arcdeg26\arcmin$ \\
\enddata
\tablenotetext{a}{u - \uly; b - BATSE; k - KONUS; s - \sax; n - {\it NEAR}}
\end{deluxetable}
\begin{multicols}{2}

\setcounter{figure}{0} 
\refstepcounter{figure} 
\PSbox{lc960416.eps hoffset=-15 voffset=0
hscale=56 vscale=56}{8.8cm}{11.7cm}{\\\\\small Fig. 1 -- The
time-series data for GRB~960416 in six energy channels
as recorded by both the ASM (2-s bins; 1.5--12~keV) and BATSE (1-s
bins; 25--320~keV).  The ASM light curve is the weighted average of
measurements by both SSC~1 and SSC~2.  Where possible, the ASM light
curves presented in these figures have been converted into Crab flux
units by the method described in Section~\protect{\ref{sec:lcdat}}.
\label{fig:lc960416_bat}}

\pfg
\noindent
shown here are extracted from the {\tt discsc} mode FITS files,
available through the public data {\tt ftp} archive at the Compton
Gamma-ray Observatory Science Support Center\footnote{{\tt
http://cossc.gsfc.nasa.gov/}}.  These data are recorded in four energy
channels with nominal energy ranges of 25--60, 60--110, 110--320, and
$>320$~keV.  In one case (GRB~970828), the event occurred during a
telemetry gap, so no {\tt discsc} data were available.  In this case,
we used the 16-channel {\tt MER} data mode, and combined channels to
approximately the same ranges as the {\tt discsc} mode.  BATSE records
counts in 64, 256, and 1024~ms bins until the on-board software
recognizes a trigger event (when the number of counts in a single bin
is a significant excess above the running average).  Then the bin size
is reduced to 16~ms for 242~s, and an alert with a rough position is
telemetered to Earth immediately.  The counts recorded with the high
time resolution are saved on-board the spacecraft and sent down at the
next convenient time.  The hard X-ray detectors for the solar
X-ray/cosmic gamma-ray burst experiment on \uly~record 25--150~keV
photons with a temporal bin size of 0.25-s~\citep{hsabc92}.
The high energy light 

\refstepcounter{figure} 
\PSbox{lc960529_kon.eps hoffset=-15 voffset=0
hscale=56 vscale=56}{8.8cm}{11.7cm}{\\\\\small Fig. 2 -- The
time-series data for ASM and Konus observations of GRB~960529.  The
ASM light curve is the weighted average of measurements by both SSC~1
and SSC~2, in 5-s bins.  A gap indicates the 6~s interval between
dwells when the ASM assembly was in motion.  The Konus count rates are
from the ``waiting-mode'' data in 3-s bins, as this event did not
cause a trigger.\label{fig:lc960529}}

\pfg
\noindent
curve for GRB~000301C presented here has been rebinned to 0.5-s bins
and shifted by $+35446$ s, to match the Earth-crossing time (which is
within $\sim0.02$~s of the time at \rxte~-- much less than the bin
size on these light curves).  KONUS records data continuously in a
``waiting mode'', with bins of 2.9 s, but when a trigger is activated,
counts are recorded in higher resolution for a certain amount of time
after the trigger, such that higher resolution data are preserved for
less time~\citep{afgim95}.  The triggered data presented in this paper
have been binned up to 256~ms.  When the \sax~GRBM registers a burst,
the 40--700~keV count rate is stored in 0.48~ms bins from 8 s before
to 98 s after the trigger time; all on-board data is sent down once
every $\sim90$-m orbit for operators to
examine~\citep{fcffn97,ffcda97}.

\section{OBSERVATIONS}\label{sec:lcs}

\citet{sblr99} describe a search through 1.5 years of archived ASM
time series data, as well as a ongoing program to scrutinize the ASM
data recorded in real time, for evidence of X-ray counterparts to
GRBs.  This section 

\refstepcounter{figure} 
\PSbox{lc960727kon.eps 
hoffset=-15 voffset=-15 hscale=56 vscale=56}{8.8cm}{11.7cm}{\\\\\small
Fig. 3 -- The ASM time-series data from SSC~2 for GRB~960727 (2-s bins) in
three energy bands, as well as the corresponding count rate histories
from Konus (0.25-s bins after the trigger) in three energy bands.
\label{fig:lc960727}}

\pfg
\noindent
presents the ASM light curves for the sixteen GRBs discovered as part
of those searches.  We will identify the major features of the light
curves and describe relevant details of the observations that will
clarify the results.  We will discuss possible interpretations of
these features in Section~\ref{sec:lcsum}.

The multiply-peaked structure of GRB~960416
(Fig.~\ref{fig:lc960416_bat}) was observed in both SSCs~1 and~2 during
a single dwell.  The BATSE light curve shows a double-peaked structure
lasting some 50~s.  The two BATSE peaks are also seen in the ASM data,
but the ASM reveals further activity.  A peak in the X-ray light curve
between the two main peaks is absent in the gamma-ray light curve.
This peak is detected in both ASM SSCs, but its spectrum is remarkably
soft.  There is no significant emission detected above 5~keV during
this intervening peak.  The ASM light curve shows an extended tail for
the final peak beyond the end of the BATSE event.

GRB 960529 (Fig.~\ref{fig:lc960529}) began around 05:34~UT.  Three
hard peaks were observed in the ASM time-series data from two SSCs
over the course of two successive dwells.  Although no Konus trigger
was explicity activated, the waiting-mode data clearly show the
multi-peak structure of this burst out to over 200~keV.  However, the
KONUS light curve contains four major peaks, such that the extended
tail of the first peak in the ASM light curve is 

\refstepcounter{figure} 
\PSbox{lc961002kon.eps 
hoffset=-15 voffset=-15 hscale=56 vscale=56}{8.8cm}{11.7cm}{\\\\\small
Fig. 4 -- The ASM time-series data from SSC~2 for GRB~961002 (2-s bins) in
three energy bands, as well as the corresponding count rate histories
from Konus (0.25-s bins after the trigger) in three energy bands.
\label{fig:lc961002}}

\pfg
\noindent
resolved into two distinct peaks at higher energies.

GRB~960727 and GRB~961002 (Figs.~\ref{fig:lc960727}
and~\ref{fig:lc961002}) were each detected only in SSC~2, each lasted
about 30~s, and each showed a singly-peaked soft X-ray light curve
without strongly significant structure on smaller time scales.
Neither burst was detected by BATSE, but each was detected by KONUS
and the GRB detector on \uly.  The high-energy light curves reveal
rich temporal structure with no corresponding X-ray variability.  Each
event seems to conclude with a weak, extended X-ray tail, absent at
high energies, that lasts for 10 or 20~s.  The accuracy of the
background estimation before the burst supports the existence of a
post-burst excess, but the extended tail is too weak in the three
sub-bands to make a useful measurement of the energy-dependence of the
decay rates.

GRB~961019 (Fig.~\ref{fig:lc961019_bat}) was detected in a single
observation of SSC~2 and was also observed with BATSE.  The BATSE
light curve shows three sub-peaks.  The GRB was only
$0.5\pm0.2\arcdeg$ from the edge of the SSC FOV, implying that only
24\% of the detector surface was exposed to the source.  The X-ray
peak emission is delayed relative to the gamma-ray maximum by 5--10~s.

GRB~961029 (not shown) was detected as a dramatic 


\refstepcounter{figure} 
\PSbox{lc961019.eps 
hoffset=-20 voffset=-10 hscale=54 vscale=54}{8.8cm}{8.1cm}{\\\\\small
Fig. 5 -- The time-series data from SSC~2 for GRB~961019 in both the
ASM (5-s bins; 1.5--12~keV) and BATSE (0.64-s bins; 25--320~keV).
\label{fig:lc961019_bat}}

\pfg

\refstepcounter{figure} 
\PSbox{lc961216.eps
hoffset=-20 voffset=-10 hscale=54 vscale=54}{8.8cm}{8cm}{\\\\\small
Fig. 6 -- The time-series data from SSC~2 for GRB~961216 in both the
ASM (5-s bins; 1.5--12~keV) and BATSE (1-s bins; $>25$~keV).  No
background subtraction has been performed on these data.
\label{fig:lc961216_bat}}

\pfg

\noindent
rise in count rate only a few seconds from the end of a dwell.  During
this dwell, the source was located only
$2\arcdeg.0\pm0.2$~\citep{sblr99} from the edge of the FOV of SSC~2.
As SSC~2 was rotated after the end of the dwell, the field of view
moved off the direction to the GRB source and the signal was lost.
KONUS reported a burst detection at 19:05:10 (UTC), which is during
the rise of the ASM event.  No other high-energy GRB detector observed
this event.

GRB~961216 (Fig.~\ref{fig:lc961216_bat}) was detected by a single SSC.
The 


\refstepcounter{figure} 
\PSbox{lc1230com.eps
hoffset=-20 voffset=-15 hscale=59 vscale=59}{8.8cm}{5.1cm}{\\\\\small
Fig. 7 -- The time-series data for GRB~961230 (9-s bins; 1.5--12~keV),
displaying the weighted average of measurements by both SSC~1 and
SSC~2.\label{fig:lc1230com}}

\pfg


\noindent
location of GRB~961216 lay only $\sim1\arcdeg$ from the edge of the
FOV, which is outside the region for which our position-determining
ability is well-calibrated~\citep{sblr99}.  Without an accurate
position, it is impossible to accurately estimate the number of counts
in the position histogram mode data that can be attributed to the
source and to thereby estimate the effective background count rate.
The 1.5--12~keV count rate as measured by the entire SSC, without
background subtraction, is shown in Figure~\ref{fig:lc961216_bat}.
The light curve suggests that there was detectable X-ray flux from the
GRB that both led and trailed the gamma-ray activity as measured by
BATSE.

GRB~961230 (Fig.~\ref{fig:lc1230com}) was a weak burst that was
nonetheless detected in both SSC~1 and SSC~2 during the same dwell.
The X-ray flux lasted about 25~s and reached a peak of
$0.23\pm0.03$~Crab.  This burst was detected by the GRB detector on
\uly~at 02:04:52 (UTC), but it was not observed by any other
instrument.

GRB~970815 (Fig.~\ref{fig:lc970815_all}) exhibited multiple peaks over
an interval of several minutes.  SSC~2 scanned onto the source during
the decay from an initial peak.  The decay timescale from this peak is
clearly longer at lower energies, across all six available energy
channels, indicative of the spectral softening common to the early
phases of GRB decay curves, but neither the onset nor the peak was
observed by the ASM.  A second peak began 70~s into the dwell, and a
third peak $\sim50$~s after that, during the following dwell.  The
third 90-s dwell, in which the burst source was $\sim0.75\arcdeg$ from
the edge of the FOV of SSC~1, yielded a $2\sigma$ upper limit of
40~mCrab (2--12~keV).

It is striking that the third peak, the strongest in the ASM light
curve, barely registers in BATSE's two lowest energy channels
(25--110~keV).  The X-ray spectrum of the burst evolves rapidly during
this peak.  It shows a distinct soft lag of $\sim12$~s between the
times of peak burst flux between the high and low energy channels of
the ASM.  The ASM spectra for the two maxima clearly differ.  Assume a
simple power law spectrum ($N\propto E^{-s}$) without absorption in
the 2--12~keV band, to find the index $s$ for the third peak is
$1.8\pm0.1$ while only $1.2\pm0.3$ for the second peak.  These fits
were achieved from a 5-s bin beginning 99~s after the BATSE trigger
time and a 2-s bin beginning at 154~s, respectively, via a comparison
with the Crab count rates in each ASM energy band.

\refstepcounter{figure} 
\PSbox{lc970815.eps
hoffset=-20 voffset=-20 hscale=54 vscale=54}{8.8cm}{11.65cm}{\\\\\small
Fig. 8 -- Light curves for GRB~970815 as measured by both the ASM and
BATSE.  The ASM scanned SSC~2 onto the GRB location during the decay
of the first peak.  Gaps in the ASM light curve indicate the 6~s
intervals between dwells when the ASM assembly was in motion.  The
first two dwells are graphed in 2-s bins.  The second dwell is
represented here by the weighted average of SSCs~1 and~2.  The GRB is
very dim during the third dwell, and although the data from SSC~1 have
been binned into 9-s bins, the flux in each bin is consistent with
zero.  The 90-s averaged flux from the GRB in this dwell is
$30\pm20$~mCrab (1.5--12~keV).  The BATSE light curve is presented in
1-s bins (25--320~keV).\label{fig:lc970815_all}}

\pfg

GRB~970828 (Fig.~\ref{fig:lc970828_4}) was a bright burst detected in
both SSC~1 and 2~\citep{rwsl97}.  The burst onset was observed midway
through a 90-s dwell.  Its FOV location was such that it was only
observed by SSC~1.  The ASM drive assembly rotated the SSCs between
dwells while the burst was still active, and the new aspect placed the
burst's FOV location just $0\arcdeg.5$ inside the edge of the 

FOV of SSC~2.  Despite the reduction in effective area, the counting
rate during the second dwell yields a clear detection of the GRB in at
least two of the three energy channels.  A second rotation brought the
source to the center of the FOV of SSC~2, in time to witness the final
decay of the event.  Spectral softening during this decay is apparent.

At the onset, the X-ray flux climbs more slowly than the gamma-ray
flux, but the 5-12 keV structure appears to echo the gamma rays.  The
time of peak emission may lag at lower X-ray energies.  After 40-s of
this initial, bright event, the burst morphology as seen by BATSE
diverges

\refstepcounter{figure} 
\PSbox{lc970828_all.eps
hoffset=-30 voffset=-15 hscale=58 vscale=58}{8.8cm}{13.9cm}{\\\\\small
Fig. 9 -- The ASM time-series data in 2-s and 9-s bins for GRB~970828,
compared with data from BATSE.  The burst was first observed with
SSC~1, and with SSC~2 during the second and third dwells.
\label{fig:lc970828_4}}

\pfg
\noindent
rapidly from that recorded by the ASM.  The high-energy flux drops to
zero, then flashes through at least five further peaks of emission.
The burst is completely finished by 150~s after the trigger.

GRB~971024 (Fig.~\ref{fig:lc971024_bat}) was an extremely weak burst
in all energy bands.  It was detected in both SSCs~1 and 2.  Large
systematic uncertainties in the estimation of the source flux stem
from relatively large uncertainties in the source
position~\citep{sblr99}.  The detection with SSC~2 is so weak that
only an average flux over the full 90~s can be derived from this
observation.  The BATSE light curves show obvious spectral softening
over the $\sim110$~s of the decay, but the ASM light curve is too
short and too weak for fruitful comparison.

GRB~971214 (Fig.~\ref{fig:lc971214_bat}) was a fairly bright,
singly-peaked event, lasting $\sim40$~s, observed with SSC~3 during a
single dwell.  Limited statistics do not allow us to probe the soft
X-ray light curve for counterparts to the complex structures in the
BATSE hard X-ray data, e.g. the sharp spike at $\sim32$~s.  As with
GRB~970828, the duration of the X-ray event is longer than its
gamma-ray counterpart.

\refstepcounter{figure} 
\PSbox{lc971024.eps
hoffset=-25 voffset=-15 hscale=55 vscale=55}{8.8cm}{8.2cm}{\\\\\small
Fig. 10 -- The time-series data for GRB~971024 in both the ASM SSC~1
(9-s bins; 1.5--12~keV) and three BATSE energy bands (3.8-s bins;
25--320~keV).\label{fig:lc971024_bat}}

\pfg

\refstepcounter{figure} 
\PSbox{lc971214.eps
hoffset=-25 voffset=-15 hscale=55 vscale=55}{8.8cm}{8.2cm}{\\\\\small
Fig. 11 -- The time-series data for GRB~971214 in both the ASM SSC~3
(5-s bins; 1.5--12~keV) and a single BATSE energy band (0.64-s bins;
60--110~keV).
\label{fig:lc971214_bat}}

\pfg

GRB~980703 (Fig.~\ref{fig:lc980703_all}) was detected in the FOVs of
both SSCs~1 and 2 simultaneously.  The flux rose over $\sim30$~s to
reach a maximum measured value at the end of a 90-s dwell.  The hard
X-ray maximum, as measured by BATSE, leads the the soft X-ray maximum
by at least 6--8~s.  The onset of the burst at $t \sim -20$~s,
relative to the BATSE trigger time, seems the same in all energy
bands, although the variability in the rise seems to decrease with
energy.  During the second dwell, the GRB source is only $0\arcdeg.6$
from the edge of the FOV of SSC~1 and out of the FOV of SSC~2.  

\refstepcounter{figure} 
\PSbox{lc980703.eps
hoffset=-25 voffset=-10 hscale=55 vscale=55}{8.8cm}{14.5cm}{\\\\\small
Fig. 12 -- The time-series data for GRB~980703 in both the ASM SSC~1
(2-s and 9-s bins; 1.5--12~keV) and four BATSE energy channels (0.64-s
bins after the trigger, 1-s bins before the trigger; $>25$~keV).
\label{fig:lc980703_all}}


\pfg

\noindent
Both BATSE and the ASM detect a lengthy tail.  Since the transition
from burst to tail in BATSE occured while the ASM was in motion, we
have no information on the X-ray properties of this transition.

GRB~981220 (Fig.~\ref{fig:lc981220_both}) was observed in SSC~2 near
the end of a 90-s dwell.  This is the brightest GRB yet observed in
the ASM data, reaching a flux of over 5~Crab in the Sum band
(1.5--12~keV).  Although this event was not observed by BATSE, it was
observed in the GRBM on board \sax.  The X-ray flux rises earlier than
and declines after the gamma-ray event, and the spectrum softens
during the decay.

GRB~990308 (Fig.~\ref{fig:990308}) was observed in SSC~3 alone. An
optical transient is associated with this burst~\citep{schaf99}, so
the source position is accurately known.  BATSE also detected this
burst, and its count rate reveals a multiply-peaked light curve, some
35~s in duration.  The soft X-ray light curve does not appear to last
significantly longer than the hard X-ray light curve.

\refstepcounter{figure} 
\PSbox{lc981220_4ch.eps
hoffset=-25 voffset=-10 hscale=55 vscale=55}{8.8cm}{8.2cm}{\\\\\small
Fig. 13 -- The bottom panel shows the time-series data (1.5--12~keV) for
the ASM SSC~2 observation of GRB~981220 in 2~s bins, while the top
panel shows the count rate in the \sax~GRB Monitor.  The spike at 20~s
is an artifact.\label{fig:lc981220_both}}

\pfg

\refstepcounter{figure} 
\PSbox{lc990308.eps hoffset=-18 voffset=-10
hscale=51 vscale=51}{8.8cm}{11.8cm}{\\\\\small
Fig. 14 -- The time-series data for GRB~990308 in both the ASM SSC~3
(5-s bins; 1.5--12~keV) and BATSE (1-s bins; 25--300~keV).
\label{fig:990308}}
 
\pfg

\refstepcounter{figure} 
\PSbox{lc000301c_u.eps
hoffset=-30 voffset=-7 hscale=55 vscale=55}{8.8cm}{8cm}{\\\\\small
Fig. 15 -- GRB 000301C as observed by the ASM SSC~2 (1-s bins;
1.5--12~keV) and the GRB detector on \uly~(0.5-s bins; 22--150~keV).
The rising count rate in the 1.5-3~keV channel indicates that the
\rxte~may be moving into a region of high activity in the Earth's
magnetosphere, or there may be interference from scattered Solar
X-rays.\label{fig:000301c}}

\pfg


GRB~000301C (Fig.~\ref{fig:000301c}) was observed in SSC~2.  It
consisted of a single peak that reached over 3~Crab in the 5--12~keV
band.  No emission was detected in the 1.5--3~keV band, with a
normalized $2\sigma$ upper limit of 90~mCrab (integrated over 90-s).
This burst was also detected by \uly~and {\it NEAR}~\citep{shc00}.
The Earth lay between BATSE and the GRB at the time of the event
(M. Kippen, private communication), and the \sax~GRBM was powered down
(M. Feroci, private communication).

The bottom panel of Figure~\ref{fig:000301c} illustrates the
limitations of the background-subtraction method described above in
section~\ref{sec:lcdat}.  Scattered solar X-rays can cause a response
in the A~channel (1.5--3~keV) such as the rising count rate evident in
the figure.  Thus we cannot derive a 1.5--3~keV light curve for this
burst.  The rising count rate is only evident in the A~channel, and we
conclude that the light curves shown for the B~and C~channels are
accurate.

\section{DISCUSSION}\label{sec:lcsum}

The external shock model has proven popular for explaining the major
features of most GRB afterglow behavior, and the internal shock model
has become the favored explanation for multiple peaks and temporal
structure of the GRB itself.  A clear observational distinction
between GRB event and afterglow remains elusive, although several
candidates have been proposed~\citep{ch98,gpkcw99,zhpf99}.  In this
section, we discuss the implications of these light curves for burst
origins in the context of the simple model outlined in
section~\ref{sec:grbmod}.  We discuss whether or not the observations
are consistent with an origin in synchrotron radiation, and in three
cases, we address the question of whether or not the ASM has observed
the onset of the X-ray afterglow.  We 

\end{multicols}

\refstepcounter{figure} 
\PSbox{ag970828.eps 
hoffset=25 voffset=-15 hscale=100 vscale=100}{16cm}{9cm}{\\\\\small 
Fig. 16 -- The 2--10~keV flux history of GRB~970828, as measured by
the ASM, the PCA~\protect{\citep{mcc97}},
\asca~\protect{\citep{muykm97}}, and \rosat~\protect{\citep{gsgg97}}.
The dashed line shows the best-fit power-law decay curve for the ASM
data, $F\propto t^{-5}$.  The dotted line shows the best-fit power-law
for the three late-time flux measurements, $F\propto t^{-0.5}$.  The
ROSAT flux is derived by extrapolating the reported 0.5--2.4~keV
spectrum out to 10~keV.  If this extrapolation is excluded from
consideration, the PCA and \asca~data are consistent with a power law
decay of index 1.4~\protect{\citep{ynokm98}}, shown here as a broken
line.\label{fig:decay}}

\begin{multicols}{2}

\noindent
close by summarizing some general properties of this data set and
noting the absence of short GRBs.

The sixteen GRB light curves presented here are diverse in form.  Very
few of these bursts appear as simple fast-rise, exponential-decay
shapes that one might expect from impulsive events.  Several bursts
show a near-symmetric single-peak (e.g. Fig.~\ref{fig:lc960727}
and~\ref{fig:000301c}), with and without significant structure on
smaller time scales (Fig.~\ref{fig:lc981220_both}).  Two bursts
(Fig.~\ref{fig:lc961002} and~\ref{fig:lc961216_bat}) even seem to have
a slow-rise, fast-decay structure!  About half the bursts show
multiple, distinct peaks, while three (Figs.~\ref{fig:lc970828_4},
\ref{fig:lc971214_bat}, and~\ref{fig:990308}) have a veritable forest
of peaks within their gamma-ray light curves.


Although each of these bursts shares some characteristics across the
entire observed energy range, there are other features that are only
detected in a few, or even one, energy channels.  The second peak of
GRB~960529 is harder than the peak immediately preceding it, such that
at low energies, it is smeared out and difficult to recognize as a
distinct event.  This smearing at low energies is common, and can be
seen in the first peak of GRB~980703 as well as GRB~960727,
GRB~961002, and GRB~981220.  In contrast, GRB~960416 shows an
intermediate peak unique to the softest ASM energy channel.  According
to Equation~\ref{eq:peaklimgam}, if a burst's peak energy lies below
5~keV, the maximum Lorenz factor of the expanding shock wave would be
only of order 20, assuming that the other factors are of order unity.
This would be considered rather small for a GRB, albiet still highly
relatvistic.

GRB~970815 (Fig.~\ref{fig:lc970815_all}) also shows multiple peaks,
but in this case only the last peak is unusually soft.  An intriguing
possibility is that the third peak is due to an external shock, and
hence represents the beginning of the afterglow, while the first two
peaks originate in internal shocks that occur before the outermost
ejecta sweep up enough matter to instigate the external shock.  It has
been predicted that the afterglow could begin tens of seconds after
the burst~\citep{sari97,sp99}.  We will therefore address this
possibility here.

If the third peak in GRB~970815 does represent the onset of the
afterglow due to an external shock, the soft lag would be the result
of the decay of $\nu_m$ (see Eq.~\ref{eq:peaklimgam}), which is
predicted by the external shock model of~\citet{mr97} to fall as
$t^{-2/3}$, where $t$ is measured by a distant observer at rest with
respect to the blast center.  There is a $\sim8$~s difference between
the times when the third peak reached its maximum in the C band
($E\sim7$~keV) and the A band ($E\sim2.25$~keV) at $t \sim 154$~s and
162~s, respectively.  If this delay is due to the evolution of
$\nu_m$, the shock that generated the third peak must have begun
cooling at $t_0 \sim 152$~s after the BATSE trigger time.  

The decay curve of the third peak can be fit with a power-law model,
such that $F(t) \propto (t-t_0)^{-\alpha}$.  All three ASM energy
bands show a decay from the third peak consistent with $\alpha =
1.3\pm0.1$.  This achromatic decay is in marked contrast to the decay
from GRB~970828, as described below, but it is consistent with the
afterglow observations from other GRBs and the predictions of the
external shock model~\citep{mr97}.  X-ray afterglow curves have been
measured from twenty-three GRBs prior to 1999 August, and the
power-law indices for the decay range from 1.1 for
GRB~970508~\citep{paabc98} to 1.57 for GRB~970402~\citep{naabc98}.
The decay from GRB~970815 is thus fully consistent with an
afterglow-type decay.  If one extrapolates 


\end{multicols}
\refstepcounter{figure} 
\PSbox{ag980703.eps
hoffset=30 voffset=-15 hscale=95 vscale=95}{16cm}{9.2cm}{\\\\\small
Fig. 16 -- The 2--10~keV flux history of GRB~980703, as measured by the ASM and
the \sax~NFI~\protect{\citep{vgoog99}}.  The ``T'' shape indicates an
upper limit.  The dashed line shows the best-fit power-law decay curve
combining both instruments, $F\propto t^{-1.3}$.\label{fig:ag0703}}
\begin{multicols}{2}

\noindent
this decay to the time of the \asca~follow-up observation,
$\sim3.5\times10^5$~s after $t_0$, the predicted 2--10~keV flux of
about $8\times10^{-15}$~ergs~cm$^{-2}$~s$^{-1}$ lies below the
\asca~upper limit of
$10^{-13}$~ergs~cm$^{-2}$~s$^{-1}$~\citep{muify97}.  The lack of an
\asca~detection does not rule out the possibility that the third peak
is the start of an afterglow decay.

An alternative scenario is that both the second and third peaks result
from shells catching up with the decelerating shell that produced the
first shock.  By 150~s, the shell that produced the first peak would
have decelerated significantly, for reasonable values of the ambient
density.  The two later shells would shock this decelerating fluid and
produce emission enhancements.  In this scenario, emission from any
external shock caused by the deceleration of the shell that produced
the first peak would have been too dim to be detected with the ASM.
The second and third peaks would then be early versions of the
enhancements seen late in the afterglow of other bursts, such as
GRB~970508~\citep{paabc98}, indicating that whatever processes produce
bursts continue to operate throughout the entire event.  If this
scenario is true, the afterglow and the burst cannot always be
considered distinct events.  It is possible, however, that the late
pulses in the afterglow are not due to collisions from behind, as
interpreted by~\citet{paabc98}, but instead are isolated instances of
the remnant colliding with a dense patch of external medium.

GRB~970828 (Fig.~\ref{fig:lc970828_4}) displays an extended interval
of X-ray emission beyond the cessation of gamma-ray activity.  As in
the case of GRB~970815, we can ask if the ASM X-ray light curve
reveals the onset of the X-ray afterglow.  In this interpretation, the
short bursts in the BATSE data would originate in internal shocks
superimposed on the longer process of an external shock.  The final
X-ray decay should then share temporal and spectral properties with
X-ray afterglow observed at later times.  A power law decay curve,
with the origin set at the BATSE trigger time, when fit to the
1.5--12~keV band data in the last ASM dwell, indicates that the flux
decays as roughly $t^{-5}$.  All observed afterglow decay curves have
indices around between 1.1~and 1.5, and the X-ray decay from this GRB
was measured by the PCA and \asca~over the following two days to decay
as $t^{-1.4}$~\citep{muykm97}, as shown in Figure~\ref{fig:decay}.
There is nothing in the theory of external shocks to explain a decay
index around 5 that later changes to 1.4.  We therefore find this
scenario unlikely.  It is more likely that the electrons that generate
the soft X-ray flux from the initial shock did not have time to cool
before they were shocked by collisions from behind by additional
shells of ejecta; collisions revealed by the sharp peaks in the
high-energy light curve.  The afterglow emission would then be from a
separate, external shock, too weak to be detected with the ASM.

A fading X-ray afterglow was associated with GRB~980703 through
observations with the \sax~Narrow Field Instruments (NFI) 22~h after
the event~\citep{vgoog99}, and in contrast to GRB~970828, the tail of
the ASM decay curve is consistent with an extrapolation back from the
\sax~flux measurements~(Fig.~\ref{fig:ag0703}).  The best-fit
power-law decay curve derived using both the ASM decay and the later
\sax~measurements has a time index of 1.3, a typical value for GRB
afterglows.  This value is consistent with the lower limit of 0.9
measured using only the NFI observations~\citep{vgoog99}.  The tail of
the burst emission may therefore represent a transition to the
afterglow.  It is worth noting that the BATSE count rates reveal a
second interval of emission from this burst, roughly 300~s after the
onset of the event, at which time the burst was outside the ASM FOV.
The apparent connection between the decay from the first peak and the
\sax~afterglow may be a coincidence.  It could also be an artifact of
the large error bars in the ASM decay curve.  If the ASM decay is the
start of the afterglow, the shock front that created the second BATSE
outburst must not have disrupted the cooling external shock enough to
distort the measured X-ray decay curve away from a single power law.

Even beyond these three examples, the X-ray flux from GRB events tends
to be less variable and in almost all cases lasts longer than the
associated gamma-ray flux.  The degree to which variability can be
measured is often limited by counting statistics, but the general
result is often understood within the context of synchrotron
radiation theory.  The cooling timescale for an electron emitting
synchrotron radiation is inversely proportional to the electron's
Lorentz factor ($t_c \propto 1/\Gamma$).  The synchrotron frequency of
an emitting electron (and the characteristic photon energy $E$ of the
emitted radiation) goes as the square of its Lorentz factor ($E
\propto \Gamma^2$).  Hence, if a population of electrons is cooling
through synchrotron radiation, one would expect to find the cooling
time scales to vary as $E^{-1/2}$~\citep{rl79,piran99,wg99}.

Although geometric effects can extend the time over which a burst is
visible to a distant observer, the relationship between the cooling
time scale and the width of observed peaks in a GRB light curve is
expected to be preserved~\citep{piran99}.  If GRBs are produced by
synchrotron radiation, then, one would expect individual peaks to be
wider in lower energy bands, in accordance with $E^{-1/2}$.  The fact
that the bursts presented here were observed serendipitously by
different instruments renders the width of the GRB peaks difficult to
measure in many cases.  Nevertheless, we have measured peak widths as
a function of energy band for seven bursts, and we display the results
in Figure~\ref{fig:widvse}.  These widths are defined as the $2\sigma$
confidence interval for the width of a best-fit Gaussian function.  We
also measured an exponential decay timescale with equivalent results.
For bursts with complex temporal structure (e.g. GRB~970828 above
25~keV), a Gaussian function is an extremely poor match to the shape
of the light curve, but we cite the best-fit Gaussian in order to
compare duration consistently with the simpler bursts.

Figure~\ref{fig:widvse} shows that only the simplest bursts are
consistent with the prediction of synchrotron cooling.  GRB~000301C
was a single-peaked Gaussian shape in both the ASM and \uly~data
(Fig.~\ref{fig:widvse}f).  A possible power-law solution with index of
$-0.5$ is shown as a dashed line.  While GRB~960416
(Fig.~\ref{fig:lc960416_bat}) displayed multiple peaks, these peaks
were widely separated, and the width of each peak individually is
consistent with a $E^{-1/2}$ scaling law (Fig.~\ref{fig:widvse}a \&
b).  Again, the dashed line in the figure shows a representative power
law with index $-0.5$.

Such a simple model, however, does not fit any of the other bursts.
GRB~960727 (Fig.~\ref{fig:lc960727}) and GRB~961002
(Fig.~\ref{fig:lc961002}) seem simple as recorded by the ASM, but the
KONUS light curves reveal complex temporal structure with multiple
peaks.  The total duration of both these bursts has a much flatter
dependence on energy than synchrotron cooling would predict; the
best-fit power law indices are inconsistent with a slope of $-0.5$
(Compare the broken and dashed lines in Figures~\ref{fig:widvse}c
and~d).  GRB~981220 also displayed a simple, single-peaked structure
in the ASM time-series data, but the \sax~count rates reveal three
smaller peaks at high energies (Fig~\ref{fig:lc981220_both}).  The ASM
data alone are consistent with a power law index of $-0.5\pm0.1$, but
the extrapolation of this scaling law to the energy range of the
\sax~GRBM is inconsistent with the measured width of the peak as
modeled by a single Gaussian (the solid box spanning the 40--700~keV
range in Figure~\ref{fig:widvse}e).  However, if the GRBM light curve
is modeled by the superposition of three Gaussian events, then their
widths are (in chronological order) $0.35\pm0.1$~s, $0.6\pm0.1$~s, and
$1.2\pm0.2$~s (shown as the dotted, dashed, and broken boxes in
Figure~\ref{fig:widvse}e, respectively).  The two latter short peaks
are consistent with the extrapolated power-law, and perhaps the
emission seen in soft X-rays is dominated by the cooling of a single
shock that produced one of these high-energy peaks.

The widths of the ASM light curves for GRB~970828
(Fig.~\ref{fig:widvse}g) and GRB~971214 (Fig.~\ref{fig:widvse}h) do
not match well with an extrapolation of the primary peak width as
observed by BATSE.  In both cases, the BATSE light curves showed
multiple peaks of emission, and a Gaussian model of the primary peak
widens much more slowly than a $E^{-0.5}$ power law would predict.  In
neither of these cases is any evidence for the smaller peaks apparent
in the ASM data, although in both cases, small fluctuations in the ASM
count rate are difficult to identify.  Within the context of the
internal shock model, the extended length of the ASM light curve for
GRBs~970828 and 971214 can be interpreted as the reheating of the
matter heated by the initial shock, which does not have time to cool
in the X-ray regime before it is shocked again.  This scenario,
however, would not explain the smaller indices for GRBs~960727 and
961002.

In short, only the simplest bursts display the $E^{-0.5}$ dependence
of width on energy predicted by synchrotron cooling.  I would also
point out that the only other burst for which this hypothesis has been
tested across X-ray and gamma-ray bands, GRB~960720, was also a
singly-peaked burst~\citep{phjcf98}.  A likely explanation of this
discrepancy is that multiple peaks are indicative of complex
interactions that violate the assumption of a single infusion of
energy followed by cooling through radiation.  It is unclear why some
complex bursts lead to abnormally long soft X-ray light curves (such
as GRB~970828), while others (like GRB~960727) show a much weaker
dependence on energy.  It is possible that individual peaks in these
complex bursts do behave consistently with the predictions of
synchrotron cooling, as the well-separated peaks in GRB~960416 do, and
the short peaks in GRB~981220 might, but most often, the statistics
and time resolution of the ASM data do not allow us to track the
behavior of individual short peaks.

When study of indivdual peaks is possible, however, several show a
soft lag in their times of maximum count rate.  It is worth noting
that several do not, at least within the limitations of the effective
time resolution.  In at least two cases, the X-rays can be seen to
rise before the onset of the GRB in the higher-energy regime.  The
first peak of GRB~960416 seems to begin earlier at $\sim5$~keV than
above 25~keV, and GRB~981220 may be active in the ASM before the count
rate in the \sax~GRBM begins to rise.  It's possible that the X-ray
count rates precede the 


\end{multicols}
\refstepcounter{figure} 
\PSbox{eightwid.eps
hoffset=60 voffset=-75 hscale=98 vscale=98}{16cm}{22.5cm}{\\\\\small
Fig. 17 -- Peak width vs. Energy for seven GRBs.  The width of the GRB
is modeled with a simple Gaussian function, and the $2\sigma$
confidence interval for $\sigma$ is plotted here on the $y$-axis,
against the appropriate energy channel on the $x$-axis (using data
from BATSE, the \sax~GRBM, \uly, Konus, and the ASM).  Details are
explained in the text.\label{fig:widvse}}

\begin{multicols}{2}

\noindent
gamma-rays in GRB~961216.  GRB~980703 is ambiguous.  The onset of
X-ray emission is concurrent with the pre-trigger emission in the
BATSE count rate, but continues smoothly past the time of maximum
gamma-ray flux.  In no case, however, did we observe any candidate for
a distinct X-ray precursor, such as that associated with
GRB~980519~\citep{zhpf99} or perhaps GRB~900126~\citep{minpf91}.
Precursor events are very rare, so their absence in the ASM sample is
unsurprising.

What is perhaps more surprising is the absence of GRBs shorter than
10~s in duration.  The GRBs in the BATSE catalog have a well-known
bimodal duration distribution with peaks at 0.1~s and
10~s~\citep{kmfbb93,fmwbh94,kpkpp96}.  All the GRBs presented here are
drawn from the longer sub-population, although GRB~000301C is a
borderline case~\citep{jfghh00}.  If the typical peak intensities of
the short bursts are of the same magnitude or less as those of the
long bursts, the ASM is less likely to detect the short bursts,
although a short burst should still stand out in the time-series data.
Our variability search was conducted on time-scales of 1/8-s, 1-s, and
9-s, and the short bursts from SGR~1627--41~\citep{sbl99} were
detected.  The population of short bursts represents roughly 25\% of
the first BATSE GRB catalog~\citep{kmfbb93}, so perhaps the absence of
short GRB events in the ASM sample is simply a statistical
fluctuation, but it is noteworthy that all of the bursts localized to
date by the \sax~WFC have also been from the population of longer
bursts~\citep{gsccd00,facmp00}.  The BATSE data suggest that the
shorter bursts also have harder spectra, so perhaps they are not
bright enough in the 1.5--12~keV range for ASM detection.

\acknowledgements

This project combined results from several instruments, and hence
could not have been completed without the help of many individuals.
Of crucial help were Scott Barthelmy's work in creating and
maintaining the GCN, and Kevin Hurley's coordination of the IPN.  We
would also like to acknowledge the support of the \rxte~team at MIT
and NASA/GSFC.  Support for this work was provided in part by NASA
Contract NAS5--30612.

\newcommand{\noopsort}[1]{} \newcommand{\printfirst}[2]{#1}
  \newcommand{\singleletter}[1]{#1} \newcommand{\switchargs}[2]{#2#1}
\begin{thebibliography}{}

\bibitem[\protect\astroncite{{Aptekar} {\rm et~al.\/}}{1995}]{afgim95}
 {Aptekar}, R., et~al.
\newblock 1995, Space Science Reviews, 71, 265

\bibitem[\protect\astroncite{{Band} {\rm et~al.\/}}{1993}]{bmfsp93}
 {Band}, D., et~al.
\newblock 1993, \apj, 413, 281

\bibitem[\protect\astroncite{Blandford \& McKee}{1976}]{blmck76}
Blandford, R. \& McKee, C.
\newblock 1976, Phys. Fluids, 19, 1130

\bibitem[\protect\astroncite{{Cavallo} \& {Rees}}{1978}]{cavree78}
{Cavallo}, G. \& {Rees}, M.~J.
\newblock 1978, \mnras, 183, 359

\bibitem[\protect\astroncite{{Connors} \& {Hueter}}{1998}]{ch98}
{Connors}, A. \& {Hueter}, G.~J.
\newblock 1998, \apj, 501, 307

\bibitem[\protect\astroncite{{Feroci} {\rm et~al.\/}}{1997}]{ffcda97}
 {Feroci}, M., et~al. 1997, in {SPIE Conference Proceedings},  eds. , Siegmund
  \& Gummin, Vol. 3114, 186.
\newblock (astro-ph/9708168)

\bibitem[\protect\astroncite{{Fishman} {\rm et~al.\/}}{1994}]{fmwbh94}
 {Fishman}, G.~J., et~al.
\newblock 1994, \apjs, 92, 229

\bibitem[\protect\astroncite{Frontera {\rm et~al.\/}}{1997}]{fcffn97}
Frontera, F., Costa, E., Fiume, D.~D., Feroci, M., Nicastro, L., Orlandini, M.,
  Palazzi, E., \& Zavattini, G.
\newblock 1997, \aaps, 122

\bibitem[\protect\astroncite{Frontera {\rm et~al.\/}}{2000}]{facmp00}
Frontera, F. et~al.
\newblock 2000, \apjs, submitted, astro-ph/9911228

\bibitem[\protect\astroncite{Gandolfi {\rm et~al.\/}}{2000}]{gsccd00}
Gandolfi, G. et~al. 2000, in Proc. of the Fifth Huntsville GRB Symposium,  eds.
  , R.~M. Kippen, R.~S. Mallozzi, \& G.~J. Fishman, .
\newblock in press (astro-ph/0001011)

\bibitem[\protect\astroncite{{Giblin} {\rm et~al.\/}}{1999}]{gpkcw99}
{Giblin}, T.~W., {van Paradijs}, J., {Kouveliotou}, C., {Connaughton}, V.,
  {Wijers}, R. A. M.~J., {Briggs}, M.~S., {Preece}, R.~D., \& {Fishman}, G.~J.
\newblock 1999, \apjl, 524, L47

\bibitem[\protect\astroncite{{Goodman}}{1986}]{goodm86}
{Goodman}, J.
\newblock 1986, \apjl, 308, L47

\bibitem[\protect\astroncite{Greiner {\rm et~al.\/}}{1997}]{gsgg97}
Greiner, J., Schwarz, R., Groot, P., \& Galama, T.
\newblock 1997, IAU Circ. No. 6757

\bibitem[\protect\astroncite{{Hurley} {\rm et~al.\/}}{1992}]{hsabc92}
 {Hurley}, K., et~al.
\newblock 1992, \aaps, 92, 401

\bibitem[\protect\astroncite{{In 't Zand} {\rm et~al.\/}}{1999}]{zhpf99}
{In 't Zand}, J. J.~M., {Heise}, J., {Van Paradijs}, J., \& {Fenimore}, E.~E.
\newblock 1999, \apjl, 516, L57

\bibitem[\protect\astroncite{Jensen {\rm et~al.\/}}{2000}]{jfghh00}
Jensen, B. et~al.
\newblock 2000, \aap, in press (astro-ph/0005609)

\bibitem[\protect\astroncite{Kobayashi {\rm et~al.\/}}{1997}]{kps97}
Kobayashi, S., Piran, T., \& Sari, R.
\newblock 1997, \apj, 490, 92

\bibitem[\protect\astroncite{Kobayashi {\rm et~al.\/}}{1999}]{kps99}
Kobayashi, S., Piran, T., \& Sari, R.
\newblock 1999, \apj, 513, 669

\bibitem[\protect\astroncite{{Koshut} {\rm et~al.\/}}{1996}]{kpkpp96}
{Koshut}, T.~M., {Paciesas}, W.~S., {Kouveliotou}, C., {Van Paradijs}, J.,
  {Pendleton}, G.~N., {Fishman}, G.~J., \& {Meegan}, C.~A.
\newblock 1996, \apj, 463, 570

\bibitem[\protect\astroncite{{Kouveliotou} {\rm et~al.\/}}{1993}]{kmfbb93}
{Kouveliotou}, C., {Meegan}, C.~A., {Fishman}, G.~J., {Bhat}, N.~P., {Briggs},
  M.~S., {Koshut}, T.~M., {Paciesas}, W.~S., \& {Pendleton}, G.~N.
\newblock 1993, \apjl, 413, L101

\bibitem[\protect\astroncite{{Laros} {\rm et~al.\/}}{1984}]{lefks84}
{Laros}, J.~G., {Evans}, W.~D., {Fenimore}, E.~E., {Klebesadel}, R.~W.,
  {Shulman}, S., \& {Fritz}, G.
\newblock 1984, \apj, 286, 681

\bibitem[\protect\astroncite{{Levine} {\rm et~al.\/}}{1996}]{lbcjm96}
{Levine}, A.~M., {Bradt}, H., {Cui}, W., {Jernigan}, J.~G., {Morgan}, E.~H.,
  {Remillard}, R., {Shirey}, R.~E., \& {Smith}, D.~A.
\newblock 1996, \apjl, 469, L33

\bibitem[\protect\astroncite{Marshall {\rm et~al.\/}}{1997}]{mcc97}
Marshall, F., Cannizzo, J., \& Corbet, R.
\newblock 1997, IAU Circ. No. 6727

\bibitem[\protect\astroncite{M\'{e}sz\'{a}ros \& Rees}{1992}]{mr92}
M\'{e}sz\'{a}ros, P. \& Rees, M.
\newblock 1992, \apj, 397, 570

\bibitem[\protect\astroncite{M\'{e}sz\'{a}ros \& Rees}{1997}]{mr97}
M\'{e}sz\'{a}ros, P. \& Rees, M.
\newblock 1997, \apj, 476, 232

\bibitem[\protect\astroncite{{Metzger} {\rm et~al.\/}}{1974}]{mpgpt74}
{Metzger}, A.~E., {Parker}, R.~H., {Gilman}, D., {Peterson}, L.~E., \&
  {Trombka}, J.~I.
\newblock 1974, \apjl, 194, L19

\bibitem[\protect\astroncite{{Murakami} {\rm et~al.\/}}{1989}]{mfhin89}
 {Murakami}, T., et~al.
\newblock 1989, \pasj, 41, 405

\bibitem[\protect\astroncite{{Murakami} {\rm et~al.\/}}{1991}]{minpf91}
{Murakami}, T., {Inoue}, H., {Nishimura}, J., {Van Paradijs}, J., \&
  {Fenimore}, E.~E.
\newblock 1991, \nat, 350, 592

\bibitem[\protect\astroncite{Murakami {\rm et~al.\/}}{1997a}]{muify97}
Murakami, T., Ueda, Y., Ishida, M., Fujimoto, R., Yoshida, A., \& Kawai, N.
\newblock 1997, IAU Circ. No. 6722

\bibitem[\protect\astroncite{Murakami {\rm et~al.\/}}{1997b}]{muykm97}
Murakami, T., Ueda, Y., Yoshida, A., Kawai, N., Marshall, F., Corbet, R., \&
  Takeshima, T.
\newblock 1997, IAU Circ. No. 6732

\bibitem[\protect\astroncite{Narayan {\rm et~al.\/}}{1992}]{npp92}
Narayan, R., Paczy\'{n}ski, B., \& Piran, T.
\newblock 1992, \apjl, 395, L83

\bibitem[\protect\astroncite{Nicastro {\rm et~al.\/}}{1998}]{naabc98}
Nicastro, L. et~al.
\newblock 1998, \aap, 338, L17

\bibitem[\protect\astroncite{{Ogasaka} {\rm et~al.\/}}{1991}]{omnyf91}
{Ogasaka}, Y., {Murakami}, T., {Nishimura}, J., {Yoshida}, A., \& {Fenimore},
  E.~E.
\newblock 1991, \apjl, 383, L61

\bibitem[\protect\astroncite{Piran}{1999}]{piran99}
Piran, T.
\newblock 1999, \physrep, 314(6), 575, (astro-ph/9810256)

\bibitem[\protect\astroncite{Piro {\rm et~al.\/}}{1998}]{paabc98}
Piro, L. et~al.
\newblock 1998, \aap, 331, L41

\bibitem[\protect\astroncite{{Piro} {\rm et~al.\/}}{1998}]{phjcf98}
 {Piro}, L., et~al.
\newblock 1998, \aap, 329, 906

\bibitem[\protect\astroncite{{Preece} {\rm et~al.\/}}{2000}]{pbmpp00}
{Preece}, R.~D., {Briggs}, M.~S., {Mallozzi}, R.~S., {Pendleton}, G.~N.,
  {Paciesas}, W.~S., \& {Band}, D.~L.
\newblock 2000, \apjs, 126, 19

\bibitem[\protect\astroncite{Rees \& M\'{e}sz\'{a}ros}{1992}]{rm92}
Rees, M. \& M\'{e}sz\'{a}ros, P.
\newblock 1992, \mnras, 258, 41

\bibitem[\protect\astroncite{Remillard {\rm et~al.\/}}{1997}]{rwsl97}
Remillard, R., Wood, A., Smith, D., \& Levine, A.
\newblock 1997, IAU Circ. No. 6726

\bibitem[\protect\astroncite{Rybicki \& Lightman}{1979}]{rl79}
Rybicki, G. \& Lightman, A. 1979.
\newblock Radiative {Processes in Astrophysics}.
\newblock John Wiley \& Sons, New York

\bibitem[\protect\astroncite{{Sari}}{1997}]{sari97}
{Sari}, R.
\newblock 1997, \apjl, 489, L37

\bibitem[\protect\astroncite{Sari {\rm et~al.\/}}{1996}]{snp96}
Sari, R., Narayan, R., \& Piran, T.
\newblock 1996, \apj, 473, 204

\bibitem[\protect\astroncite{{Sari} \& {Piran}}{1999}]{sp99}
{Sari}, R. \& {Piran}, T.
\newblock 1999, \apj, 520, 641

\bibitem[\protect\astroncite{Sari {\rm et~al.\/}}{1998}]{spn98}
Sari, R., Piran, T., \& Narayan, R.
\newblock 1998, \apjl, 497, L17

\bibitem[\protect\astroncite{{Sazonov} {\rm et~al.\/}}{1998}]{sstlb98}
{Sazonov}, S.~Y., {Sunyaev}, R.~A., {Terekhov}, O.~V., {Lund}, N., {Brandt},
  S., \& {Castro-Tirado}, A.~J.
\newblock 1998, \aaps, 129, 1

\bibitem[\protect\astroncite{Schaefer {\rm et~al.\/}}{1999}]{schaf99}
Schaefer, B. et~al.
\newblock 1999, \apjl, 524

\bibitem[\protect\astroncite{{Shemi} \& {Piran}}{1990}]{shpi90}
{Shemi}, A. \& {Piran}, T.
\newblock 1990, \apjl, 365, L55

\bibitem[\protect\astroncite{Smith}{1999}]{smithe99}
Smith, D.
\newblock 1999, {PhD} dissertation, Massachusetts Institute of Technology

\bibitem[\protect\astroncite{Smith {\rm et~al.\/}}{1999}]{sblr99}
 Smith, D.~A., et~al.
\newblock 1999, \apj, 526, 683

\bibitem[\protect\astroncite{{Smith}, {Bradt}, \& {Levine}}{1999}]{sbl99}
Smith, D.~A., Bradt, H.~V., \& Levine, A.~M.
\newblock 1999, \apjl, 519, L147

\bibitem[\protect\astroncite{{Smith}, {Hurley}, \& {Cline}}{2000}]{shc00}
Smith, D.~A., Hurley, K., \& Cline, T.
\newblock 2000, GCN Circ. No. 568

\bibitem[\protect\astroncite{{Strohmayer} {\rm et~al.\/}}{1998}]{sfmy98}
{Strohmayer}, T.~E., {Fenimore}, E.~E., {Murakami}, T., \& {Yoshida}, A.
\newblock 1998, \apj, 500, 873

\bibitem[\protect\astroncite{{Trombka} {\rm et~al.\/}}{1974}]{tesam74}
{Trombka}, J.~I., {Eller}, E.~L., {Schmadebeck}, R.~L., {Adler}, I., {Metzger},
  A.~E., {Gilman}, D., {Gorenstein}, P., \& {Bjorkholm}, P.
\newblock 1974, \apjl, 194, L27

\bibitem[\protect\astroncite{Vreeswijk {\rm et~al.\/}}{1999}]{vgoog99}
Vreeswijk, P. et~al.
\newblock 1999, \apjl, 523, 171

\bibitem[\protect\astroncite{{Wheaton} {\rm et~al.\/}}{1973}]{wubde73}
 {Wheaton}, W.~A., et~al.
\newblock 1973, \apjl, 185, L57

\bibitem[\protect\astroncite{{Wijers} \& {Galama}}{1999}]{wg99}
{Wijers}, R. \& {Galama}, T.
\newblock 1999, \apj, 523, 177

\bibitem[\protect\astroncite{{Yoshida} {\rm et~al.\/}}{1998}]{ynokm98}
{Yoshida}, A., {Namiki}, M., {Otani}, C., {Kawai}, N., {Murakami}, T., {Ueda},
  Y., {Shibata}, R., \& {Uno}, S. 1998, , in Fourth Huntsville Gamma-Ray Burst
  Symposium, 441

\end{thebibliography}


\end{document}
